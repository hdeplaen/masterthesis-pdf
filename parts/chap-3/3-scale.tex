\section{The SCALE-MAMBA framework}

\subsection{Security of the model}
Semi-honest model + some explanations obout the active case.


\subsubsection{Passive case}



\subsubsection{Active case}


\subsection{Key performance indicators}
Now that we know when a comparison protocol is secure in the Semi honest model, we should discuss when such protocols are considered “good”. This will not be done by defining a single attribute to be good, rather we pick key performance indicators (KPI) so we get a transparent view of the performance different protocols might display in different settings.

\subsubsection{Communication Complexity}
The total communication needed by a protocol can be measured by the number of Kilo Bytes (KB) of data which have to be send over a communication line during a protocol. Obviously, a protocol which enforces 1 GB = 10242 KB of communication is not considered a good protocol when compared to another protocol which only needs 100 KB of communication data. Usually, the communication performance of a protocol depends greatly on the bit-length of the input. So in order to keep the performance evaluation fair, one should make sure to optimize the total communication as good as possible considering a given input length ? for the protocol.

\subsubsection{Round Complexity}
In many applications, the number of communication rounds tends to be the bottle neck of a protocol. This is due to the fact that a lot of overhead capacity might be needed to initiate and terminate a (secure) communication line between the other party. In general, however, this really depends on the communication technology used in practise. We use Toft’s definition of a round of communication, because it gives us a workable and intuitive definition of this concept. Toft argues that a communication round consists of sending information to other parties and performing a limitless number of arithmetic computations with a sole restriction: variables which are received by a party during this round can not be used in any arithmetic operation performed by that party in the same round.

\subsubsection{Bandwidth}
The bandwidth of a protocol is given by the maximum number of bits send during a single communication round. This performance indicator strikes an interesting issue consider the previous two KPI. When the communication complexity of a protocol doesn’t change, but we manage to decrease the round complexity, then odds are that the necessary bandwidth for the protocol will increase. This is obviously not always the case, but shows that we might encounter some trade off considering these KPI. Note that when one finds the necessary bandwidth of a protocol to hight in practise, one can always implement the rounds with the highest communication complexity complexity as two separate communications to decrease the bandwidth.

\subsubsection{Computational Complexity}
This KPI is mentioned way less in recent literature than the previous ones. Which is one of the main reasons why this research project exists in the first place. The Computational complexity is measured as the amount of time it takes ones CPU to compute the desired result of a protocol. This KPI is shunned so often as it takes a lot of extra time and effort to implement the proposed protocol. To our knowledge, the only secure comparison protocol, described in the next chapter, for which the computation complexity was studied previous to our research, was that of Garbled Circuits.