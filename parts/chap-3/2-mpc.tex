\section{Operations with additive secret sharing}
From now on, we will use the bracket-notation for a secret $\secretI{a}$ in the space of secrets $\mathbb{Z}_{\secret{k}}$. The shares are denoted by $\secretI{a_i}$ an the bar on top is there to indicate that the secret represents an integer. No brackets means that the value is not secret and thus known to everybody.

\subsection{Performing basic operations}
We will now look how to perform the basic operations that are at the core of every arithmetic circuit in additive secret sharing, i.e. addition and multiplication, with a public value or between secrets.

\subsubsection{Addition and multiplication of public constants}
Let us first consider the case where a shared value is multiplied or added to constant, i.e. $\secretI{b} = \secretI{a} + c$ and $\secretI{b} = c \cdot \secretI{a}$. This case is simple. Each party just adds or multiplies his share with the constant. This rests on some arithmetic properties of polynomials: one can easily verify that $\secretI{a_i}+c$ is a share of $\secretI{a}+c$ and $c \cdot \secretI{a_i}$ of $c \cdot \secretI{a_i}$.

\subsubsection{Addition of two secrets}
Let us consider now the addition of two hidden values $\secretI{d} = \secretI{a} + \secretI{b}$. This can be done by just adding the respective shares of secrets $\secretI{a}$ and $\secretI{b}$: $d_i = a_i+b_i$. This is a direct consequence of equation~\ref{eqn:recover-secret}.


\subsubsection{Multiplication of two secrets}
Unfortunately, it is not possible to adopt the same strategy for multiplication as the multiplication of two polynomials of order $t$ will lead to a new polynomial of order $2t$ which will double the number of parties needed to recover the secret output. In real-case functions with a lot of multiplications, this becomes rapidly impracticable and theoretically unsolvable if the degree of the new polynomial exceeds $n$. Another problem is of statistical order: the new polynomial is not random anymore as it is for example not irreducible anymore by construction.

A solution to this problem is to use multiplication triples. Each triple is a set of three pre-computed values $\left\{ \secretI{x}, \secretI{y}, \secretI{z} \right\}$ that verify $\secretI{z}=\secretI{x}\cdot \secretI{y}$ before the algorithm starts to evaluate the circuit in what is called an \emph{offline phase}. The part where the algorithm evaluates the arithmetic circuit in itself is called the \emph{online phase}. The computation of these triples requires fully homomorphic encryption and will not be further considered in this thesis (the exact method used in this thesis is based on \emph{Beaver triples} and more details on the protocol can be found in \noteH{add references to Beaver triples}). Using these triples, we can now easily compute a multiplication of two arbitrary secret values $\secretI{a}$ and $\secretI{b}$. The protocol is given in algorithm~\ref{alg:sec-mult}. The computation of secret $\secretI{e}$ and $\secretI{d}$ corresponds to the security of a one-time pad which has an perfect theoretical security. In other words, if the pad --- multiplication triple here --- is statistically secure, $\secretI{e}$ and $\secretI{d}$ are also and can be revealed without revealing anything on the input or the triple. Finally, we can algebraically verify that $\secretI{c}$ results in the correct output. The trick here is to use the one-time pad to allow to reveal $e$ and $d$. This allows to avoid making a multiplication of two shared values which in turn avoids augmenting the degree of the underlying polynomial used for the sharing. Multiplication can thus be performed using polynomial secret sharing but at the cost of having to use fully homomorphic encryption for the generation of multiplication triples and one extra round to reveal the values of $\secretI{d}$ and $\secretI{e}$.

In practice, specific \emph{square pairs} are also used to perform squaring operations in a similar manner instead of using the multiplication described here.

\begin{center}
\begin{algorithm}[H]
 \KwData{Multiplication tripe $\left\{ \secretI{x}, \secretI{y}, \secretI{z} \right\}$ and inputs $\secretI{a}$ and $\secretI{b}$}
 \KwResult{Product $\secretI{c} = \secretI{a} \cdot \secretI{b}$}
\DontPrintSemicolon
\SetKwFunction{FOpen}{Open}
$\secretI{e} \leftarrow \secretI{x}+\secretI{a}$ \;
$\secretI{d} \leftarrow \secretI{y}+\secretI{b}$ \;
$e \leftarrow$ \FOpen{$\secretI{e}$} \;
$d \leftarrow$ \FOpen{$\secretI{d}$} \;
$\secretI{c} \leftarrow \secretI{z} + e\secretI{a} + d\secretI{b} - ed$ \;
\Return{$\secretI{c}$}
\caption{Secure multiplication protocol. The function $\mathtt{Open}$ refers to the revelation of its input. In the active-case model, this also comprises a verification step (cf. infra).}
\label{alg:sec-mult}
\end{algorithm}
\end{center}

\subsection{Performing more advanced operations}
The last operation we should be able to perform to run most of the algorithms with additive secret sharing is the comparison and the conditional assignment. Comparative assignment is the assignation to a secret value of one of two secret values depending on a secret condition without knowing which of the two has been assigned. However, the comparison of two integers is not straightforward and is based on some other protocols which we first need to describe. The first thing to note is that additive secret sharing has no support for secret boolean values, and uses a secret integer value of 1 and 0.

\subsubsection{Modulo}
The modulo protocol computes the value of a secret value $\secretI{a} \in \mathbb{Z}_{\secret{k}}$ modulo $2^m$ or in other words, its first digits in base 2. More specifically $\mathtt{Mod2m}(\secretI{a},k,m) = \secretI{a} / 2^m$. The protocol is given at algorithm~\ref{alg:mod2m} and rests on two sub-protocols $\mathtt{BitLT}$ and $\mathtt{PRandM}$. 

Without going into too much details, $\mathtt{BitLT}$ essentially computes an inequality test between a two bit-wise decomposed values (noted with a subscript $B$), a public $a_B$ and a secret $\secretI{b}_B$: $\secretI{s} = (a_B < \secretI{b}_B)? \, 1:0$. Still without going into too much details, it basically rests on a circuit computing the carry-out of a boolean addition using recursivity and having logarithmic round complexity $\mathcal{O}(\log k)$. Indeed, boolean operations can easily be computed as an OR-gate corresponds to $\secretI{a}+\secretI{b} - \secretI{a}\cdot \secretI{b}$, a XOR-gate $\secretI{a}+\secretI{b} - 2\secretI{a}\cdot \secretI{b}$ and an AND-gate $\secretI{a}\cdot \secretI{b}$. More details on the implementation can be found at~\ref{} \noteH{add references to documentation and Catrina}.

The $\mathtt{PRandM}$ protocol generates two random values $\secretI{r'}$ (of length $l_1 = k + \kappa -m$) and $\secretI{r''}$ (of length $l_2 = m$) where the second is also given in its bit-wise decomposition $\secretI{r''}_B$. Similarly as the multiplication, the generation of these resides in the pre-sharing of a bit list in the offline phase. To construct the bit-wise decomposition of these values, each player just has to pick the right amount of bits $l$ in the shared list. We can then reconstruct the non-decomposed share with
\begin{equation}
    \secretI{r} \leftarrow \sum_i^l 2^i \cdot \secretI{r_i}_B
\end{equation}
For the first value $\secretI{r'}$ of length $l_1 = k + \kappa -m$, the bits can be discarded after this computation and are kept for the other share $\secretI{r''}$ of shorter length $l_2 = m$.

The reason we have to use these shared random values is because the whole operation on which the modulo resides here is $\mathtt{BitLT}$ and this only works with bit-wise decompositions. Though, the value $\secretI{a}$ we want to compute the modulo of is not given in bit-wise decomposition. Similarly as we did for addition, we use a intermediate value $c$ we can reveal and easily compute the modulo and then the bit-wise decomposition from.

In the begin of this section, we said that arithmetic circuits avoided the bit-wise decomposition unlike e.g. garbled circuits. However, it seems here that we are still residing on bit-wise decomposition for the comparison. In fact, this protocol does not really use bit-wise decomposition, at least not in the same sense as the schemes based on boolean circuits. A first difference is that the secret values we want to compute operations on, here $\secretI{a}$ are never really decomposed in bits but are solely using additions and multiplications. Here, only $c$ is decomposed into bits during the protocol and that when it is not secret anymore. The other value to be decomposed in bits is $\secretI{r'}$, but this is done by construction and it contains no single information on $\secretI{a}$ (in the sense that $\secretI{r'}$ is totally independent from $\secretI{a}$). A second difference here is that the boolean gates are reduced to additions and multiplications and no truth tables. We can thus not really talk about boolean operations as they are arithmetic operations in essence. Garbled circuits reside on oblivious transfer and the garbling of encrypted truth tables, which is not the same: not a single arithmetic operations takes place. The sole similarity is that both decompose a comparison into a boolean circuit, but these circuits are computed differently and the comparison in itself doesn't happen with the same inputs.

\begin{center}
\begin{algorithm}[H]
\KwData{$\secretI{a}$, field size $k$ and $m$}
\KwResult{$\secretI{a'} = \secretI{a} / 2^m$}
\DontPrintSemicolon
\SetKwFunction{FPRandM}{PRandM}
\SetKwFunction{FOpen}{Open}
\SetKwFunction{FBitLT}{FBitLT}
($\secretI{r''}, \secretI{r'}, \secretI{r'}_B) \leftarrow$ \FPRandM{$k,m$} \;
$\secretI{c} \leftarrow 2^{k-1} + \secretI{a} + 2^m\secretI{r''} + \secretI{r'}$ \;
$c \leftarrow$ \FOpen{$\secretI{c}$} \;
$c' \leftarrow c \mod 2^m$ \;
$\secretI{u} \leftarrow$ \FBitLT{$c'_B,\secretI{r'}_B$} \;
$\secretI{a'} \leftarrow c' - \secretI{r'} + 2^m\secretI{u}$ \;
\Return{$\secretI{a'}$}
\caption{Secure modulo $2^m$ protocol.}
\label{alg:mod2m}
\end{algorithm}
\end{center}

\subsubsection{Truncation}
Now that we are able to compute the modulo of a secret value, we can easily compute its truncation, which is essentially the opposite of the modulo. Instead of computing the rest of a division by $2^m$, we compute the value of this division. This is noted $\mathtt{Trunc(\secretI{a},k,m)} = \lfloor \secretI{a}/2^m \rfloor$ Where the modulo returns the $m$ last digits $\secretI{a'}$ in base 2, the truncation returns the $k-m$ first digits $\secretI{d}$. This can be easily computed using the rest of a division by $2^m$: $\secretI{a} = 2^m\secretI{d} + \secretI{a'}$. The protocol is given at algorithm~\ref{alg:trunc}.

\begin{center}
\begin{algorithm}[H]
\KwData{$\secretI{a}$, field size $k$ and $m$}
\KwResult{$\secretI{d} = \lfloor \secretI{a}/2^m \rfloor$}
\DontPrintSemicolon
\SetKwFunction{FModm}{Modm}
$\secretI{a'} \leftarrow$ \FModm{$\secretI{a},k,m$} \;
$t \leftarrow 1/2^m \mod p$ \;
$\secretI{d} \leftarrow$ $t \cdot (\secretI{a}-\secretI{a'})$ \;
\Return{$\secretI{d}$}
\caption{Secure truncation protocol.}
\label{alg:trunc}
\end{algorithm}
\end{center}

\subsubsection{Comparisons}
The truncation protocol allows now to compute the comparison of two secret values $\secretI{a}, \secretI{b} \in \mathbb{Z}_{\secret{k}}$. One just has to observe that the truncation protocols allows to compute the sign of a secret value as the sign is basically just the $k$-th bit of its bit-wise decomposition. Hence, we can define a lower-than protocol $\mathtt{LTZ}(\secretI{a},k) = -\mathtt{Trunc(\secretI{a},k,k-1)}$. The comparison $\secretI{a}<\secretI{b}$ can now be computed as $\mathtt{LTZ}(\secretI{a}-\secretI{b},k)$. The other comparison operators are defined similarly and the equality uses the same ideas with slight changes. \noteH{add references}.

\subsubsection{Conditional assignment}
We can now use the found boolean. The last protocol we have to design is the secure conditional assignment protocol which assigns a secret value to a variable depending on the boolean. The protocol described in algorithm~\ref{alg:sec-ass} assigns secret value $\secretI{a}$ to variable $\secretI{c}$ if $\secretI{s}$ is equal to 1, otherwise it assigning $\secretI{b}$. This protocol will be noted $\secretI{c} \leftarrow_{\secretI{s}} \secretI{a} : \secretI{b}$ further in this thesis.

\begin{center}
\begin{algorithm}[H]
\KwData{$\secretI{a}$, $\secretI{b}$ and boolean $\secretI{s}$}
\KwResult{$\secretI{c} \leftarrow \secretI{a}$ if $\secretI{s}=1$ and $\secretI{c} \leftarrow \secretI{b}$ if $\secretI{s}=0$}
\DontPrintSemicolon
$\secretI{d} \leftarrow \secretI{s} \cdot (\secretI{a} - \secretI{b})$ \;
$\secretI{c} \leftarrow \secretI{b} + \secretI{d}$ \;
\Return{$\secretI{c}$}
\caption{Secure comparative assignment protocol.}
\label{alg:sec-ass}
\end{algorithm}
\end{center}

\subsection{Working with decimal numbers}
Up to now, we were solely working with secret integers. We can also work with secret decimal numbers. 

\subsubsection{Representation} 
In this work, the decimal numbers are numerically represented as fixed points and are noted with a tilde $\secretF{a}$. A fixed point secret value is represented as a pair of two secret integers: $\secretI{a}$ representing the significand and $\secret{f}$ the exponent. The set of secret rational numbers $\mathbb{Q}_{\secret{k,f}}$ can defined using the mapping $\secretF{a} = \secretI{a}\cdot 2^{-f}$ with $\secretI{a} \in \mathbb{Z}_{\secret{k}}$. This intuitively corresponds by scaling the range of real values by a vlue of $2^{-f}$. With integers, the security condition was that $k+\kappa < p$, it now becomes $k+\kappa+f<p$.

\subsubsection{Basic operations}
All operations defined before can now be computed on the significand if the values have the same exponent. When this is not the case, we can easily change the exponent using a scaling protocol as defined in algorithm~\ref{alg:scale-sfix}. This can be the case when performing operations between secret fixed point numbers and secret integers --- that are in fact secret fixed point values with $f=0$. It just multiplies the significand if the exponent has to shrink and truncates if it has to grow. When performing the operations on the significands, we just have to prevent that it exceeds the size allowed for the significand and therefore select the right part of it using the modulo for addition (the last digits in base 2) and the truncation for multiplication (the first digits in base 2). The protocols for addition and multiplication are given in respectively algorithm~\ref{alg:sfix-add} and \ref{alg:sfix-mult}. Comparison is similarly just made on the significands and the same conditional assignment protocol as for secret integers can be used. A more specific protocols is also used for division and is described in~\ref{} \noteH{add references to Catrina sfix}.

\begin{center}
\begin{algorithm}[H]
\KwData{$\secretF{a} \in \mathbb{Q}_{\secret{k,f_1}}$, $k$, $f_1$ and $f_2$}
\KwResult{$\secretF{a'} \in \mathbb{Q}_{\secret{k,f_2}}$}
\DontPrintSemicolon
$m \leftarrow f_2 - f_1$ \;
\lIf{$m \geq 0$}{$\secretF{a'} \leftarrow 2^m \cdot \secretF{a}$}
\lElse{$\secretF{a'} \leftarrow \mathtt{Trunc(\secretF{a},k,-m)}$}
\Return{$\secretF{a'}$}
\caption{Secure scaling protocol for secret fixed point numbers.}
\label{alg:scale-sfix}
\end{algorithm}
\end{center}

\begin{center}
\begin{algorithm}[H]
\KwData{significands $\secretI{a}, \secretI{b}$ of $\secretF{a}, \secretF{b} \in \mathbb{Q}_{\secret{k,f}}$, $k$ and $f$.}
\KwResult{significand $\secretI{c}$ of $\secretF{c} = \secretF{a} + \secretF{b} \in \mathbb{Q}_{\secret{k,f}}$}
\DontPrintSemicolon
$\secretI{d} \leftarrow \secretI{a} + \secretI{b}$ \;
$\secretI{c} \leftarrow \mathtt{Mod2m(\secretI{d},k+1,k)}$ \;
\Return{$\secretI{c}$}
\caption{Secure addition protocol for secret fixed point numbers.}
\label{alg:sfix-add}
\end{algorithm}
\end{center}

\begin{center}
\begin{algorithm}[H]
\KwData{significands $\secretI{a}, \secretI{b}$ of $\secretF{a}, \secretF{b} \in \mathbb{Q}_{\secret{k,f}}$, $k$ and $f$.}
\KwResult{significand $\secretI{c}$ of $\secretF{c} = \secretF{a} \cdot \secretF{b} \in \mathbb{Q}_{\secret{k,f}}$}
\DontPrintSemicolon
$\secretI{d} \leftarrow \secretI{a} \cdot \secretI{b}$ \;
$\secretI{c} \leftarrow \mathtt{Trunc(\secretI{d},2k,f)}$ \;
\Return{$\secretI{c}$}
\caption{Secure multiplication protocol for secret fixed point numbers.}
\label{alg:sfix-mult}
\end{algorithm}
\end{center}

\subsection{Security}
The protocol is called secure when adversaries are not able to reveal any information they are not supposed to. In the case of this thesis, this also means guaranteeing the privacy of the data. By extension, we then talk about a privacy-friendly algorithm. The additive secret sharing cryptographic primitive is by design secure $\emph{information-theoretic}$ model. This means that the information is not protected by a too high complexity of an algorithm and thus by the limited computing power of an adversary, but by its the limited information contained in the data to obtain a unique solution. Indeed, the cryptographic primitive is based on polynomials in a sense that no group of players of a degree lower than the polynomial can find the secret value as solving the equation system leads to an infinite number of solutions. This also means that the scheme is cryptanalytically unbreakable: no possible loophole can be found in the scheme.

The only possible attacks are based upon the fact that the shares can release statistical information about the secret. However, this was solved using a much bigger interval than the size of the field in which the secret are defined, using security parameter $\kappa$.

Another possible attack resides on the external cryptographic primitives used for the computation of the elements needed for the offline phase, e.g. the multiplication triples. These are based on fully homomorphic encryption that are not information theoretically secure, but \emph{cryptographically secure}. However, this thesis concentrates on the secure multi-party protocols and we will consider them secure.

\subsubsection{Active and passive case}
A party that uses all its computation power to try to recover the secret, but follows the protocol is called an \emph{honest but curious} adversary. As with all the information that it has, no party can possibly recover the secret, the scheme is secure in the \emph{passive case} also called \emph{semi-honest model}. To protect the system against a \emph{malicious} adversaries, i.e. an adversary that deviates from the protocol intentionally to recover information, shared message authentication codes (MACs) are used to commit each party to its share when it reveals it to everybody. At the public revelation, or opening, the MAC value is reconstructed and everybody checks its correctness. Once the MAC is reconstructed and public, no further operation takes place.

In short, a long-term MAC key $\alpha$ is commonly constructed is such a way that each party receives a share from it $\secretI{\alpha_i}$. This can then be used to commit each player during the addition: in addition to receiving the secret shares $\secretI{a_i}$, each party $P_i$ receives a secret share $\gamma(a)_i$ such that $\alpha \cdot a = \gamma(a) + \ldots + \gamma(a)_n$ (the MAC on $a$). When performing the addition of their local shares, each party also computes the addition of its shared MACs of the values. Let's consider the addition $\secretI{c} = \secretI{a} + \secretI{b}$: each party $P_i$ computes the addition of its shares $\secretI{c_i} \leftarrow \secretI{a_i} + \secretI{b_i}$ and the addition of the MACs on the secrets $\gamma(c)_i = \gamma(a)_i + \gamma(b)_i$. We can verify that $\alpha \cdot c = \sum \gamma(c)_i = \sum(\gamma(a)_i + \gamma(b)_i) = \alpha \cdot (a + b)$. To check the commitment of everybody to its share, we have to check if indeed $\gamma= \alpha a$. We can do this avoiding to open $\alpha$ as it is the long-term key. It suffices to observe that the value $\gamma - \alpha a$ is a linear function of shared values $\gamma$ and $\alpha$ which means that players can locally compute shares and then wheck if it is 0. If this is not the case, the protocol immediately halts. As multiplication using multiplication rests on the opening of additions of secret shares and other additions, we can implement this scheme into the multiplication as well \cite{Damgard2012PracticalLimits.}.

The use of MACs to commit the parties to their shares makes the protocol secure in the \emph{active case}.

\subsection{The SCALE-MAMBA framework}
Different implementations of multi-party computation protocols exist. For protocols based on bit-wise decomposition and circuit garbling, both Fariplay \cite{Malkhi2004FairplaySystem} and BMR \cite{Beaver1990TheProtocols} are multi-party. Some hybrid using bit-wise decomposition as well as secret sharing protocols also exist such as SEPIA \cite{Burkhart2010SEPIA:Statistics} or Rmind \cite{Bogdanov2018Rmind:Analysis}, but are limited to certain pre-computed functions executions and do not allow for real programming. General frameworks that allow programming include VIFF \cite{Damgard2009AsynchronousImplementation} (which is obsolete as it has not been updated anymore since 2009), Sharemind \cite{Bogdanov2008Sharemind:Computations,Bogdanov2013Sharemind:Applications} that uses its own programming language SecreC, based on the C-language \cite{Ristioja2010AnLanguage,Jagomagis2010SecreCMining,Bogdanov2014Domain-PolymorphicOf} and that is proprietary and fixed to three parties and ABY \cite{Demmler2015ABYComputation}, fixed to two parties. The sole framework that is open-source and $n$-party is SPDZ \cite{Damgard2012MultipartyEncryption,Damgard2012PracticalLimits.}, which is recently discontinued and taken over by the SCALE-MAMBA project \cite{Aly2018SCALEDocumentation}. This is the one used in this thesis.

SCALE-MAMBA (short for Secure Compuation Algorithms from LEuven and Multiparty AlorithMs Basic Argot) also incorporates the works of \cite{Bendlin2011Semi-homomorphicComputation,Nielsen2012AComputation} and has developed its own language based on Python. A major difference with SPDZ is that is thought as of an entire system and not a set a functions that can be used, but that the programmer still has to compose together. It is therefore not suited for benchmarks on specific protocols, but very well suited for the benchmarking of general algorithms: the goal of this thesis. The implementations in this thesis are using a 128-bits prime $p$ with $k=64$ and statistical security parameter $\kappa=40$. 

SCALE-MAMBA also has a native support for high-level multi-thread protocols where the programmer should not care about shared variables, there is only a protocol for starting the different threads and another one for waiting that all threads are done. 

Furhtermore, multi-threading is at the core of SCALE-MAMBA as it natively uses four continuous seperate "feeder" threads for the offline phase, one for producing the multiplication triples, another for the square pairs, a third for shared bits and the last for other various aoperations such as the sacrificing\footnote{The multiplication triples are continuously produced but need to be sacrificed before consumption by the online stage. This thread continuously sacrifices triples and add them to a sacrificed-list, ready for use.}. This use of "feeder" threads allow to continuously produced the elements needed for the online phase without having to match both of them, which appears to be a complex enterprise. Additional global threads are also used to produce the cyphertexts used for the FHE component. However, the user has no control on all these threads as the are built at the core of the system.

\subsection{Performance indicators}
When evaluating the algorithm, we should be carefull at four performance indicators: the round complexity, the computational complexity, the communication complexity and the classification performance. As protocols run online, we should also monitor the consequences on the network usage as well as the complexity on each machine. At last the real performance of the classification itself should not be forgotten. 

\subsubsection{Round Complexity}
The round complexity is the number of round performed in total for the scheme and is a measure for the network usage. As the protocol has to wait for the exchange of every shares before going further --- waiting that a round finishes ---, a high round number means a high dependence on the latency of the network. Indeed, if the network is slow, the exchanges will be the bottleneck of the protocol and a high number of rounds will by synonymous with low execution.

\subsubsection{Computational Complexity}
The computational complexity is a measure of the complexity locally, for each party. This is measured as the amount of time each CPU takes toexecute its part of the protocol. Computational security seems less studied than other indicators \cite{Blom2014AThesis} but is key to a rapid execution and to measure the practicality of an algorithm. As we aim to measure the practicality of algorithms in real-case scenarios, this indicator should absolutely not be neglected.

\subsubsection{Communication complexity}
The communication complexity indicator measures the quantity of data transmitted over the network. This can be a bottleneck when the communication on the network is very slow, when it has a small bandwidth. For a round to be completed, all the data has to be transmitted. The communication complexity is therefore closely related to the round complexity.

\subsubsection{Classification performance}
This has nothing to do with the multi-party computation in itself but is crucial to a working system. This whole thesis aims at finding the trade-offs to improve the practical feasibility of privacy-friendly machine-learning algorithms. The trade-off between the classification performance and the MPC-feasibility is surely the most important. An very fast MPC-based algorithm is not relevant if it does not do its job correctly enough. The opposite is less true: a very slow machine-learning algorithm is --- it is true not so interesting --- but still relevant when it classifies the data correctly. Of course, the real goal is to find one that satisfies both criteria.