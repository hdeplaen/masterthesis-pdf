\documentclass[master=wit,english,fleqn,twoside,openright]{misc/kulemt}
\usepackage{misc/kulemtx}
%\setup{coverpageonly}
\headstyles{kulemtman}
\kulemtmanToC

%%%%%%%%%%%%%%%%
%%% PACKAGES %%%
%%%%%%%%%%%%%%%%

%\usepackage{ucs}
\usepackage[TS1,T1]{fontenc}
\usepackage{amsmath}
\usepackage{amsthm}
\usepackage{dsfont}
\usepackage{amsfonts}
\usepackage{amssymb}
\usepackage{mathtools}
\usepackage{tikz}
\usepackage[framemethod=tikz]{mdframed}
\usepackage{microtype}
\usepackage{textcomp}
\usepackage[pdfusetitle,plainpages=false]{hyperref}
\usepackage[titletoc, toc, page]{appendix}

%packages spéciaux
%\usepackage{fullpage}
\usepackage{mathrsfs}
%\usepackage{numprint}
\usepackage{graphicx}
\usepackage{float}
\usepackage{wrapfig}
\usepackage[font=small,labelfont=sc,justification=centering]{caption}
\usepackage{makecell}
\usepackage{array}
\usepackage{csquotes}
\usepackage[boxed]{algorithm2e}
\usepackage{lipsum}

\usepackage[numbered,framed,autolinebreaks]{misc/mcode}

%% FONTS
\usepackage[lf]{Baskervaldx} % lining figures
\usepackage[bigdelims,vvarbb]{newtxmath} % math italic letters from Nimbus Roman
\usepackage[cal=boondoxo]{mathalfa} % mathcal from STIX, unslanted a bit
\renewcommand*\oldstylenums[1]{\textosf{#1}}
%\usepackage{sfmath}
%\usepackage{sansmathaccent}
%\usepackage{beramono}

\usepackage{caption}
\usepackage{subcaption}
\usepackage{upquote}

%% STYLE
\pagestyle{headings}
\usepackage{misc/slashbox}
\usepackage{placeins} % permet d'utiliser \FloatBarrier et donc d'imposer le placement des images dans une section donnée.
\usepackage{multirow}

\newcommand\scalemath[2]{\scalebox{#1}{\mbox{\ensuremath{\displaystyle #2}}}}
\newcommand{\HRule}{\rule{\linewidth}{0.5mm}}

\newcommand{\overbar}[1]{\mkern 1.5mu\overline{\mkern-1.5mu#1\mkern-1.5mu}\mkern 1.5mu}

\definecolor{grisclair}{HTML}{F2F2F2}
\lstset{language=Matlab, breaklines=true, backgroundcolor=\color{grisclair} , numbers=left, frame=none, aboveskip=0cm} %, basicstyle=\tiny}

\usepackage{fancyhdr}
%\addto\captionsenglish{\renewcommand{\chaptername}{Assignment}}

%% TABLES
\usepackage{tabularx}
\usepackage{longtable}
\usepackage{ltxtable}
\newcommand\hlineI{\noalign{\hrule height 1.2pt}} % I line 1.2pt
\newcolumntype{I}{!{\vline width 1.2pt}} % I column 1.2pt

%\newcolumntype{L}[1]{>{\raggedright\let\newline\\\arraybackslash\hspace{0pt}}m{#1}}
%\newcolumntype{C}[1]{>{\centering\let\newline\\\arraybackslash\hspace{0pt}}m{#1}}
%\newcolumntype{R}[1]{>{\raggedleft\let\newline\\\arraybackslash\hspace{0pt}}m{#1}}

\newcolumntype{C}[1]{%
 >{\vbox to 2ex\bgroup\vfill\centering\arraybackslash}%
 p{#1}%
 <{\vskip-\baselineskip\vfill\egroup}}

%% SCALED VERBATIM
\usepackage{verbatim}
\usepackage{adjustbox}
\usepackage{fancyvrb}
\newenvironment{myverb}{%
 \VerbatimEnvironment
 \begin{adjustbox}{max width=1\linewidth}
 \begin{BVerbatim}
  }{
  \end{BVerbatim}
 \end{adjustbox}
}

%% PERSONAL COMMANDS
\newcommand{\noteH}[1]{{\color{red} \textit{#1}}}
\newcommand{\norm}[1]{\left\lVert#1\right\rVert}

\newcommand{\secret}[1]{\langle #1 \rangle}
\newcommand{\secretI}[1]{\langle \bar{#1} \rangle}
\newcommand{\secretF}[1]{\langle \tilde{#1} \rangle}

%% OTHERS
\DeclareMathOperator{\tr}{tr}

%% BIBLIOGRAPHY
\usepackage[backend=biber,style=ieee,url=false,dashed=false,doi=false,bibwarn=false,sorting=nyt]{biblatex}
\addbibresource{misc/references.bib}
%\usepackage{cite}

%%%%%%%%%%%%%
%%% SETUP %%%
%%%%%%%%%%%%%

\setup{title={Privacy-friendly machine learning algorithms for intrusion detection systems},
  author={Henri De Plaen},
  promotor={Prof.\,Dr.\,ir.\,Bart~Preneel},
  assessor={Prof.\,Dr.\,ir.\,Karl~Meerbergen \and Prof.\,Dr.\,ir.\,Sabine~Van~Huffel},
  assistant={ Dr.\,Aysajan~Abidin \and Dr.\,Abdelrahaman~Aly}}

\setup{filingcard,
  %translatedtitle=Privacyvriendelijke algoritmes voor het machinaal leren van inbraakdetectiesystemen,
  udc=51-7,
  shortabstract={In this thesis, we present a set of practically usable machine-learning algorithms for intrusion detection systems using multi-party computation (secret sharing). This allows a user to query an already trained model preserving both the privacy of the query and of the model. Two classical algorithms used for intrusion detection systems are investigated: support vector machines, both linear and non-linear, and nearest neighbors. Although non-linear support vector machines not seem very suited for the problem, linear support vector machines and nearest neighbors methods deliver very satisfying results. To make nearest neighbors feasible, condensed nearest neighbors are applied beforehand exploiting the trade-off between expensive clear pre-processing and a lightweight secret model. Furthermore, PCA and chi-square feature selection methods are investigated to reduce the evaluation costs even more.}}

%\setup{font=lm}

%%%%%%%%%%%%%%%%
%%% DOCUMENT %%%
%%%%%%%%%%%%%%%%

\begin{document}
\thispagestyle{empty}

 \begin{preface}
 I would like to thank everybody who kept me busy the last year, especially my promotor and my assistants. I would also like to thank the jury for reading the text. My sincere gratitude also goes to my wive and the rest of my family.
 \end{preface}
\begin{abstract}
In today's information age, the risks and security threats to data are commonplace. Vital enterprise and user information should be protected from uninvited intruders lurking in our systems. Imagine the case where intruders are accessing to sensitive medical, industrial or governmental records using stolen or hacked but valid credentials. Machine learning techniques and algorithms have been used as a tool for intrusion detection in such scenarios. They typically monitor the behavior of the users, detecting anomalies that indicate possible infiltrations.

\end{abstract}
\tableofcontents*
% A list of figures and tables is optional
%\listoffigures
%\listoftables
% If you only have a few figures and tables you can use the following instead
\listoffiguresandtables
% The list of symbols is also optional.
% This list must be created manually, e.g., as follows:
\chapter{List of Abbreviations and Symbols}
\section*{Abbreviations}
\begin{flushleft}
  \renewcommand{\arraystretch}{1.1}
  \begin{tabularx}{\textwidth}{@{}p{12mm}X@{}}
    LoG   & Laplacian-of-Gaussian \\
    MSE   & Mean Square error \\
    PSNR  & Peak Signal-to-Noise ratio \\
  \end{tabularx}
\end{flushleft}
\section*{Symbols}
\begin{flushleft}
  \renewcommand{\arraystretch}{1.1}
  \begin{tabularx}{\textwidth}{@{}p{12mm}X@{}}
    42    & ``The Answer to the Ultimate Question of Life, the Universe,
            and Everything'' according to H2G2 \\
    $c$   & Speed of light \\
    $E$   & Energy \\
    $m$   & Mass \\
    $\pi$ & The number pi \\
  \end{tabularx}
\end{flushleft}

\captionsetup[figure]{list=yes}
\captionsetup[table]{list=yes}
%\captionsetup[algorithm]{list=yes}

\mainmatter
\chapter{Introduction}
\label{cha:1}
More than ever, information today is equal to money and power. The four most valued companies in the world (Microsoft, Apple, Google and Amazon) all have a significant part of their business model based on information and data. Furthermore, the rest of the industry has also fully entered into this information mutation: data is now everywhere. And so is sensitive data such as medical, governmental or industrial data. This characterizes the new information age we now live in. However, networks are now experiencing more and more attacks of various types trying to recover this information. The new information paradigm has to come with its new defense mechanisms.

Machine-learning has allowed improvement of many models, including the ones identifying network attacks. Though, these models have to be constructed using significant amounts of data, which are often based on previous real attacks making them sensitive. These models also ideally should be used as much as possible to better protect the information. We thus face the following paradox: sensitive data should be used to improve the protection of sensitive data.

This paradox can be solved by using multi-party computation (MPC). This allows for different parties to commonly compute an agreed-upon function where its input remains private and the output is revealed to certain parties. This solution allows a user to query a machine-learning algorithm trained on external data-bases, for its own defense while maintaining the privacy of all the data: the query and the data-base, on which the machine-learning algorithm has been trained. In other words, these solutions allow the use of sensitive data for a certain purpose without revealing it. This is a specific case of the now trendy \emph{private-data as a service} (PDaaS).

However, if MPC allows us to solve our paradox, it also has a huge drawback: every operation is proportionally much more costly. We can thus not use MPC as we would use classical algorithms and must actively investigate the trade-offs we can make to reduce these costs. This thesis investigates how machine-learning can be used to evaluate a query based on an external data-base while preserving the privacy of the data, in the specific case of \emph{intrusion detection systems}.

\section{Related works}
% IDS IN GENERAL
Intrusion detection systems are network security techniques that aim to detect malicious activity by comparing it to an existing data-base based on previous malicious activities \cite{Amudhavel2016AReview}. The general philosophy is that an intruder has different behaviour than a legitimate user. It can be based on two methods \cite{Winter2018}. Rule-based, which is also known as \emph{signature-based}, aims to compare a new query to a data-base of specific patterns --- the so-called signatures --- that contain the sole reduced information to detect an anomaly. The big advantage of these methods is that they are very lightweight and can easily detect attacks that correspond to these signatures. Their disadvantage however is their very strong dependence on these signatures and the inability to detect anomalies that are not contained in the signatures. These signatures are furthermore often specific to certain systems or architectures \cite{Ilgun1995StateApproach}. In other words, signature specific methods almost always detect the same anomalies as the ones it has in its signature data-base, but fails at detecting anything that diverges from it \cite{Liao2013IntrusionReview}. This problem is solved by an alternative to signature-based methods: \emph{anomaly-based methods}. They rely on more abstract, statistical models that are able to generalize the attack patterns and by this way detect attacks that have not been previously identified \cite{Dali2015ASystem}. However, the generalization property has also its drawback as they tend to also over-identify normal activities as anomalies, resulting in a high false positive rate and a low false negative rate \cite{Mukherjee1994NetworkDetection}. A lot of research has been conducted on these issues in the last 20 years, leading to very effective anomaly-based systems, by replacing more simple statistical models by more complex machine-learning algorithms \cite{Tsai2010ADetection}. The use of intrusion detection system is more and more critical in our connected world and is now widely used in the industry due to the potential high financial cost of failing to detect anomalies \cite{Bhatt2014TheSystems,Rossi2009UnderstandingSignaling}. Although signature-based systems are still used by some industry majors such as Cisco Systems, the wide majority is now consisted of anomaly-based models \cite{Rubio2017AnalysisEcosystems}.

% DISTRIBUTED
A big improvement in the last years is the use of distributed intrusion detection systems to monitor data across multiple nodes. Yegneswaran et al. have built the DOMINO overlay system, which proves to be able to detect attacks that could not be detected on isolated nodes and furthermore increases the general detection speed \cite{Yegneswaran2004GlobalSystem}. The general philosophy behind distributed intrusion detection systems is that users are not fixed, they change their ID and their target \cite{Snapp1991DIDSPrototype}. In a certain way, distributed intrusion detection systems gain more efficiency through the correlation of different individual data-bases \cite{RoyceRobbins2004DistributedReview}. The main problem when working with different data-bases is the privacy of the data. Information about the network use of a user cannot be made available to the public, nor in the distributed intrusion detection system itself, especially when it is distributed among different and independent networks. The processing of these data-bases through the system's algorithms has to be made privacy-friendly.

% PRIVACY-FRIENDLY
There are different approaches to privacy-preserving data-mining \cite{Narwaria2016PrivacyArt}. The first one is based on differential privacy, where the results are still treated in clear but in anonymized statistical manner to limit a maximum the possibility of de-anonmyzing the data-points \cite{Dwork2008DifferentialResults}. This approach has been applied to various machine learning algorithms such as support vector machines or image processing \cite{Qin2018Privacy-PreservingCloud,Zhan2005Privacy-preservingLearning}.

The other approach is to use cryptographically secure algorithms where the data is cryptographically hidden and processed in its hidden form to be finally revealed to the original data-owner. The big advantage of this method is that there is no possible way of finding statistical information about the data processed, which is not the case of differential privacy. However, these advantages have a cost: they are computationally very expensive algorithms. This is called homomorphic encryption. An alternative is to use \emph{multi-party computation} where the data is distributed between different players which then process the information in such a way that the information cannot be revealed if sufficient participants are staying honest. These methods are based on ideas developed by Yao and Rabin or secret sharing \cite{Yao1986HowSecrets}, \cite{Rabin1981HowTransfer.} and \cite{Shamir1979HowSecret}.

Multiple algorithms have been implemented using multi-party computation for aggregation and statistical analysis of network events such as SEPIA \cite{Burkhart2010SEPIA:Statistics}. However, it has the drawback of being limited to two parties and only simple aggregation operations and basic statistical measures. Bogdanov et al. are proposing Rmind, a set of similar cryptographically secure toolboxes, this time with more than one party, but also limited to simple statistical algorithms, e.g. linear regression \cite{Bogdanov2018Rmind:Analysis}. More complex machine-learning algorithms are still timid. 

Cryptographically private support vector machines have been theorized by Laur et al. \cite{Laur2006CryptographicallyMachines}, but the only real implementations known to us are given in the domain of image-processing using linear kernels \cite{Makri2017PICS:SVM}. Barnett et al. are using polynomial kernels, but using homomorphic encryption and not multi-party computation, still in the domain of image classification~\cite{Barnett2017ImageData}. There are also some implementations using neural networks, always residing on homomorphic encryption \cite{Dowlin2016CryptoNets:Research}. 

Shaneck et al. have been the first to propose privacy-friendly nearest neighbors using multi-party computation, but with a clear query point \cite{Shaneck2009PrivacySearch}. Methods encrypting both the query and the data-base have been implemented using homomorphic encryption \cite{Wong2009SecureDatabases,Hu2011ProcessingHomomorphism}, but appear not to be secure over plaintext attack \cite{Yao2013SecureRevisited}. Qi et al. have proposed a specific two-party protocol based on additive secret sharing, without any implementation~\cite{QiEfficient}.

To our knowledge, no privacy-friendly machine-learning algorithms for intrusion detection systems have been implemented using multi-party computation and not much has been done in privacy-friendly machine-learning algorithms using MPC in general.

In general, intrusion detection systems are considered to be classification problems. However, the classifiers cannot use the raw data as such. Before being used in a machine-learning, the packets --- which are of huge size --- have to be reduced to a set of features \cite{Winter2018} which is usually quite big independently of the method chosen to generate them \cite{Sekar2002Specification-basedDetection,Cho2003EfficientModel} \cite{NewsomePolygraph:Worms}. A lot of different machine-learning algorithms have been tested and used on intrusion detecting systems \cite{Tsai2009IntrusionReview}. The first ones are neural networks which tend to be very effective for binary classification but less at multi-class \cite{Mukkamala2002IntrusionMachines,Akashdeep2017AClassifier,Farnaaz2016RandomSystem}. A second type of algorithms are decision trees \cite{Meeragandhi2010EffectiveRules,Papamartzivanos2018Dendron:Systems} and more generally random forest classifiers \cite{Soheily-Khah2018IntrusionDataset}. However, decision trees and random forests are rule-based methods which are much more difficult to hide and to make privacy-friendly. Multi-party computation is difficultly compatible with rule-based methods. Fuzzy logic \cite{Shanmugavadivu2011NetworkLogic,Rout2015ADetection} is also used but much less common. Another technique is naive Bayes \cite{Chebrolu2005FeatureSystems}, but tends to have a lesser accuracy, and hidden Markov models \cite{Chen2016AnomalyModel,Tsai2007DetectingModels} that suffer from the same problem. The last two and most used methods are support vector machines and nearest neighbours methods. A lot of less classical methods have also been tested, but little research has been focused on them \cite{Sabhnani2003ApplicationContext}.

Mukkamala et al. have successfully used support vector machines to make the distinction between normal and attack classes using a simple SVM with Gaussian kernels~\cite{Mukkamala2002IntrusionMachines}. Much more complex models have also been used combining KPCA and SVM and training the parameters with genetic algorithms \cite{Kuang2014ADetection}. Ming et al. have showed that adding data-points to the data-set based on aggregation of that same data-base allows to discover more complicated attacks with support vector machines~\cite{MingTian2004UsingAttacks}. This same idea has also been successfully implemented using random forests classifiers~\cite{Akashdeep2017AClassifier}.  Investigations have also been conducted on how to reduce the training set size of SVMs for intrusion detection systems training~\cite{Khan2007AClustering,Al-Yaseen2017Multi-levelSystem}. In general SVMs are able to detect almost all kinds of problems if the model it is used in is well built. This shows the importance of the research around support vector machines for intrusion detection systems. 

Due to the cost of multi-party computation, the data-set sizes used during the secure computation of the algorithm have to be reduced as much as possible without losing accuracy. The feature-size reduction has been implemented using principal component analysis decomposition and $\chi^2$ feature selection~\cite{Eid2010PrincipleSystem,Ikram2016ImprovingSVM,SumaiyaThaseen2017IntrusionSVM,Yang:1997:CSF:645526.657137}. In general, it seems that SVM are almost always using Gaussian kernel functions, sometimes polynomial, but never linear.

The nearest neighbours algorithm has also been studied for feature reduction with PCA and KPCA~\cite{Elkhadir2016IntrusionMethods}. However, algorithms for reducing the data-points number like the one proposed in \cite{Angiulli2005FastRule} seem to never have been tested.








\section{Claims}
In this thesis, we describe and analyze a way for a user to query an external machine-learning algorithm for the classification of a possible attack (figure~\ref{fig:eval-model}). More specifically, The machine-learning model has been trained beforehand and is described my model parameters. Both the privacy of the query and the model parameters are preserved trough multi-party computation; the type of machine-learning algorithm queried is known as well as the reductions used. As the execution of MPC-based algorithms is very expensive, we suggest outsourcing (in a cloud) it to a group of servers (in our case 3), but the protocols described can directly be used between the different parties without outsourcing, in any $n$-party setting. All the interactions between the cloud, the user and model owners(s) are done using secure connections (TCP), as well as the connections between the servers of the cloud.

\begin{figure}
    \centering
    \includegraphics[width=.95\textwidth]{parts/chap-1/img-1/eval-model.jpg}
    \caption[Proposed setting]{Proposed setting for privacy-friendly machine-learning algorithms for intrusion detection systems.} 
    \label{fig:eval-model}
\end{figure}

We provide algorithmic solutions to the evaluation of support vector machines using secret sharing. Both linear as radial-based functions are implemented. To the best of our knowledge, this is the first time non-linear support vector machines are implemented under MPC security constraints. Due to the cost of MPC, the training of SVMs becomes very expensive. These algorithms are thus limited to evaluation against already trained models and thus the use of only one model owner. We showed that linear SVMs allow rapid evaluation of a query, especially when reduction methods are used. Non-linear SVMs are too slow, even with the reductions investigated.

We also provide algorithmic solutions to the nearest neighbors evaluation using the same secret sharing. We show that condensed nearest neighbors allow to make nearest neighbors practically feasible in the case of privacy-friendly intrusion detection systems.

Trade-offs that can be made to reduce the various costs of classical support vector machines and nearest neighbors, are investigated. Various reduction methods are pushed to their limits to see if they allow both of these algorithm to run in reasonable times to be used with multi-party computation. This comprises research in both the domains of machine-learning algorithms for intrusion detection systems and MPC-based machine-learning algorithms.

Furthermore, this thesis systematically compares the implementation of two different SVMs multi-class models: one-against-all and tree-based models. Up to now, they were alternatively used in the literature without systematic comparison. Tree-based models appear to be much faster without a loss of performance, both with linear as non-linear support vector machines.

We also investigate how feature size reduction methods such as PCA reduction or $\chi^2$ feature selection impact the classification performance and speed of the algorithm. To our knowledge, $\chi^2$ feature selection has never been applied before to nearest neighbors algorithms in the case of intrusion detection systems.

Next to investigating feature size reduction, we also investigate instance-set (or training set) size reduction: $k$-means and condensed nearest neighbors. To our knowledge, none of these algorithms have been applied to nearest neighbors in the case of intrusion detection systems. The condensed nearest neighbors allow to dramatically decrease the computation cost of the evaluation.

The two algorithms we present are practically usable in the proposed setting as they have limited computational, communication and round cost, for a totally satisfying classification accuracy.
\section{Organization of this thesis}
This thesis is organized in five main chapters, including this one. 

\begin{itemize}
	%\item This chapter gives a general overview of the subject, the related works and what to find in this work.
	\item The second chapter presents the different reduction methods used as well as the SVM and the nearest neighbors models.
	\item The third chapter explains the existing techniques to achieve privacy and a justification for the method we opted for, the context and theoretical foundations used in the method we opted for as well as the algorithms we used using this setting.
	\item The fourth chapter presents the results for each model, first the reduction methods and then the results on MPC. The models are tested in this order: linear SVMs, non-linear SVMs and finally nearest neighbors.
	\item The fifth chapter finally concludes the thesis summarizing the results and presenting directions for future works.
\end{itemize}

A lot of data has been produced during the evaluation of the results of the different algorithms. We tried to limit ourselves to a strict minimum in the fourth section. Additional results can be found in the appendix. The appendices also contain information on the code used for the various implementations. 
\chapter{Machine-learning algorithms for intrusion detection systems}
\label{cha:2}
The main task in statistical learning is to avoid any \emph{overfitting}: any useless addition of complexity may allow the regression to capture the variance of the specific data-set on which it is trained on top of the underlying relation is it supposed to capture. The regression would then reproduce the training set itself instead of generalising it. I therefore would like to cite John von Neumann
\begin{displayquote}
\emph{With four parameters I can fit an elephant, and with five I can make him wiggle his trunk.}
\end{displayquote}

In other words, one parameter is sufficient to add a lot of complexity to the model and increase a lot the risk of overfitting. The general rule of thumb is thus to always reduce as much as possible the complexity of the model and always stick to the lowest one able to reproduce the general scheme of the data. This principle is also known as \emph{Occam's razor}.

\section{Intrusion Detection Systems}
\emph{Intrusion detection systems} (IDS) are a brick in the existing defence algorithms arsenal wall of information security. More specifically, it comprises a series of mechanisms that monitor network nodes and detect intrusions, i.e. malicious activity or policy violations. An IDS usually analyses the incoming packets and notifies the suspect ones. In the most often cases, IDS are defined to be the sole surveillance application and does not comprise the control application: how the suspect packets are treated after a notification is not considered as being part of the IDS, the latter only focuses on the monitoring, analysis and notification \cite{Mukherjee1994NetworkDetection}. Classically, the reports are made to an administrator or another competent entity, as \emph{security information and event management} (SIEM) \cite{Bhatt2014TheSystems}, which are then in charge of the control application. 

IDS should not be confused with \emph{firewalls}, but merely be seen as a complement of it. The sole role of firewalls is to ensure that communication policies are followed carefully. A first
difference is that firewalls have an upstream role whereas IDS are working downstream. In other words, firewalls are verifying that each packet is following carefully one of the pre-defined allowed communication protocols, before it enters the local network. An IDS analyses the packets after they entered the local network to control if they shows no abnormal behaviour. Another second difference has already been mentioned: firewalls consider each packet separately whereas IDS can consider group of packets and thus look at a communication as whole. In this sense, IDS are much more suited against \emph{denial of service} (DoS) attacks than classical firewalls. A last difference concerns the exact scope of the packet analysis. As firewalls only have to enforce communication policies, they only have to look at the packet header, whereas IDS are searching for abnormal behaviours and are thus looking at the packets on their whole. To summarise this all, let's consider a high security building: the firewall would be equivalent to the agents allowing or not each individual to enter the site by carefully inspecting their papers, whereas the IDS would be the security agents monitoring the cameras inside to building searching for abnormal behaviour.

IDS should also not be confused with \emph{anti-viruses} application --- though the term \emph{anti-malware} would be more suited nowadays --- that refer to the application layer in charge of the detection and control of malicious code, or malware. The first difference concerns the scope of their analysis: anti-viruses are analysing (executable) code on a system more specifically than packets on a network. The second difference is similar as before: anti-viruses analyses code before it is allowed to be executed by the system and IDS are analysing packets that already entered the network.

However, all these taxonomy classifications are to be considered with some flexibility. As the attacks become more and more sophisticated, security entities are incorporating more and more subtleties are extending the scope of their detection methods. As such, they integrate other type of methods classically defined by other entities.

\subsection{How IDS work}
As briefly stated before, IDS have three main components: monitoring, analysis and notification. 

The monitoring can be achieved in real time or at regular interval on different types of nodes, which define the type of IDS: network based IDS (NIDS), host based IDS (HIDS) and hybrid of they monitor on both. 

The analysis is the core part of the IDS and is again divided in three main components: the extraction of features, the pattern analysis and the final classification \cite{Winter2018}. This will be the part which will interest us in this thesis.

As written before, the notification is done to a controller, either ban administrator or a SIEM. Classically, this takes place as the form of a series of logs which are later examined by the controller. In this sense, the speed is not the main focus of an IDS, but rather the correct identification of intrusions. However, one can also consider \emph{intrusion prevention systems} (IPS) which are working upstream. The literature sometimes consider these systems to be a specific class of IDS or to be a category on their own. However in opposition to IDS, IPS need to be fast and thus usually use signature-based detection. In this sense, IPS can be seen as an extension of firewalls as they also analyze the content of packets and not just the enforcing of protocols. Taking the IPS into account, one should still consider analysis speed in IDS.

The general structure of IDS is summarized at figure~\ref{img:ids-model}.

\begin{figure}[t]
    \centering
    \includegraphics[width=.95\textwidth]{parts/chap-2/img-2/ids-model.jpg}
    \caption{General structure of an intrusion detection system. The different attack classes are based here on the chapter~\ref{cha:4}.} 
    \label{img:ids-model}
\end{figure}

\subsection{Extraction of features}
Packets to be analyzed are of huge size and cannot be analyzed as such by the IDS. They typically incorporate huge redundancy and other non-necessary information such as padding. The idea of feature reduction is to reduce the packets to a limited number of features, representing as much non-redundant information as possible. Examples of features extracted consist of the connection length, the protocol used, the number of bytes transferred from the source to the destination and inversely.


\subsection{Pattern analysis}
The goal of the analysis of the IDS is to categorize the packets into different classes, typically a normal class and some different attack classes. As stated in the introduction, this can be done by two different manners based on a signature (\emph{signature-based}) and using some statistical rules (\emph{anomaly-based}). In this thesis, we will interest us to anomaly-based IDS, more specifically using classification machine-learning algorithms. The goal is that some party is able to classify its data based (query) on trained machine-learning algorithms from another party. Furthermore, these algorithms have to be privacy-friendly in the sense that nor the query, nor the information resulting from the training to be revealed. This chapter aims at describing the two main machine-learning algorithms used without any consideration of the privacy-friendliness that will covered in chapter~\ref{cha:3}.

\FloatBarrier
\section{Data reduction}


\subsection{Instance set size reduction}
Instance-based learning algorithms are often faced with the problem of deciding which instances to store for use during generalization. Storing too many instances can result in large memory requirements and slow execution speed, and can cause an oversensitivity to noise. \noteH{Input size so big as possible to reduce variance of the results}

In the practice, the data can be classified into two kind of classes, attack and no attack, or normal. In a practical intrusion detection system, most of the traffic is normal and sole a few proportion is attacking. We thus want to reduce the size of the normal data-set to make it of a similar size to the attack classes. Furthermore, the normal instances are very diversified and one may not just reduce the set randomly in the risk of missing relevant instances. One must thus find a more intelligent heuristic to decide which normal instances to keep.

How to decide which instances keep?
Keep only the ones near to the decision boundary. (+ explain what is a decision boundary). Way of keeping the sole normal instances near the decision boundary out of a way bigger set and thus build a more relevant training set of normal instances.

\subsubsection{$k$-means clustering}
The $k$-means clustering algorithm is a semi-supervised instance set reduction algorithm. This a specific case of clustering methods, that aim to divide the original data-set into smaller $k$ smaller clusters that should share some common properties. The challenge here is to find the most homogeneous possible clusters, which instances are similar enough --- referred to as the minimisation of \emph{intra-class inertia} ---, but not too many so that the clusters are well differentiated -- also referred to as the maximiastion of \emph{inter-class intertia}. After this being achieved, one can thus reduce each cluster far from the decision boundary to an archetype instance of this cluster and keep the clusters near the decision boundary. The algorithm is semi-supervised as the clustering in itself is only based on the features set, but is only applied on the data-points that are classified as normal, which of course depends on the target space.

To determine the different clusters, the $k$-means algorithm tries to minimize the distance between the instances in each cluster $\mathcal{S}_i$. The number clusters is fixed to $k$, which an hyper-parameter of the model set by the user. Computing the distance between each of the points would require a quadratic at each evaluation of cluster. To keep the complexity linear, the distances are not computed between each of the instances, but between each of the instances and the mean of these instances $\mu_i = \left( \sum_{x_j \in \mathcal{S}_i} x_j\right)/\vert \mathcal{S}_i \vert$ where $\vert \mathcal{S}_i \vert$ is the number of instances in the cluster.
\begin{equation}
    \underset{\mathcal{S}}{\mathrm{arg}\,\mathrm{min}} \sum_{i=1}^k \sum_{x_i \in \mathcal{S}_i} \norm{x_j-\mu_i}^2
\end{equation}
Defining the optimal clusters would at least be exponentially complex in function of the number of total instances as each cluster combination would have to be tested. Therefore, the $k$-means algorithm starts with assigning the starting clusters to initial data-points, chosen at random in our case. It then goes through the whole data-set and assigns each point to the cluster with the nearest mean $\mu_i$. The means are then updated and the process starts again. This is given at algorithm~\ref{alg:k-means}. Although doesn't guarantees any optimality nor polynomial computational time, it is considered very effective in the practice \cite{Arthur2006Worst-caseMethod}.

\begin{center}
\begin{algorithm}[H]
 \KwData{Feature data-set $\mathcal{F} = \left\{x_i\right\}_{i=1 \ldots n}$ and number of cluster sets $k$}
 \KwResult{$k$ clusters $\mathcal{S}_i = \left\{x_i\right\}_{i=1 \ldots m_i}$}
\DontPrintSemicolon

\lForEach{i = 1\ldots k}{
$\mu_i \leftarrow$ random $x_j$
}
\Repeat{$\mathrm{convergence}$}{
\ForEach{$x_j \in \mathcal{F}$}{
\lForEach{$i = 1\ldots k$}{$d_{i} \leftarrow \norm{x_j-\mu_i}$ \;}
$l \leftarrow \mathrm{arg}\,\mathrm{min} d_i$ \;
assign $x_j$ to $\mathcal{S}_l$ \;
}
\ForEach{$i = 1\ldots k$}{
$\mu_i \leftarrow \mu_i = \frac{1}{\vert \mathcal{S}_i \vert}\left( \sum_{x_j \in \mathcal{S}_i} x_j\right)$
}
}
\Return{$\mathcal{S}_{1 \ldots k}$}
\caption{The $k$-means algorithm. The convergence criterion typically is no more evolution in the means or the composition of the clusters, which is almost always equivalent in the practice.}
\label{alg:k-means}
\end{algorithm}
\end{center}

Once the clusters are determined, we can remove the normal data instances of the farthest from the decision boundary and replace the whole cluster by its mean $\mu_i$. In this way, we reduce the number of points far from the decision boundary and keep all the ones near the decision boundary. The number of set to be reduced $p<k$ is also an hyper-parameter set by the user. A way to determine which one are the farthest is just to compute the distance between the mean of each cluster and the mean of the attack instances defined in the same way.

\subsection{Feature size reduction}
+ explain why reduce the feature size. Complexity gain and better suited for distance calculation. Avoid curse of dimensionality. Avoid participation of non-relevant features in the training of the classification algorithm.

\subsubsection{Principal components analysis}
A linear Principal Components Analysis dimensionality reduction consists in a spectral analysis of the covariance matrix $\mathbf{\hat{\Sigma}}$ \noteH{(+ give forula for covariance matrix)} which can diagonalized with nonnegative eigenvalues $\lambda_i$ as it is semi-positive definite
\begin{equation}
\hat{\Sigma}x = \lambda_ix
\end{equation}
The $M$ greatest eigenvalues are then kept and the data transformed accordingly:
\begin{equation}
z_i = x_i^Tx
\end{equation}
Here is an attempt of an intuitive understanding of the process. Each input variable of the original input vector is independent, but some have strong underlying relations between them, measured by their covariance matrix. By diagonalizing it, there result $n_f$ new variables without any linear underlying relation. In a certain way, the eigenvectors $x_j$ represent these relations and their respective eigenvalues $\lambda_j$ their contribution, thus their importance. We can then afford to drop the non-important ones.

In our case, each feature is first being normalized, i.e. zero mean and unit variance. In other words, the covariance matrix has an unit diagonal and with a feature dimension of $n_f$, we have $tr(\hat{\mathbf{\Sigma}})=n_f$ which corresponds in a certain way to the total variance of the whole feature space. Quite logically, the latter is invariant to any linear transformation as the trace is an invariant to any linear transformation. The main idea behind linear PCA analysis reduction is thus to keep the richness of the data a maximum, thus its total variance, and reducing its dimension a maximum by dropping its smallest contributors. A good measure for the relative importance of each eigenvector, or underlying linear relation, is its fraction or variance contribution given by
\begin{equation}
f_j = \lambda_j/\Sigma_{\forall j}\lambda_j = \lambda_j/\tr\left(\hat{\mathbf{\Sigma}}\right)
\end{equation}

One can then choose the input values it decides to keep in descending order of the respective fractions $f_j$.

\subsubsection{Kernel principal components analysis and Mercer's trick}
The main problem of PCA is that it limits to linear relations between the feature elements. However, feature relations, especially in high dimension spaces require non-linear relations. Therefore, it would be interesting to map the instances into a space of higher dimension $\phi(x)$.

Therefore, we may now compute the covariance matrix in the mapped space $\phi\left( \mathcal{F} \right)$ of the feature space $\mathcal{F}$
\begin{equation}
    \hat{\Sigma}_{\phi} = \frac{1}{n_f} \sum_{x_i,x_j \in \mathcal{F}} \left( \phi(x_i) - \mathrm{mean} \right) \cdot \left( \phi(x_j) - \mathrm{mean} \right)
\end{equation}

Fortunately, there exists a method to compute this easily. Mercer's trick \cite{Minh2006MercersSmoothing}, also known as \emph{the kernel trick} can be used when an algorithm is based on the sole scalar product between feature vectors instances. It can then be replaced by a \emph{kernel matrix} representing the scalar product of the transformations.
\begin{equation}
    K_{ij} = K(x_i,x_j) = \phi(x_i) \cdot \phi(x_j)
\end{equation}

To compute the covariance matrix, we first have to centralize the data, which can be easily done with simple matrices multiplication
\begin{eqnarray}
    K_{ij}'&=& K_{ij} - \frac{2}{n_f} \sum_k^{n_f}\left( K_{ik}+K_{jk} \right) + \frac{1}{n_f^2} \sum_{l,m,n,o}^{n_f}K_{lm}K_{no} \\
    K' &=& K - 1_{n_f}K - K1_{n_f}+ 1_{n_f}K1_{n_f}
\end{eqnarray}
where $1_{n_f}$ represents matrices of the size $\left( n_f , n_f \right)$ where all elements are equal to $1/n_f$.

We can now perform a diagonalization on this new covariance matrix of the transformed data-points and perform the classification on them. This method is called \emph{kernel principal components analysis} (KPCA). 

A very interesting property of the kernel matrix is there is no need for computing the explicit transformation of each data-point as we are only interested in their scalar product. Indeed, they are never used on their own, but always in the kernel matrix. As such, one may directly compute the direct matrix. This is even more interesting as the scalar product of most of the transformations are mapped into an infinite dimension feature space and the scalar product can only theoretically be computed in Hilbert sace. Different kernel functions are possible, that represent the scalar product of different transformations. These kernel functions rely more often on one or more hyper-parameter that has to be trained. The most common choices are
\begin{itemize}
    \item \textbf{Linear kernel function:} this corresponds to computing directly the scalar product onto the identity transformation $\phi(x_i)=x_i$ and is thus equivalent to a linear PCA analysis.
    \begin{equation}
        K(x_i,x_j) = x_i \cdot x_j
    \end{equation}
    \item \textbf{Radial basis function kernel function (RBF):} this kernel function has one hyper-parameter. At a scale factor excepted \noteH{(à une constante près ?)}, this corresponds to the probability of the second instance given the first one with standard deviation $\sigma$. An interesting property of this kernel function is its ability to measure similarity as it converges to zero as the distance between both data-points increase and the point thus becoming more and more dissimilar. RBF are known to have excellent generalization capabilities.
    \begin{equation}
        K(x_i,x_j) = e^{\frac{-\norm{x_i-x_j}^2}{2\sigma^2}}
    \end{equation}
    \item \textbf{Polynomial kernel function:} this kernel function also has one hyper-parameter $d$. In the practice, $d=2$ is often chosen as higher order tend to overfit.
    \begin{equation}
        K(x_i,x_j) = (x_i \cdot x_j +1)^d
    \end{equation}
\end{itemize}

Due to their strong generalization power, RBF kernel generally perform better than other kernels, especially if no additional knowledge of the data is available or multiclass problems as the different classes often are often more diverse and require more generalization. In the case of intrusion detection systems, RBF kernel functions also tend to show better results \cite{Kuang2014ADetection}. Though, other kernels seem to show better results in some specific binary cases \cite{Elkhadir2016IntrusionMethods}. In our case and as we will work with multi-class classifiers, we will also opt for RBF kernel functions.

\subsubsection{Chi-square feature selection}
Up to new, the idea was to reduce the feature size by constructing new ones as transformations of the original feature vectors. However, one also choose to to select a number of features and just not care about the other ones. Another interesting property would be to be able to reduce the feature size in a supervised way, taking the machine learning algorithm into account. This can be tackled with $\chi^2$-feature reduction.

This feature selection method is based on the $\chi^2$ statistical hypothesis test that measures the dependence of two variables.





$\chi^2$-feature reduction test have successfully been applied to some machine learning algorithms for intrusion detection systems and seem to deliver better results than PCA or KPCA \cite{SumaiyaThaseen2017IntrusionSVM}.
\section{Support Vector Machines}
Support vector machines are a set of supervised machine learning algorithms than can be used either for classification or regression proposed by Vapnik in 1963 \cite{VapLer63}. The main idea originates with binary classification and consists of finding an optimal hyper-plane between the data-points separating both target classes.

\subsection{Linear support vector machines}
In its primal form, the support vector machine corresponds to the following minimization problem
\begin{equation}
    \begin{aligned}
& \underset{x}{\text{minimize}} 
& & \frac{1}{2}w^Tw + C \sum_{k=1}^N \xi_k \\
& \text{subject to}
& & y_i\left[ w \cdot x_i+b \right] \geq 1-\xi_i, \; i = 1, \ldots, N \\
& 
& & \xi_i \geq 0, \; i = 1, \ldots, N.
\end{aligned}
\end{equation}

This corresponds to finding a hyper-plane defined by the vector $w$ and intercept $b$ that separates all the data-points. The minimization of $w^Tw = \norm{w}^2$ corresponds to maximizing the margin between the hyper-plane and the nearest data-points. This is the criterion used to define a unique hyper-plane as an infinity of potential hyper-planes that separates the data-points correctly may exist. Indeed, the nearest data-points $x$ are on the parallel hyper-plane $w^Tx-b= \pm 1$. The distance between the margin and the nearest point is thus $1/\norm{w}$. Maximizing it means minimizing $\norm{w}$ and due to the monotony of the norm $\norm{w}^2 = w^Tw$.

Of course, it can happen that no hyper-plane is able to classify all points correctly. Slack variables $\xi_k$ are therefore introduced. The trade-off between the best possible classification of the data-points (the contribution of the slack variables) and the maximizing of the margin is controlled by the box contraint parameter $C$.

New data-points (queries) are estimated as
\begin{equation}
    \mathtt{SVM}(x) = w \cdot x + b.
\end{equation}

The classification is given by the side of the hyper-plane on which the data-point resides which corresponds to taking the sign of this estimation.

In comparison to other machine learning algorithms like neural networks for example, the SVMs present the big advantage of taking the form of a convex optimization problem. But their greatest strength in my opinion is the ability to use transformations for representing a non-linear separation. Therefore, we have to to work in the dual space, where the SVM optimization problem now reformulates
\begin{equation}
    \begin{aligned}
& \underset{\alpha_i}{\text{maximize}} 
& & \sum_{i=1}^N \alpha_i - \frac12 \sum_{i,j=1}^N \alpha_i \alpha_j y_i y_j (x_i \cdot x_j) \\
& \text{subject to}
& & 0 \leq \alpha_i \leq C \; i = 1, \ldots, N \\
& 
& & \sum_{i=1}^N\alpha_i y_i = 0,
\end{aligned}
\end{equation}
and the estimation of a new data-point now becomes
\begin{equation}
    \mathtt{SVM}(x) = \sum_{i=1}^N \alpha_i y_i (x_i \cdot x) + b.
\end{equation}

The dual formulation gives its name to support vector machines. A new data-point is estimated based on some vectors of the training size and their corresponding weights. The box constraint parameter here corresponds to the maximum weight of a support vector. In the case of privacy-friendliness, support vectors are very sensitive data as they directly represent instances from the original training data-set.

\subsection{Mercer's trick and non-linear support vector machines}
The main problem of support vector machines as described above is that they are limited to linear relations between the feature elements. However, feature relations, especially in high dimension spaces, require non-linear relations. Therefore, it would be interesting to map the instances into a space of higher dimension $\phi(x)$.

The feature vectors are only appearing in the form of a scalar product, which allows us to use Mercer's trick. The idea is to replace the scalar products by a kernel function $K(x_i,x_j) = \phi(x_i) \cdot \phi(x_j)$. In most kernel functions, the feature vector is mapped into a space of infinite dimension. We thereby have to consider the scalar product in a Hilbert space to keep the mathematical sense of it: $\phi(x_i) \cdot \phi(x_j) = \langle \phi(x_i), \phi(x_j) \rangle_{\mathcal{H}}$.

We can define the kernel matrix as $K_{ij} = K(x_i,x_j)$. A very interesting property of the kernel matrix is there is no need for computing the explicit transformation of each data-point as we are only interested in their scalar product. Indeed, they are never used on their own, but always in the kernel matrix. As such, one may directly compute the direct matrix. This is even more interesting as the scalar product in Hilbert spaces are practically unfeasible as such. Different kernel functions are possible, that represent the scalar product of different transformations. These kernel functions rely more often on one or more hyper-parameter that has to be trained. The most common choices are
\begin{itemize}
    \item \textbf{Linear kernel function:} this corresponds to computing directly the scalar product onto the identity transformation $\phi(x_i)=x_i$ and is thus equivalent to a linear support vector machine.
    \begin{equation}
        K(x_i,x_j) = x_i \cdot x_j.
    \end{equation}
    \item \textbf{Radial basis kernel function (RBF):} this kernel function has one hyper-parameter. At a scale factor excepted, this corresponds to the probability of the second instance given the first one with standard deviation $\sigma$. An interesting property of this kernel function is its ability to measure similarity as it converges to zero as the distance between both data-points increase and the point thus becoming more and more dissimilar. RBF are known to have excellent generalization capabilities.
    \begin{equation}
        K(x_i,x_j) = e^{\frac{-\norm{x_i-x_j}^2}{2\sigma^2}}.
    \end{equation}
    \item \textbf{Polynomial kernel function:} this kernel function also has one hyper-parameter $d$. In practice, $d=2$ is often chosen as higher order tend to overfit.
    \begin{equation}
        K(x_i,x_j) = (x_i \cdot x_j +1)^d.
    \end{equation}
\end{itemize}

Due to their strong generalization power, RBF kernel generally perform better than other kernels, especially if no additional knowledge of the data is available or multi-class problems as the different classes often are often more diverse and require more generalization. In the case of intrusion detection systems, RBF kernel functions also tend to show better results \cite{Kuang2014ADetection}. Though, other kernels seem to show better results in some specific binary cases \cite{Elkhadir2016IntrusionMethods}. In our case and as we will work with binary classifiers, but multi-class classifiers, we will also opt for RBF kernel functions.

The SVM dual formulation now becomes
\begin{equation}
    \begin{aligned}
& \underset{\alpha_i}{\text{maximize}} 
& & \sum_{i=1}^N \alpha_i - \frac12 \sum_{i,j=1}^N \alpha_i \alpha_j y_i y_j K(x_i, x_j) \\
& \text{subject to}
& & 0 \leq \alpha_i \leq C \; i = 1, \ldots, N \\
& 
& & \sum_{i=1}^N\alpha_i y_i = 0,
\end{aligned}
\end{equation}
with estimation given by
\begin{equation}
    \mathtt{SVM}(x) = \sum_{i=1}^N \alpha_i y_i K(x_i,x) + b.
\end{equation}

However, there is also a downside: we are now facing an optimization problem in the dimension of the number of input instances and not of the dimension of the feature space anymore. This means that the number of support vector will drastically increase, which is a bad scenario for our privacy-friendly models. This is one of the investigated trade-offs.


\subsection{Multi-class SVMs}
The main problem of SVM for multi-class classification is that they are not natively suited for it. Indeed, SVM return a single number representing how far the tested data-point is from the hyper-plane. SVMs are by definition created for binary classification as the whole idea behind it is to separate the feature space into two parts. The kernel trick doesn't change anything to that as it only projects the data-point into a space of higher --- often unlimited --- dimension where a new hyper-plane is searched for, but still dividing this new higher dimension space into two parts.

Ideas for classifying SVMs into different classes could be for example to put the output number into bins. However, this is a very bad idea as it would in fact suggest that all the different classes are divided by parallel hyper-planes at a distances corresponding the the range of the bins, which range could also be trained (figure~\ref{mach:svm-model-gr-1}). It almost never occurs that one hyper-plane separates two classes perfectly, the fact that $n-1$ parallel hyper-planes separate the feature space according the the $n$ different classes is even less likely. This hypothesis is much too strong and must be discarded. We therefore have to use more than one SVM and combine them.

We will present here two different ways of combining SVMs that have both been tested in the case of intrusion detection systems \cite{Kuang2014ADetection,SumaiyaThaseen2017IntrusionSVM} and produce fairly similar results although they never have been compared using exactly the same model for the rest.

The first way and the most frequent one is to create $n$ different classifiers, each for a class, and to look at which one produces the best results:
\begin{equation}
    \mathrm{min}\left\{ \mathtt{SVM}_k (x_i)   \right\}_{1 \ldots n},
\end{equation}
where $\mathtt{SVM}_k$ represents the SVM for each class $k$. Each classifier takes as binary input, +1 for the class and -1 for all the rest data-points  (figure~\ref{mach:svm-model-1}). A way of interpreting it is looking for which SVM the data-point lies the farthest from the hyper-plane, at its good size (i.e. positive). This can be graphically observed at figure~\ref{mach:svm-model-gr-2}. This is called the \emph{one-against-all} model.

\begin{figure}[ht!]
    \centering
    \includegraphics[width=.85\textwidth]{parts/chap-2/img-2/model-svm-1.png}
    \caption{One-against-all multi-class SVM model.} 
    \label{mach:svm-model-1}
\end{figure}

Another way of doing it is to work with successive SVMs as in figure~\ref{mach:svm-model-2} and is referred to as \emph{tree-based} models in this thesis. The first SVM is trained using one class as input versus all the others, as the first model. However the next classes are trained using one of the remaining classes versus the others remaining classes. The big advantage is that is only need $n-1$ different SVMs, which is one less compared to the previous model. Another advantage is that the last SVMs are trained one more specific data. However, the drawback is -- which is recurrent in series model --- that if one of the SVM isn't efficient, all the model is suffering. Graphically, this can be interpreted as first dividing the space with an hyper plane into two parts. One of the two parts is the subdivided into two new parts and so on (figure~\ref{mach:svm-model-gr-3}). Other tree structures are also to be considered, but this is not the scope of this thesis. We will limit ourselves to the consideration of these tree-structures in different orders. The goal of the machine-learning study of this thesis is to try to reduce the number of operations made by the evaluation of a query on the machine-learning algorithm. Computing 4 SVMs instead of 5 indeed decreases these number of computations. We thus want to investigate if this has an impact on the classification performance. A comparison study between the different tree-based models is a totally different topic which is not pursued here. We just limit ourselves to the different models found in the papers.

For the sake of completion, we also have to mention the existence of \emph{one-against-one} models where a binary classifier is trained for all combinations of two classes. The winning class it the one whose corresponding classifiers have the most positive outputs. However, this model needs $n_c(n_c-1)/2$ different classifiers where $n_c$ is the total number of classes, which is the opposite of the goal we are seeking.


\begin{figure}[ht!]
    \centering
    \includegraphics[width=.75\textwidth]{parts/chap-2/img-2/model-svm-2.jpg}
    \caption{Tree-based multi-class SVM model.} 
    \label{mach:svm-model-2}
\end{figure}

\begin{figure}
\begin{subfigure}[b]{0.32\textwidth}  
            \centering 
            \includegraphics[width=.85\textwidth]{parts/chap-2/img-2/svm-par.png}
            \caption{Single SVM with bins.} 
            \label{mach:svm-model-gr-1}
        \end{subfigure}
        \hfill
        \begin{subfigure}[b]{0.32\textwidth}  
            \centering 
            \includegraphics[width=.98\textwidth]{parts/chap-2/img-2/svm-multi.png}
            \caption{$n$ SVMs in a one-against-all model.} 
            \label{mach:svm-model-gr-2}
        \end{subfigure}
        \hfill
        \begin{subfigure}[b]{0.32\textwidth}   
            \centering 
            \includegraphics[width=.98\textwidth]{parts/chap-2/img-2/svm-seq.png}
            \caption{$n-1$ SVMs in a tree-based model.} 
            \label{mach:svm-model-gr-3}
        \end{subfigure}
        \caption[Comparison of different multi-class models]{Comparison of different multi-class models in the feature space. In this case, there are three classes $n=3$: dots, circles and triangles.}
\end{figure}
\section{Nearest neighbours}
A classification algorithm that is often used in intrusion detection systems is the $k$-nearest neighbour ($k$-NN). This is a simple and effective method based on the distance of the elements in the feature space. Intuitively, when we want to characterize new data elements, we compare it to existent data elements and try to assign to it the characteristics --- in this case, the intrusion target --- of the similar ones. In other words, the test set elements are classified as similar training elements. Though, how do we count elements to be similar, how do we measure similarity? In the $k$-NN algorithm, the elements are considered as data-points in the feature space and the similarity is measured as the euclidean distance between them. The most common target of the similar elements is then assigned to the test element.

More concretely, let's first consider a binary classifier of observations and targets $\left( x_1, y_1 \right), \ldots , \left( x_n, y_n \right)$ where observations $x_i \in \mathbb{R}^d$ and targets $y_i \in \left\{0,1\right\}$. To classify a new observation $\left( x_i, y_i \right)$, we just have to compare it to the $k$ other nearest observations and assign to the test observation the dominant target among its $k$ neighbours. In the classical $k$-NN algorithm, the neighbours are defined by the Euclidean distance. However, because of the monotony of the square function, one may use the square of the distances instead, avoiding to compute an extra square root, which is very useful in frameworks where operations are expensive like MPCs:
\begin{equation}
    \mathrm{d}^2\left( x_i, x_j \right) = \norm{x_i-x_j}^2 = \sum_{k=1}^d \left( x_{ik}, x_{jk} \right)^2.
\end{equation}

Unlike binary classification algorithms like SVMs, $k$-NN algorithms are suited for multi-class problems without requiring any construction. However, they have also been used for binary classification in a construction \cite{Aburomman2016ASystem}. 

\subsection{Condensed Nearest Neighbours Data Reduction}
The problem of the nearest neighbours algorithm is the time it takes to compute for large data-bases. First, the computation of the distances increases linearly with the number of training points for each test points. Secondly, finding the nearest neighbours requires at least each distance to be touched once, a linear complexity being thus here a lower bound. As we will see, one of the main trade-offs of MPC is the drastic increase in execution time. Therefore, the reduction of data-points becomes a necessity for the use of MPC-based $k$-NN.

The \emph{condensed nearest neighbours data reduction} (CNN) algorithm aims at finding the only relevant data-points in the whole training set. The unclassified data-points then only need to be compared with the prototype points and not the whole data-base anymore. It is based on the idea that misclassified data-points lie close to the decision boundary, and one can just keep the data-points near the decision boundary and discard the others. The goal of the algorithm is to classify all training data-points into three categories:
\begin{itemize}
    \item \textbf{Outliers:} points which would not be recognized as the correct type if added to the data-base later. In other words, these data-points are the ones making the model perform worse with them than without. They increase the complexity of the decision boundary and increase significantly the number of kept data-points and thus the relevance of the algorithm as it tries to keep a minimum of points defining the decision boundary. These points are to be discarded.
    \item \textbf{Prototypes:} the minimum set of points required in the training set for all the other non-outlier points to be correctly recognized. These data-points thus contain almost all of the relevant data. These points define the decision boundary and are to be kept.
    \item \textbf{Absorbed points:} points which are not outliers, and would be correctly recognized based on the sole prototype points. This could also be characterized as the redundant data. These points are far from the decision boundary and are to be discarded.
\end{itemize}

The outliers are first found by testing all existing points against the rest of the data-base with the chosen number $k$ of neighbours. If the point isn't recognized as the correct class, it is considered as an outlier, otherwise not. After defining all the outliers, one can proceed into the classification the remaining data-points as prototypes or absorbed points. This is done as following. Each point is once again tested against the data-set without the outliers, but with one sole neighbour, i.e. $k=1$. If it is well classified, it is considered as an absorbed point and can be left out of the final data-set. Otherwise, it is considered as a prototype ans is considered relevant to the classification or not redundant. These data-points are the sole ones making it to the final data-set. All points are tested until no prototype anymore makes it into the final data-set. The CNN data-reduction is given in algorithm~\ref{alg:cnn}.

In the condensed nearest neighbours algorithm, all new data-points are tested against the reduced data-base. Another big advantage of this algorithm is that the algorithm now always has to be used with $k=1$ because of the way we defined our prototypes and the final algorithm is thus faster. However, the classification with CNN will most of the time lead to a slightly different classification than with the classical $k$-NN against the whole data-set.

\begin{center}
\begin{algorithm}[H]
 \KwData{Set $\mathcal{S}$ of $n$ feature points $\left\{x_i\right\}_{i=1 \ldots n}$ and corresponding targets $\left\{y_i\right\}_{i=1 \ldots n}$, number of neighbours $k$ for outlier detection}
 \KwResult{Reduced set $\mathcal{R}$ of $m<n$ feature points $\left\{x_i\right\}_{i=1 \ldots m}$ and corresponding targets $\left\{y_i\right\}_{i=1 \ldots m}$}
\DontPrintSemicolon
\SetKwFunction{FkNN}{kNN}

\ForEach{$\left(x_i, y_i\right) \in \mathcal{S}$}{
$t \leftarrow $\FkNN{$k$, $x_i$, $\mathcal{S}$} \;
\lIf{$t \neq y_i$}{remove chosen $\left(x_i, y_i\right)$ from $\mathcal{S}$}
}
$\mathcal{R}_0 \leftarrow \emptyset$ \;
add random $\left(x_i, y_i\right)$ to $\mathcal{S}$ \;

\Repeat{$\mathcal{R}_0 = \mathcal{R}_1$}{
$\mathcal{R}_1 \leftarrow \mathcal{R}_0$ \;
\For{$\left(x_i, y_i\right) \in \mathcal{S}$}{
$s \leftarrow $\FkNN{$k=1$, $x_i$, $\mathcal{R}_1$} \;
\lIf{$s \neq y_i$}{add $\left(x_i, y_i\right)$ to $\mathcal{R}_1$}
}
}
\Return{$\mathcal{R}_1$}
\caption[The condensed nearest neighbors algorithm.]{The condensed nearest neighbours algorithm. This algorithm relies on an implementation of the $k$-NN, represented here by the $\mathtt{kNN}$ function that takes as input the number of neighbours $k$, the data-points to be classified $x_i$ and the set in which it should search for the neighbours $\mathcal{S}$.}
\label{alg:cnn}
\end{algorithm}
\end{center}
\section{Ensemble methods}
\subsection{Bagging}
Bagging, short for \emph{boostrap aggregating} has first been described by Breiman \cite{Breiman1996BaggingPredictors}. Instead of one instance of a model trained on a whole learning set, different instances are trained with different bootstrap replicates of the original learning set. The final inference is then made by a majority vote on the different results. In other words, this method averages the set of the different possible learned models and reduces the instability of the prediction method. 

Let's consider a data-set of $N$ elements $\mathcal{L}=\left\{ \left( x_n,y_n\right),n=1,...,N\right\}$ where $x$ is the input vector and $y$ the output vector. The idea is to minimise the variation of the predictor $\hat{y}=\phi\left(x,\mathcal{L}\right)$ by calculating its expectation over the distribution of $\mathcal{L}$
\begin{equation}
    \phi_A\left(x,\mathcal{L}\right) = \mathbb{E}_\mathcal{L} \phi\left(x,\mathcal{L}\right) \approx A \left( \left\{ \phi\left(x,\mathcal{L}_k\right) \right\} \right) \qquad k=1,\, \ldots,\, N_k
\end{equation}
where $A$ is an aggregation function (\emph{i.e.} mean for a regression or a majority vote for a classification) and $\mathcal{L}_k$ different instances of the distribution of $\mathcal{L}$. As the instances set $\{ \mathcal{L}_k \}$ is not available, a bootstrap set $\{\mathcal{L}^{(B)}\}$ is constructed, each instance consisting of $N$ elements drawn randomly from the original $\mathcal{L}$, but \emph{with replacement}. The average prediction is then calculated as
\begin{equation}
    \phi_B\left(x,\mathcal{L}\right) = A(\{ \phi(x,\mathcal{L}^{(B)}) \})
\end{equation}
In his experiments, Breiman noted that 25 a 50 bootstrap replicates $N_k$ seemed a reasonable choice. In a certain sense, one could see bagging as a variant of cross-validation method on the whole learning process: a sort of \emph{cross-learning} with replacement.

The reason why bagging works is the much lower mean-squared prediction error of $\phi_A$ compared to $\phi$. Nevertheless, the fact of using the available $\phi_B$ instead of the theoretical $\phi_A$ has also drawbacks as it can deteriorate the prediction of already stable classifiers. In other words, bagging unstable classifiers such as neural nets, classification and linear regressions, usually improves them, whereas bagging usually stable classifiers such as $k$-nearest neighbours is not a good idea \cite{Breiman1996HeuristicsSelection}.

\subsection{Influence of MPC}
Compare naïve bootstrapping (each bootstraps its own copy) and full bootstrapping as described above. Can be done in parallel (MPC wins a lot from parallelization).

\subsubsection{Boosting}
The boosting method is based on a proof by Schapire \cite{Schapire1989} that weak learnability, an algorithm that slightly out-performs a random classifier, is equivalent to a strong classifier. To prove this equivalence, he used an algorithm that sequentially trains classifiers. Each classifier has a training set consisting of half well-classified elements of the previous one and half of wrongly classified elements, the first classifier starting from the original training set. Based on this idea, a later algorithm, called \emph{adaptive boosting} was developed based on a better distribution of the missclassified elements for creating the subsequent training sets \cite{Freund1997ABoosting}. + majority vote from all classifiers.

+ intuitively the algorithm will perform better when learning from the more difficult elements. In a certain sense, the more complex elements will have more degrees of freedom (not sure of this one).
\input{parts/chap-2/6-combining.tex}

\FloatBarrier
\chapter{Achieving the preservation of privacy}
\label{cha:3}

\section{Different approaches on privacy-friendliness}
In a world of constant data exchange between different entities, it might be usefull to develop methods not only to protect data during the transit as described before, but also when in possession of an entity that should not be able to read all of it. This is of particular interest for computation outsourcing where a specific data-set has to be processed by an external entity that should not be able to infer anything more than what is asked from it. Furthermore, it might even be wished that one or more external entities might not be able to read the output of their computations, solely readable by the owner(s) of the data. This is of particular interest for the --- now everywhere --- cloud solutions. A protocol that respects the private character of data when treated by other entities than the owner is called \emph{privacy-friendly} or \emph{privacy-preserving}. There exists different approaches to privacy-friendliness and for the stake of completeness, hereafter follows a short survey of them which also justifies our choice.

\subsection{Differential privacy}
Instead of encrypting all the data, one could alternatively directly address the core problem of why we want them encrypted: to prevent other parties to get any information on what data we possess. In the case that will interest us in this thesis, we will be in possession of a lot of data from personal users which is confidential and can therefore not be traced back to the user. In 2009, Netflix launched the Neltfix prize on data recommendation: the first group to improve their recommendation score by 10\% or more would win 1.000.000\$. They provided a data-set to let the participants train their models and took care of anonymising the data before they made it accessible. However, Narayanan and Shmatikov showed how they could re-identify a lot of the users using the scores of the users on IMDb. \emph{Differential privacy} \cite{Dwork2008DifferentialResults} addresses this problem by adding noise to the data and thereby achieving a better anonymisation. In this way, the data is still usable for statistical models but cannot be used to identify anyone as easily as before. Still, differential privacy is limited by the fundamental and intrinsic relation between anonymisation and statistical relevance. One cannot obtain the first without inevitably having an influence on the second one, and reciprocally.

\subsection{Homomorphic encryption}
Differential analysis is a statistical approach of anonymity, but there exists also some cryptographic approach, where the external entity is not able to read the results of what it produces. It is possible to construct a protocol with one or more third parties in a way that they cannot possibly learn anything from what they are receiving nor what they are sending back: the information is processed in an encrypted and not a clear form. Encryption schemes that allow mathematical operations to be executed on the encrypted data are called \emph{homomorphic} and was first proposed by Rivest et al. in 1978. For example, the RSA encryption scheme preserves the multiplication over the encrypted data. As a reminder, the RSA encryption scheme is given by $\mathscr{E}(m)=m^e \mod N$. We thus have $\mathscr{E}(m_1) \cdot  \mathscr{E}(m_2) = \left(m_1^e \mod N \right)\left(m_2^e \mod N \right) = \left(m_1m_2\right)^e \mod N = \mathscr{E}(m_1m_2)$. Unfortunately, this property is only true for multiplication and is therefore quite limited in its applications. We therefore refer to RSA as a \emph{somewhat-homomorphic} encryption scheme (SHE). When all mathematical operations are possible, we say from the encryption scheme that it is \emph{fully-homomorphic} (FHE). Up to now, sole some schemes based on finite fields possess this property. 

The most accomplished method up to now is called the \emph{Approximate Eigenvector Method} and is based on the \emph{Learning With Errors} (LWE) encryption scheme, which is also known for still being secure in a post-quantum era. If $C_1$ and $C_2$ are two matrices with common eigenvector $\vec{s}$, we notice that the sum or multiplication of their respective eigenvalues $m_1$ and $m_2$ corresponds to the eigenvalue of the sum or multiplication of $C_1$ and $C_2$ with respect to $\vec{s}$. The eigenvector act as a private key and the eigenvalues as the secrect messages. The scheme is thus fully homomorphic. However, eigenvectors are easy to find and the scheme is thus also insecure. To resolve this, the method uses approximate eigenvectors $\vec{s}C=m\vec{s}+\vec{e}\approx m\vec{s}$ which is known to be still solvable in finite fields under a few assumptions about the error $\vec{e}$.

\subsection{Multi-party computation}
When different parties participate to the input, the homomorphic encryption described above cannot be used anymore: all parties have to share the same secret key which makes their data still private with respect to the third party, but not to each other. Multi-party computation address this problem. Furthermore, it also allows the parties to compute a common function on their private inputs without needing one or more third parties. 

More concretely, let us now imagine a problem where the goal is to compute some common function $f$ over private inputs $x_i$. On other words we want to compute $f\left(x_1, \, \ldots, \, x_n\right) = \left( y_1, \, \ldots , \, y_n\right)$ where each input $x_i$ is privately provided by player $i$, which ultimately learns $y_i$ and nothing more: nor the other outputs, not the other inputs. A first naive implementation would be to trust a third party for privately receiving each player's input, computing the function and privately communicating the corresponding responses to everyone. However, it also possible to obtain the same results without the trust of a third party, where the sole players are participating to the protocol. This is called \emph{multi-party computation} (MPC) also referred to as \emph{secure multi-party computation} (SMC). This approach has been chosen to solve our problem and will be now more extensively described in the next section.
\section{Multi-party computation in a nutshell}
Multi-party computation originated with the toy example presented by Yao in 1982 \cite{Yao1982ProtocolsComputations} and now known as the \emph{Millionaire's Problem}: two millionaires both want to know who is richer, but none of them want to disclose their fortune nor trust a third party. Other applications of MPC may concern electronic voting or solutions of private-data as a service (PDaaS).

+ general description of rounds etc...

\subsection{Bit-wise decomposition}
Two types: bit-wise decomposition and additive sharing. Arithmetic and boolean circuits.

\subsubsection{Oblivious transfer}
The idea behind \emph{oblivious transfer} originally described by Rabin in 1981 \cite{Rabin1981HowTransfer.} is the transfer of an information in possession of a first party and asked by a second party without the first knowing which information has been transferred. Hence, the name oblivious, or alternatively, unconscious. Different protocols exist and are all based on the \emph{RSA scheme}. 

The most common version is the \emph{1-2 oblivious transfer} \cite{Even1985AContracts} and goes as follows: Alice is in possession of two messages $m_0$ and $m_1$ and Bob wants to get message $m_p$. Alice first generates a set of private key $d$ and public key $(N,e)$ and sends two random messages $x_0$ and $x_1$ to Bob. He then generates a random message $k$ and encrypts it with the $x_i$ corresponding to the wanted message: $v = \left(x_p + k^e\right) \mod N$ and sends it to Alice. She then recovers both $k$ without knowing which one corresponds to Bob's original one: $k_i = \left(v-x_i^d\right) \mod N$. These $k_i$ then serve to encrypt the messages which are finally sent to Bob $s_i = m_i+k_i$. Bob can then only decrypt the wanted message $m_p = s_p-k$. 

This protocol has been generalised to more than two parties \cite{Ishai1997PrivateApplications,Shankar2008AlternativeTransfer,Tassa2011GeneralizedSharing}

\subsubsection{Yao's garbled circuits}
\emph{Garbled circuits} (GC) were first introduced by Yao in 1986 \cite{Yao1986HowSecrets} and now one of the most efficient solutions for generic secure two-party computation. A function has to be decomposed into a boolean circuit consisting of two-input gates (e.g. XOR and AND). Let's consider the simplest example of evaluating an AND-gate between Alice and Bob. Alice first generates a different random sequence --- also called \emph{labels} --- for each possible value of each input --- also called \emph{wires} --- and output. In the truth table, the output are then symmetrically encrypted with the hash of each corresponding input. These four resulting cyphertexts are then randomly permuted --- hence the name \emph{garbled} --- and sent to Bob. The garbling of an AND-gate is illustrated at table~\ref{tab:ang-garb}. Once Bob receives the garbled gate, he then asks Alice for her label. As they have been randomly chosen, she can send it to him without him possibly knowing what value it corresponds to. Afterwards, he also needs to know the label of his input. This part is a bit more tricky and is solved using the previously described \emph{oblivious transfer}. Bob can now compute the hash of the two labels and decrypt each element of the garbled gate until we find one corresponding with the garbled gate he recieved from Alice. He can then reveal the value to Alice either she can reveal the mapping of the garbling. The same principle can be used on a multi-gate circuit by garbling the sole end result.

\begin{figure}
        \begin{subfigure}[b]{.32\textwidth} 
            \centering 
            \begin{tabular}{IC{.6cm}|C{.6cm}IC{1.3cm}I}
            \hlineI
            A & B & output \\ \hlineI
            0 & 0 & 0  \\ \hline
            1 & 0 & 0 \\ \hline
            0 & 1 & 0 \\ \hline
            1 & 1 & 1 \\ \hlineI
            \end{tabular}
            \caption{Truth table.} 
        \end{subfigure}
        \hfill
        \begin{subfigure}[b]{.32\textwidth} 
            \centering 
            \begin{tabular}{IC{.6cm}|C{.6cm}IC{1.3cm}I}
            \hlineI
            A & B & output \\ \hlineI
            $x_A^0$ & $x_B^0$ & $x_{\textnormal{output}}^0$ \\ \hline
            $x_A^1$ & $x_B^0$ & $x_{\textnormal{output}}^0$ \\ \hline
            $x_A^0$ & $x_B^1$ & $x_{\textnormal{output}}^0$ \\ \hline
            $x_A^1$ & $x_B^1$ & $x_{\textnormal{output}}^1$ \\ \hlineI
            \end{tabular}
            \caption{Labelled truth table.} 
        \end{subfigure}
        \hfill
        \begin{subfigure}[b]{.32\textwidth}
            \centering 
            \begin{tabular}{IC{3.2cm}I}
            \hlineI
            output \\ \hlineI
            $\mathscr{E}_{H\left(x_A^1,x_B^0\right)}\left(x_{\textnormal{output}}^0\right)$ \\ \hline
            $\mathscr{E}_{H\left(x_A^1,x_B^1\right)}\left(x_{\textnormal{output}}^1\right)$ \\ \hline
            $\mathscr{E}_{H\left(x_A^0,x_B^0\right)}\left(x_{\textnormal{output}}^0\right)$ \\ \hline
            $\mathscr{E}_{H\left(x_A^0,x_B^1\right)}\left(x_{\textnormal{output}}^0\right)$ \\ \hlineI
            \end{tabular}
            \caption{Garbled output.} 
        \end{subfigure}
        \captionof{table}{Garbling of the AND-gate.}
        \label{tab:ang-garb}
\end{figure}

It is interesting to note that the secure evaluation of a sole AND-gate does not respect the principles of the multi-party computation, by definition of the AND-gate. Indeed, if the final solution is 1, both players know their respective input value, which is thus disclosed\footnote{This is not the case for the XOR-gate as an 1-output has two corresponding inputs possible, as has the 0-output}. Therefore, the total functions evaluated have to be totally surjective for each input. The circuit corresponding to the Millionaire's problem is given at figure~\ref{c2:yao-comp} and while it consists of AND-gates, the function is totally subjective with respect to each millionaire's fortune.

\begin{figure}[ht!]
    \centering
    \includegraphics[width=.7\textwidth]{parts/chap-3/img/yao-comp.jpg}
    \caption{The figital comporator is the boolean circuit used to soleve Yao's millionnaire problem.} 
    \label{c2:yao-comp}
\end{figure}

This protocol executes in polynomial time, but there exists a lots of optimisations that allow to garble and evaluate the gates more rapidly.

\subsubsection{GMW protocol}
The \emph{Goldreich-Micali-Widgerson} (GMW) protocol can be seen as an extension of garbled circuits to multiple parties using Yao's idea of using oblivious transfer \cite{Goldreich1987HowGame}. The main principles here are based upon bit-sharing: each party shares its input bits among the $n$ players $b = \sum_{i=1\ldots n}b_i \mod 2$. Each player then processes their shares among the circuit. XOR-gates are easy as they can just addition the shares $c_i = a_i + b_i$. However AND-gates are more tricky: we can see from the decomposition that $c = a \cdot b = \sum_{i\neq j}a_ib_j+\sum_{1\leq i< j\neq n}\left(a_ib_j + a_jb_i \right) \mod 2$. By consequence, each party will have to compute $a_ib_i+ \sum_{i\neq j}\left( a_ib_j + a_jb_i\right)$. As for the XOR-gate, the first part is trivial to evaluate, however, the second cannot be computed by party $i$ without more information from party $j$. This is solved by using a variant of garbled circuits with oblivious transfer between parties $i$ and $j$.

\subsection{Avoiding bit-wise decomposition}
Alternatively, some arithmetic circuits can also be used for multi-party computation. The problem of the bit-wise decomposition and the use of the boolean circuit transcription of the function we jointly want to evaluate is their expensiveness in terms of performance. Indeed, a simple operation can rapidly lead to a lot of gates. For example, let's consider the addition of two number: for two numbers of $n$ bits, the total number of gates is $5n$ in a full adder composition. This is even worse for multiplication. Of course, some optimisations can be made, but the general number of gates is very high compared to the arithmetic circuit of the same function, where it would just be one single gate for addition and multiplication. We will see later that MPC over arithmetic circuits has a much higher \emph{round complexity} --- the dependence of the different rounds on each other --- which leads to less possible parallelisation than with boolean circuits. Nevertheless, one can argue that the parallelisation of bit-wise decomposition does not compensate the much higher number of gates and is thus less efficient than arithmetic circuits in general \cite{Aly2018PracticallyBit-Decomposition}.

+ general comparison of bit-wise decomposition performance and additive secret sharing comparison \cite{Blom2014AThesis}.

Comparisons are much more feasible in boolean circuits.

\subsubsection{How to share a secret}
The building block of multi-party computation over arithmetic is based upon secret sharing. Each party computes its own version of the circuit with the shares of the different parties. At the end of the circuit processing, each party has a share of the final output, which can then be put together to obtain the final output. We first have to define a way for a party to share its secret among $n$ parties, including itself.

\paragraph{Additive secret sharing}
The simplest idea is just to divide the secret $a$ in $n$ shares $a_i$ using a simple summation: $a = \sum_{i=1 \ldots n}a_i$. However, doing it in this manner allows the shares to release some information about the secret, as the shares are not random and strongly depend on the secret. They are two solutions to this problem and the first one is to consider additive sharing over $\mathbb{Z}_q$. The sharing now becomes $a = \sum_{i=1\ldots n}a_i \mod q$ which solves the problem, as the shares can now really be chosen at random. The other solution is over $\mathbb{Z}$ and consists in choosing a sufficiently large interval in which the shares are chosen to dilute sufficiently the statistical information about the secret, typically $a_i \in \left[-A2^\rho,A2^\rho\right]$ with $A$ the size of the interval of the secret $a \in \left[-A,A\right]$ and typically $\rho=128$.

Another consequence of this scheme is that all parties are vital to the recovery of secret as a loss of one secret unables us to reconstruct the secret or any statistical information about it as we took care of that. The scheme does not tolerate the loss or treason of one party and is therefore very sensitive to any failure or malicious player. This problem is solved by polynomial secret sharing.



\paragraph{Polynomial secret sharing}
The idea of polynomial sharing was originally proposed by Shamir in 1979 \cite{Shamir1979HowSecret}. This method allows $n$ parties to share a secret in a way such that any subset of $t+1$ parties can later reconstruct the secret but any subgroup of maximum $t$ parties can do so. The scheme is based on the single fact that for any polynomial of degree $d$, any subset of $d+1$ or more different points can reconstruct the polynomial completely whereas any subset of at most $d$ points is left with an infinite number of possibilities.

The scheme goes as follows. The party that wants to share its secret first constructs a polynomial of degree $t$.
\begin{equation}
    h(z) = a + \sum_{i=1}^t b_i z^i
\end{equation}
with secret $a$ random coefficients $b_i$. For the same reasons as for additive secret sharing, the coefficients can either be chosen in $b_i \in \mathbb{Z}_q$ (which also leads to the consideration of polynomial $h(z) \mod q$ instead), either in the interval $b_i \in \left[-A2^\rho,A2^\rho\right]$.
\noteH{should I demonstrate ?}

We verify that $h(0)=a$ and distribute shares $a_i$ to each party $i$ --- including ourselves --- as follows $a_i=h(i)$. $t+1$ parties can now reconstruct the polynomial together using e.g. Lagrange's polynomials and compute $f(0)$ to recover the secret share. An interesting property is that the recovery can be done as a simple linear combination. Indeed, we have
\begin{eqnarray*}
    h(0) &=& \sum_{i=1}^{n} l_i(0)h(i) \\
    a &=& \sum_{i=1}^{n} r_ia_i
\end{eqnarray*}
where $r = \left(r_1, \ldots , r_n\right) = \left(l_1(0), \ldots , l_n(0)\right)$ is called the \emph{recombination vector} with $l_i(z)$ the $i$-th order Lagrange polynomial, for example
\begin{equation*}
    l_i(z) = \prod_{1\leq j \leq n,j\neq i}\frac{z-j}{i-j}
\end{equation*}
This also works with any generating set of $\mathbb{Z}\left[z\right]$ (or alternatively $\mathbb{Z}_q\left[z\right]$), as long all parties use the same set.

\subsubsection{Performing basic operations}

\paragraph{Addition and multiplication of public constants}
Each party just adds or multiplies his share with the constant. This rests on some arithmetic properties of polynomials: one can easily verify that $a_i+c$ is a share of $a+c$ and $c \cdot a_i$ of $c \cdot a$.

\paragraph{Addition of two secrets}
Let's consider the addition of two polynomials: the respective coefficients just add up. By consequence, two secrets $a$ and $b$ can be added up if every party adds their local shares $a_i+b_i$.
\noteH{False, the values $h(i)$ are added up, not the coefficients}


\paragraph{Multiplication of two secrets}
Unfortunately, it is not possible to adopt the same strategy for multiplication as the multiplication of two polynomials of order $t$ will lead to a new polynomial of order $2t$ which will double the number of parties needed to recover the secret output. In real-case functions with a lot of multiplications, this becomes rapidly impracticable and theoretically unsolvable if the degree of the new polynomial exceeds $n$. Another problem is of statistical order: the new polynomial is not random anymore as it is for example not irreducible anymore by construction. Different algorithms exist to solve this problem \noteH{cite different algorithms ?}

\paragraph{Exponentiation}


\paragraph{Boolean operations}
\section{The SCALE-MAMBA framework}

\subsection{Security of the model}
Semi-honest model + some explanations obout the active case.


\subsubsection{Passive case}



\subsubsection{Active case}


\subsection{Key performance indicators}
Now that we know when a comparison protocol is secure in the Semi honest model, we should discuss when such protocols are considered “good”. This will not be done by defining a single attribute to be good, rather we pick key performance indicators (KPI) so we get a transparent view of the performance different protocols might display in different settings.

\subsubsection{Communication Complexity}
The total communication needed by a protocol can be measured by the number of Kilo Bytes (KB) of data which have to be send over a communication line during a protocol. Obviously, a protocol which enforces 1 GB = 10242 KB of communication is not considered a good protocol when compared to another protocol which only needs 100 KB of communication data. Usually, the communication performance of a protocol depends greatly on the bit-length of the input. So in order to keep the performance evaluation fair, one should make sure to optimize the total communication as good as possible considering a given input length ? for the protocol.

\subsubsection{Round Complexity}
In many applications, the number of communication rounds tends to be the bottle neck of a protocol. This is due to the fact that a lot of overhead capacity might be needed to initiate and terminate a (secure) communication line between the other party. In general, however, this really depends on the communication technology used in practise. We use Toft’s definition of a round of communication, because it gives us a workable and intuitive definition of this concept. Toft argues that a communication round consists of sending information to other parties and performing a limitless number of arithmetic computations with a sole restriction: variables which are received by a party during this round can not be used in any arithmetic operation performed by that party in the same round.

\subsubsection{Bandwidth}
The bandwidth of a protocol is given by the maximum number of bits send during a single communication round. This performance indicator strikes an interesting issue consider the previous two KPI. When the communication complexity of a protocol doesn’t change, but we manage to decrease the round complexity, then odds are that the necessary bandwidth for the protocol will increase. This is obviously not always the case, but shows that we might encounter some trade off considering these KPI. Note that when one finds the necessary bandwidth of a protocol to hight in practise, one can always implement the rounds with the highest communication complexity complexity as two separate communications to decrease the bandwidth.

\subsubsection{Computational Complexity}
This KPI is mentioned way less in recent literature than the previous ones. Which is one of the main reasons why this research project exists in the first place. The Computational complexity is measured as the amount of time it takes ones CPU to compute the desired result of a protocol. This KPI is shunned so often as it takes a lot of extra time and effort to implement the proposed protocol. To our knowledge, the only secure comparison protocol, described in the next chapter, for which the computation complexity was studied previous to our research, was that of Garbled Circuits.

\chapter{Experimental results and discussion}
\label{cha:4}

\section{Methodology}
This chapter aims at evaluating the different algorithms and the possible feature and instance set reductions possible to make them suitable for a practically feasible secure usage. Each algorithm is first tested with all data reduction methods to assess which trade-offs can be made on the classification performance. Finally, they are tested in a MPC setting to observe how the reductions reduce their computational, round and communication costs.

To investigate the classification performance, we use different indicators based on the classification \emph{confusion matrix}, also called \emph{error matrix}. This matrix represents the results of the classification. Each column stands for the number of occurrences of a real class, whereas the lines stand for the observed classes. If the classifier is perfect, all estimated classes should correspond to the real ones and the classification matrix should only have non-zero elements on its diagonal. In the specific case of binary classifiers (class/non-class), the elements of the classification matrix often are referred to as true positives ($f_p$), true negatives ($t_n$), false positives ($f_p$) and false negatives ($f_n$):
\begin{equation}
    \begin{pmatrix}
    t_p & f_n \\
    f_p & t_n
    \end{pmatrix}.
\end{equation}

Still, we would like to have some classification performance indicators that are one-dimensional for the ease of the comparison of the different methods. From the confusion matrix, we can build different classifiers. Of course, it is never possible to build a perfect indicator as the reduction of the confusion matrix to one dimension is inevitably accompanied by some information loss. The first one is the accuracy that represents the proportion of well-classified elements
\begin{equation}
    \mathtt{Accuracy} = \frac{t_p + t_n}{t_p + t_n + f_p + f_n},
\end{equation}
and can be generalized in multi-class systems (the trace of the confusion matrix divided by its total number of elements). It's maximum is always 1 and minimum always 0. The accuracy indicator has a few limitations and the most notable one is its lack of sensibility to the initial distribution. Let's imagine a binary system where the first class is very well classified and comprises the majority of the proportion of the test set. The other class is highly misclassified but represents a very limited proportion of the test set. The accuracy would be high as most of the data (the first class) is well classified, though the other class is not and almost doesn't count in the accuracy indicator, whatever its performance.

This problem can be solved by considering \emph{Matthew's correlation coefficient}~\cite{Matthews1975ComparisonLysozyme} that is defined as
\begin{equation}
    \mathtt{MCC} = \frac{t_p \times t_n - f_p \times f_n}{\sqrt{(t_p + f_p)(t_p+f_n)(t_n+f_p)(t_n+f_n)}}.
\end{equation}

This indicator can be interpreted as the correlation between predictions and observations, hence its name. For multi-class models, the MCC will be computed as the mean for all the classes. This makes sense as the indicator doesn't depend on the order of the class (e.g. in the binary case: the indicator is the same whatever is considered as the positive or negative class). Its value always range from -1 for the worst case to 1 forst the best case.

A last indicator will be considered: \emph{Cohen's kappa coefficient} measures the difference between the measured classifier and a random classifier. More specifically, it is given by
\begin{equation}
    \mathtt{Kappa} = \frac{\mathtt{Accuracy}-p_e}{1-p_e},
\end{equation}
where $p_e$ represents the accuracy of a random classifier and is given by
\begin{equation}
    p_e = \frac{p(t_p+f_p) + n(f_n+t_n)}{(t_p+t_n+f_p+f_n)^2},
\end{equation}
where $p$ and $n$ are the proportion of positive and negatives occurrences. For multi-class models, it can be generalized by considering $t_p$ (resp. $t_n$, $f_p$ and $f_n$) as the sum of the same $t_p$ (resp. $t_n$, $f_p$ and $f_n$) for each individual binary classifier~\cite{doi:10.1177/001316446002000104}. Here again, it gives an answer to the problem of unequal classes distributions. In a binary system where a class is much more present than the other, the value for $p_e$ will be high as the random classifier will assign each element with a high probability of being the much more present class. In the extreme case of all the elements being in the same class, the random classifier will assign each element to that only class. Its respective probability would be 1 and $\mathtt{Kappa}$ 0. Though, the accuracy would be 100\%.

All experiments were done on a 3,3GHz Intel Core i7 dual-core (Skylake) processor with 16GB RAM computer. The machine-learning classification performance experiments have been done using MATLAB R2018a.
\section{The NSL-KDD dataset}
Big table explaining the different features.

Categorical data into numerical. Why frequency and not num for 3. Why bin for 2 and 4.
\newpage
\section{Support Vector Machines}
Let us first benchmark the support vector machines and see what improvements (reductions) can be made to increase their evaluation time in an MPC setting.

\subsection{Linear support vector machines}
Figure~\ref{fig:svm-l} shows the results of a single support vector machine with a linear kernel function. A first observation is that the tree-based classifier does not perform worse that the one-against-all model. Indeed, the order of the SVMs in the tree-based model is very important as it can make it perform worse or better than one-against-all model. In this sense, the best tree-based structure does perform better and is supposed to have a faster evaluation using MPC as it comprises one SVM less.

Secondly, as we can see, the results are quite satisfying. The accuracy is high, nevertheless, the results must be nuanced. Indeed, as discussed before, the accuracy doesn't take into account the initial distribution of the classes. This is very important if the initial distribution the classes is not uniform, as in our case. We therefore must have a more detailed look at the results (table~\ref{tab:svm-l-1}, more results for other training set sizes are given in appendix~\ref{app:lsvm}).

\begin{figure} [h!]
        \begin{subfigure}[b]{1\textwidth}  
            \centering 
            \includegraphics[width=.98\textwidth]{parts/chap-4/img-svm/lin-svm-1.png}
            %\caption{Mean Accuracy on the test set.} 
        \end{subfigure}
        %\vfill
        %\begin{subfigure}[b]{1\textwidth}   
        %    \centering 
        %    \includegraphics[width=.98\textwidth]{parts/chap-4/img-svm/lin-svm-2.png}
        %    \caption{Mean Mathews correlation coefficient.} 
        %\end{subfigure}
        %\vfill
        %\begin{subfigure}[b]{1\textwidth}   
        %    \centering 
        %    \includegraphics[width=.98\textwidth]{parts/chap-4/img-svm/lin-svm-3.png}
        %    \caption{Mean Cohen's kappa coefficient.} 
        %\end{subfigure}
        \caption[Comparison of LSVM models.]{Evaluation of three different models in function of the training set size. The one-against-all model is in dash-dotted line, the tree-bases model are the plain and dotted line. For the plain line, the order of the SVMs is \{Normal, DoS, Prob, R2L, U2R\} and the dotted line is \{Probe, U2R, R2L, DoS, Normal\}. Every result is the mean of 5 different experiments with different training and test set.}
        \label{fig:svm-l}
\end{figure}

\begin{table}[h!]
    \centering
    \begin{tabularx}{\textwidth}{lXXXXXX}
    \hlineI
    Model & Normal & Probe & DoS & R2L & U2R & Total \\ \hlineI
    \textbf{Tree 1} with $n=30,000$ & & & & & &\\
    Accuracy [\%] & 91.37 & 97.40 & 96.34 & 94.07 & 73.33 & 94.62\\ 
    MCC [\%] & 89.00 & 96.03 & 93.15 & 85.20 & 67.38 & 86.15\\
    Kappa [\%] & 27.06 & 42.34 & 42.25 & 93.59 & 99.66 & 83.19\\ \hline
    Obs. Normal  & 2741 & 63 & 136 & 55 & 5 & \\ 
    Obs. Probe  & 53 & 2195 & 2 & 3 & 0 & \\ 
    Obs. DoS  & 76 & 5 & 2171 & 1 & 0 & \\ 
    Obs. R2L  & 13 & 0 & 0 & 213 & 0 & \\ 
    Obs. U2R  & 2 & 0 & 0 & 1 & 9 & \\  \hlineI
    
    \textbf{Tree 2} with $n=30,000$ & & & & & &\\
    Accuracy [\%] & 92.56 & 97.11 & 96.18 & 93.27 & 66.67 & 95.12\\ 
    MCC [\%] & 89.84 & 96.82 & 92.39 & 87.16 & 73.46 & 89.12\\ 
    Kappa [\%] & 26.35 & 42.70 & 42.14 & 93.79 & 99.72 & 84.74\\ \hline 
    Obs. Normal  & 2777 & 29 & 151 & 42 & 2 & \\ 
    Obs. Probe  & 56 & 2189 & 9 & 0 & 0 & \\ 
    Obs. DoS  & 80 & 6 & 2168 & 0 & 0 & \\ 
    Obs. R2L  & 14 & 1 & 0 & 211 & 0 & \\ 
    Obs. U2R  & 0 & 0 & 0 & 4 & 8 & \\ \hlineI
    
    \textbf{O-A-A} with $n=30,000$ & & & & & &\\
    Accuracy [\%] & 93.03 & 96.67 & 96.53 & 94.60 & 70 & 94.62\\ 
    MCC [\%] & 90.13 & 96.84 & 92.81 & 88.31 & 77.52 & 87.93\\ 
    Kappa [\%] & 26.07 & 42.96 & 42.05 & 93.78 & 99.72 & 84.12\\ \hline
    Obs. Normal  & 2791 & 18 & 150 & 40 & 1 & \\ 
    Obs. Probe  & 71 & 2179 & 4 & 0 & 0 & \\ 
    Obs. DoS  & 70 & 8 & 2176 & 0 & 0 & \\ 
    Obs. R2L  & 12 & 0 & 0 & 214 & 0 & \\ 
    Obs. U2R  & 0 & 0 & 0 & 3 & 8 &\\ \hlineI
    \end{tabularx}
    \caption[Detailed comparison of LSVM models.]{Detailed results of the linear SVM classification algorithm for different multi-class models. The first one is a tree-based model of order \{Normal, DoS, Prob, R2L, U2R\}, the second of order \{Probe, U2R, R2L, DoS, Normal\} and the last one a one-against-all model. Every result if the mean of 5 independent experiments.}
    \label{tab:svm-l-1}
\end{table}

A first observation is that the very low number of U2R and in a lesser extent of R2L instances in the training set size lead to a not very satisfying classification. The data of this class is not able to be classified well, this leads to a low MCC coefficient. However, the wrong results are mainly attributed to the normal class. This makes sense as the low number of instances makes the classifier not really able to generalize the properties of this class, not being able to recognize it well and thus unable to distinguish from a normal instance. However, the fact that the false negatives being mainly attributed to the normal class and not randomly assigned to the other classes explains the high kappa coefficient. This example justifies the use of those two coefficients together as they are complementary in this example. Not much can be done to solve this issue except massively augmenting the presence of the U2R --- and in a lesser way R2L --- presence in the training data-set. This low result is also to be nuanced as their scarce appearance is also an indication for their low frequency in real-life cases. We can thus conclude that the model doesn't detect much of these attacks, but hopefully they are scarce. 

A second observation is the very low kappa coefficient for the normal class. Checking the details indicates that the normal instance identified as attacks are proportionally distributed among the other classes, which translates into a much higher Matthews correlation coefficient (MCC). In other words, misclassified normal classes don't tend to be identified as one class specifically above another.

A third observation is that the results get better with the training size, which is an expected result in machine learning. However, for small training sizes the one-against-all model performs better than both tree-based models. This is to be nuanced due to the small training size which can lead to a much higher variance in the results.

Overall, there are two key observations The fist one is that one-against-all models are not better than tree-based models, it just depends on the order of the tree. The second one is that training sets larger than 30,000 don't make a lot of difference anymore. These facts certainly matters as the number of support vectors is suspected to increase with the training set size, but it is to be nuanced as linear SVMs only depend on the feature size in their primal form and not on the number of support vectors. However, this will have an impact when the evaluation is done in the dual, e.g. RBFSVMs --- but this is a talk for section~\ref{sec:4-non-lin-svm}.

\subsubsection{PCA reduction}

\begin{wrapfigure}[17]{r}{0.45\textwidth}
\begin{center}
    \includegraphics[width=.45\textwidth]{parts/chap-4/img-svm/pca-var.png}
    \caption{Variance participation of the 11 first components.}
    \label{fig:pca-var}
\end{center}
\end{wrapfigure}
Let us now investigate how a principal components decomposition affects the accuracy and allows us to win execution time. The variance contribution of the first principal components is given in figure~\ref{fig:pca-var}. As the variance contribution is not drastically decreasing, this plots indicates that most of the features are relevant and not so much of them are due to linear combinations of the others features. This also suggests by consequence that we will not be able to limit ourselves to a projection into a space of very small dimension. The influence of a varying number of principal components retained --- which corresponds to the dimension of the projected space --- is given in figure~\ref{fig:svm-pca}. We thereby conclude that we cannot limit ourselves to 6 features as a elbow rule\footnote{The number of principal compenents retained after which the variance gain becomes marginal.} would suggest, but that we need more of them, e.g. 16. The more detailed results for 16 components retained are given in table~\ref{tab:pca-2} (more details to be found in appendix~\ref{app:lsvm-pca}). Here again, we can conclude that there is no significant difference between the tree-based model and the one-against-all model. 

\begin{figure}[h!]
        \begin{subfigure}[b]{1\textwidth}  
            \centering 
            \includegraphics[width=.98\textwidth]{parts/chap-4/img-svm/pca-acc.png}
            %\caption{Mean of the accuracy on the test set.} 
        \end{subfigure}
        %\vfill
        %\begin{subfigure}[b]{1\textwidth}   
        %    \centering 
        %    \includegraphics[width=.98\textwidth]{parts/chap-4/img-svm/pca-kappa.png}
        %    \caption{Mean of Cohen's kappa coefficient.} 
        %\end{subfigure}
        \caption[Comparison of PCA-LSVM models.]{Evaluation of three different models in function of number of principal components retained. The one-against-all (or parallel) model is in dash-dotted line, the tree-bases model (or parallel) are the plain and dotted line. For the plain line, the order of the SVMs is \{Normal, DoS, Prob, R2L, U2R\} and the dotted line is \{Probe, U2R, R2L, DoS, Normal\}. Every result is the mean of 5 different experiments with different training and test sets.}
        \label{fig:svm-pca}
\end{figure}

\begin{table}[th!]
    \centering
    \begin{tabularx}{\textwidth}{lXXXXXX}
    \hlineI
    Model & Normal & Probe & DoS & R2L & U2R & Total \\ \hlineI
    \textbf{Tree} with $n_{pca}=16$ & & & & & &\\
    Accuracy [\%] & 91.33 & 94.94 & 94.19 & 96.11 & 33.33 & 93.24\\ 
    MCC [\%] & 87.13 & 95.35 & 90.77 & 79.68 & $\emptyset$ & $\emptyset$\\ 
    Kappa [\%] & 26.76 & 43.51 & 42.83 & 93.25 & 99.62 & 78.86 \\  \hline
    Obs. Normal  & 2740 & 12 & 150 & 89 & 8 & \\ 
    Obs. Probe  &98 & 2142 & 16 & 1 & 0 & \\ 
    Obs. DoS  & 100 & 22 & 2125 & 8 & 1 & \\ 
    Obs. R2L  & 8 & 0 & 0 & 208 & 0 & \\ 
    Obs. U2R  & 6 & 0 & 0 & 4 & 5 & \\   \hlineI
    
    \textbf{O-A-A} with $n_{pca}=16$ & & & & & &\\
    Accuracy [\%] & 91.11 & 95.55 & 95.63 & 95.09 & 18.67 & 93.31\\ 
    MCC [\%] & 88.16 & 95.62 & 91.18 & 80.59 & 25.99 & 76.31\\ 
    Kappa [\%] & 27.08 & 43.23 & 42.09 & 93.42 & 99.74 & 80.29\\  \hline
    Obs. Normal  & 2733 & 24 & 162 & 80 & 1 & \\ 
    Obs. Probe  & 72 & 2156 & 26 & 3 & 0 & \\ 
    Obs. DoS  & 80 & 15 & 2157 & 2 & 1 & \\ 
    Obs. R2L  & 10 & 0 & 1 & 205 & 0 & \\ 
    Obs. U2R  & 6 & 0 & 0 & 6 & 3 & \\  \hlineI
    \end{tabularx}
    \caption[Detailed comparison of PCA-LSVM models]{Detailed results of the linear SVM classification algorithm for different multi-class models with PCA decomposition and $n_{pca}=16$ components retained. The first one is a tree-based model of order \{Normal, DoS, Prob, R2L, U2R\} and the second one a one-against-all model. Every result if the mean of 5 independent experiments.}
    \label{tab:pca-2}
\end{table}

Now, could we only perform one of the operations in a secure manner (PCA or SVM) and do the other one in clear? Let's investigate the consequences of doing this:
\begin{itemize}
    \item \textbf{Secure PCA - Clear SVM.} The first idea of doing the PCA reduction using MPC and evaluating the SVM in clear is not very good. It is very naive to think that PCA destructs the data in such a way that the transformed feature lose most of their information. By transforming using a principal component analysis, we aim at the exact opposite which is to keep as much variance possible, thus as much information as possible. By looking here above at the results of the classifier after the PCA reduction, we notice that the PCA reduction does approach the same classification performance as without. Using clear SVM has also the results that everyone that participates in the evaluation of the transformed data-point can see its classification result. 
    
    The clear SVM evaluation should therefore be limited to the strict minimum parties, i.e. the model owner sending the SVM-model to the query owner which then performs the classification on its own. In this way, his data-point is always hidden, but the model owner reveals a part of his model. However, this model part is unusable in itself as it needs the PCA reduction, which is still hidden.

    The other possibility to keep the entire model (PCA and SVM) hidden is that the query owner sends its query after PCA to the model owner for him to perform the classification. The owner of the model data, once he knows the transformation of the query point, cannot recover it completely. One can notice than computing the inverse of the PCA reduction will lead to some noisy pre-image in the sense that the pre-image is a not a single data-point but a distribution with variance corresponding to the variance list with the PCA reduction. To summarize, this will of course reveal the final output (the eventual attack and its type), but somewhat hide the features about the query (e.g. duration or status of the connection). This justifies the preference for the model owner sending the SVM-model to the query owner which then performs the classification on its own. In this scenario, the owner however keeps all its model protected.
    
    Finally, one should add that this whole idea of computing the principal component analysis in MPC and then the SVM in clear is computationally doubtful: the MPC-based PCA-transformation of the query point demands $\mathcal{O}\left(n_{pca}d\right)$ MPC multiplication operations where $n_f$ is the feature size and $n_{pca}$ the number of principal components retained, here more than 6 for reasonable performance. Evaluating multi-class support vector machines without PCA and totally hidden has a complexity of $\mathcal{O}(n_{svm}d)$ MPC multiplication operations where $n_{svm}$ is the number of SVMs in the multi-class model, here 4 or 5 depending on which model structure is used. Once the PCA reduction is complete, you still have to add the designation of a winner between the different binary SVMs output which takes $\mathcal{O}(n_{svm})$ MPC comparison operations which are known to be more expensive than multiplications. The question is if the fact that $n_{pca} > n_{svm}$ is compensated by the hidden election of a winner.
    
    \item \textbf{Clear PCA - Secure SVM.}
    This alternative has the same property as before, the model is still partially hidden. Furthermore, the hidden evaluation of the SVM guarantees the privacy of the final classification output. In the contrary to here above, the question of who evaluates the PCA decomposition has only one possibility, the user performing the query: the opposite would reveal the initial feature vector to someone else and loose all the privacy of the queried point.
    
    The first question if we reveal the PCA components is what information is revealed. Well, not much as this is the result of the only diagonalization of the correlation matrix of the model owner's initial data-points. The correlation matrix intrinsically reveals the linear relations between different features. This would barely reveal anything about the original data-points as this relation is computed among ideally several thousands of points from all classes all together. Furthermore, this correlation matrix is diagonalized and the only first components are retained, revealing in a certain way the only important relations. With the hypothesis that each class has different subjacent relations and thus different principal components, an attacker could --- but this is going quite far --- deduce which classes were present in the model owner original data-set and thus deduce to which attack he is subject to, or at least detected. However, the exact nature of the original individual data-points remains secret and so does the queried data-point as it never leaves its owner in clear form.
    
    The advantage of this method is that performing an hidden SVM with pre-computed PCA reduction reduces the complexity from $\mathcal{O}(n_{svm}n_f)$ to $\mathcal{O}(n_{svm}n_{pca})$ and in the practice of a factor between 2 and 4. However, the designation of the winning SVM remains $\mathcal{O}(n_{svm})$.
    
    \item \textbf{No PCA - Clear SVM.} In this case, the query owner receives the entire model from the model owner. Of course, not much here is still hidden. In fact, all the model is clear an can be used by anybody who receives it and there is no need at all for MPC techniques. However linear SVM have the big advantage against the other ones, using kernel trick, that it doesn't need the dual to be computed. This means that instead of sending the $\alpha_i$ with their corresponding support vectors $x_i$ and targets $l_i$ (which will be of course a huge breach of privacy-friendliness), one can just send the weights vector $w$ (and the bias $b$), which reveals much less information about the model. The exact relation between the weights and the support vectors is 
    \begin{equation}
        w = \sum_{\forall i}\alpha_i y_i x_i.
    \end{equation}
    
    The bias stays the same. The computation of the weights is of course a very big compression of the support vectors and destroys the information they contain. In this sense, a totally clear model could be an option, depending on the choices of the model owner. This has no competitor considering the execution speed among the models we test. Indeed, avoiding the use of MPC leads to drastic speed improvements. There is however a drawback less related to privacy-friendliness but is still relevant: the model owner loses control over his model. Indeed, the big advantage of using MPC on a significant part of the model is that the other celar parts of the model are unusable without the MPC-part, whose parameters are never revealed. In this sense, the owner of the data still has a full control as he is the only one who can ultimately decide whether or not his model can be used. Once all is in clear, this property disappears and there is no more secure lock against the proliferation of the models use without its consent. This can be relevant for commercial uses for example\footnote{Besides, this raises an interesting question on how to secure software against illegal proliferation and use. Instead of using license keys, why just not performing a very small, but essential task in MPC without which the program could possibly work (in an information-theoretic scheme). In this sense, the owner is sure that only identified users --- which he can control at each moment --- are allowed to use the program. This of course would need to rest on a permanent internet connection (which is nowadays less an issue, more and more smartphone apps are requiring a permanent internet connection) and avoiding the MPC task being cracked and substituted by local version. This is a totally out of the scope of my thesis, but I believe the question to be interesting. Of course, this idea could also be used without MPC and just data sent in an encrypted form, but this would not be privacy-friendly for the user. Another question would also be: who should the eventual third party be to avoid any malicious majority (the first party being the user and the second the company). The third party would be a party that has no interest in working with the user and thus cracking the program nor with the firm issuing the program revealing the user's data. This could be replaced by homomorphic encryption to avoid the use a of third party but is computationally more costly.}.
\end{itemize}


\subsubsection{$\chi^2$-reduction}
Where the PCA decomposition is the same for all binary SVM, the $\chi^2$ feature selection allows us to select the most relevant features for each SVM. This allows more specifically pre-processed classifiers. Here, the big advantage is a reduction of complexity from $\mathcal{O}(n_{svm}n_f)$ to $\mathcal{O}(n_{svm}n_{\chi^2})$ where $n_{\chi^2}$ s the number of features retained. In our case, the reduction results in a reduction of 30\% of the feature inputs and thus an expected similar speed gain. Table~\ref{tab:svm-l-chi2} shows the execution of the three models and we can see that there is no significant loss of accuracy compared to the other models. However, the training of a $\chi^2$ selection is much slower because it requires to train $n_f$ support vector machines per binary classification instead of one. However, as we said, the evaluation time got a speed increase. In other words, we won at evaluating time at the cost of training time. The good part is that the training is noramlly only done once, conversely to the MPC-evaluation which is proportionally much more costly. This is a clear example of the training-evaluating trade-off we try to take advantage of.

\begin{table}[h!]
    \centering
    \begin{tabularx}{\textwidth}{lXXXXXX}
    \hlineI
    Model & Normal & Probe & DoS & R2L & U2R & Total \\ \hlineI
    \textbf{Tree} with $\chi^2$ and $n=30,000$ & & & & & &\\
    Accuracy [\%] & 91.57 & 97.83 & 96.24 & 92.89 & 87.50 & 94.79\\
    MCC [\%] & 89.41 & 97.00 & 92.74 & 83.52 & 61.82 & 84.90\\ 
    Kappa [\%] & 26.93 & 42.13 & 42.00 & 93.97 & 99.69 & 83.73\\ \hline
    Obs. Normal & 2747 & 37 & 146 & 61 & 9 & \\ 
    Obs. Probe & 45 & 2211 & 3 & 1 & 0 & \\ 
    Obs. DoS  & 75 & 10 & 2175 & 0 & 0 & \\ 
    Obs. R2L  & 14 & 0 & 1 & 196 & 0 & \\ 
    Obs. U2R  & 1 & 0 & 0 & 0 & 7 & \\  \hlineI
    
    \textbf{O-A-A} with $\chi^2$ and $n=30,000$ & & & & & &\\
    Accuracy [\%] & 92.20 & 97.17 & 96.55 & 91.47 & 50 & 94.89 \\
    MCC [\%] & 89.93 & 97.12 & 92.15 & 83.73 & 63.22 & 85.23 \\
    Kappa [\%] & 26.57 & 42.54 & 41.67 & 94.08 & 99.83 & 83.93 \\ \hline
    Obs. Normal & 2766 & 24 & 155 & 54 & 1 & \\
    Obs. Probe & 43 & 2196 & 20 & 1 & 0 & \\ 
    Obs. DoS  & 74 & 4 & 2182 & 0 & 0 & \\
    Obs. R2L  & 17 & 0 & 1 & 193 & 0 & \\ 
    Obs. U2R  & 1 & 0 & 2 & 1 & 4 & \\  \hlineI
    \end{tabularx}
    \caption[Detailed comparison of $\chi^2$-LSVM models.]{Detailed results of the linear SVM classification algorithm for different multi-class models with $\chi^2$ feature selection. The first one is a tree-based model of order \{Normal, DoS, Prob, R2L, U2R\} and the second one a one-against-all model. Every result if the mean of 5 independent experiments.}
    \label{tab:svm-l-chi2}
\end{table}

All these experiments also allow us to gain some insight on the relevance of each input feature. Figure~\ref{fig:features-part} shows the relevance of each input feature based on the different methods we tested. We can for example see that some features are very relevant while others not at all. For example, the proportion of connections to the same IP address or the type of protocol used are very relevant for the classification while the number of shell prompts or the consistency between IP addresses seem not very relevant.

\begin{figure}[h!]
        \begin{subfigure}[b]{1\textwidth}  
            \centering 
            \includegraphics[width=.98\textwidth]{parts/chap-4/img-svm/pca-features-3.png}
            \caption{Mean of the 6 first PCA coefficients.} 
        \end{subfigure}
        \vfill
        \begin{subfigure}[b]{1\textwidth}   
            \centering 
            \includegraphics[width=.98\textwidth]{parts/chap-4/img-svm/pca-features-3-bis.png}
            \caption{Mean of the first 16 PCA coefficients.} 
        \end{subfigure}
        \vfill
        \begin{subfigure}[b]{1\textwidth}   
            \centering 
            \includegraphics[width=.98\textwidth]{parts/chap-4/img-svm/chi2-features.png}
            \caption{Mean of the $\chi^2$ measure on all the model's SVMs.} 
        \end{subfigure}
        \caption[Feature relevance using the $\chi^2$-measure on linear SVMs.]{Feature relevance using the $\chi^2$-measure on linear SVMs.}
        \label{fig:features-part}
\end{figure}

A last thing, performing a $k$-means clustering algorithm isn't very interesting in the case of support vector machines as the sole parameter influencing the evaluation time are the support vectors close to the decision boundary, by essence of support vector machines. The number of data-points far from the decision boundary --- which we try to reduce with a $k$-means algorithm --- has thus no influence on the number of support vectors. The only eventual gain of reducing the data-set size with a $k$-means algorithm is to reduce training time, but this is not much of our concern here.


\subsubsection{Secure evaluation}
We can now estimate the different models according to the 4 performance indicators described in section~\ref{sec:perf} (figure~\ref{fig:eval-lsvm}). As observed in the previous section, $n=30,000$ samples are a good value for the training set size as the learning performance doesn't increase significantly for higher values. All the models tested are thus based on this value. The models tested are
\begin{itemize}
    \item \textbf{(P)LSVM:} secure linear support vector machine;
    \item \textbf{(P)PCA8-(P)LSVM:} secure linear support vector machine with secure principal component analysis decomposition with 8 principal components retained;
    \item \textbf{(P)PCA16-(P)LSVM:} secure linear support vector machine with secure principal component analysis decomposition with 16 principal components retained;
    \item \textbf{(P)PCA8-(C)LSVM:} clear linear support vector machine with secure principal component analysis decomposition with 8 principal components retained;
    \item \textbf{(P)PCA16-(C)LSVM:} clear linear support vector machine with secure principal component analysis decomposition with 16 principal components retained;
    \item \textbf{(P)PCA8-(C)LSVM:} secure linear support vector machine with clear principal component analysis decomposition with 8 principal components retained;
    \item \textbf{(P)PCA16-(C)LSVM:} secure linear support vector machine with clear principal component analysis decomposition with 16 principal components retained;
    \item \textbf{$\chi^2$-(P)LSVM:} secure linear support vector machine with $\chi^2$ feature selection.
\end{itemize}

\begin{figure}[h!]
        \begin{subfigure}[b]{.49\textwidth}  
            \centering 
            \includegraphics[width=.98\textwidth]{parts/chap-4/img-svm/lsvm-timing/acc.png}
            \caption{Classification performance.} 
        \end{subfigure}
        \hfill
        \begin{subfigure}[b]{.49\textwidth}   
            \centering 
            \includegraphics[width=.98\textwidth]{parts/chap-4/img-svm/lsvm-timing/rounds.png}
            \caption{Round cost.} 
        \end{subfigure}
        \hfill
        \begin{subfigure}[b]{.49\textwidth}   
            \centering 
            \includegraphics[width=.98\textwidth]{parts/chap-4/img-svm/lsvm-timing/time.png}
            \caption{Computational cost.} 
        \end{subfigure}
        \hfill
        \begin{subfigure}[b]{.49\textwidth}   
            \centering 
            \includegraphics[width=.98\textwidth]{parts/chap-4/img-svm/lsvm-timing/comm.png}
            \caption{Communication cost.} 
        \end{subfigure}
        \caption[Comparison of the different LSVM models using MPC.]{Comparison of different protocols for secure linear support vector machine evaluation with and without various feature size reduction methods. The black bars correspond to the one-against-all multi-class model and the white ones to the tree-based multi-class model. The results correspond here of the secret evaluation of 10 queries.}
        \label{fig:eval-lsvm}
\end{figure}

As predicted we can observe that performing a secure PCA decomposition together with a secure support vector machine is not very interesting. The accuracy is worse and we increase all cost indicators. 

Similarly, performing a secure PCA decomposition using a clear SVM is not interesting considering the accuracy and the computational cost. The big advantage is that the clear computation of support vector machines avoids the need for secure comparisons (when evaluating the multi-class model) and drastically reduces the number of rounds and communication cost. This could maybe be interesting for very high-latency or slow networks, though the advantage seems relatively limited considering the costs including the revelation of the SVM model parameters.

As predicted again, the clear PCA decomposition together with secure SVM evaluation seems an interesting alternative. The need for comparisons still results in a high round cost, however the lower dimension of the feature vector drastically reduces the computational and the communication cost. However, the accuracy is still lower than the full model and the PCA principal components may reveal (limited) information about the training set. This model could be considered if absolute secrecy about the training isn't required as some statistical information about it could be deduced from the PCA coefficients. For the $n_{pca}=16$, the accuracy is not as good as the full model, but not significantly lower.

The best model here seems to be the $\chi^2$ feature selection with full secrecy. The accuracy is as good as the full model if not better and there is a significant reduction of the computational and communication cost. The round cost is similar as the other models due to the need for comparisons.

Furthermore and as predicted, the tree-based model seems to perform in general as well as the one-against-all model, but allows for improvements in all other indicators, the most significant being the round cost.

To summarize the case of secure linear support vector machines for intrusion detection systems, the model to be preferred is the tree-based model with $\chi^2$ feature selection.

%%%%%%%%%%%%%%%%%%%%%%%%%%%%%%%%%%%%%%%%%%%%%%%%%%%%%%%%%%%%%%%%%%%%%%%%%%%%%%%%%%%%%%%%%%%%%%%%%%%%%%%%%%%%%%

\subsection{Non-linear support vector machines}
\label{sec:4-non-lin-svm}
Now that we have benchmarked the linear model, we can have a look at some non-linear models. Figure~\ref{fig:svm-nl} shows the results of a radial based function support vector machine on the NSL-KDD data-set. The box constraints $C$ and the kernel function parameter $\sigma^2$ are optimized at each specific training through a 10-fold cross-validation. 

A first observation compared to the linear support vector machines models is the much better accuracy. This is due to the added non-linearity which allows more complexity. The individual scores of the support vector machines are attaining values very close to 100\%. We can observe that accuracy and Cohen's kappa coefficient are following almost exactly the same graph, at the difference of a factor. This is typical when the models are attaining very high accuracies and make little mistakes. This comforts us in our claim of a good model.

A second observation is the same we made here before with the linear support vector machines: the tree-based model are performing worse than the one-against-all model. Here again, the accuracy doesn't increase much more after $n=15,000$. A difference with before is the better performance of the other sequence of the tree-based model, even though the difference is minimal. Results for the best tree-based model and the one-against-all model are given in table~\ref{tab:svm-nl} (more in appendix~\ref{app:rbfsvm}).

\begin{figure}[h!]
        \begin{subfigure}[b]{1\textwidth}  
            \centering 
            \includegraphics[width=.98\textwidth]{parts/chap-4/img-svm/svm-nl-acc.png}
            %\caption{Mean Accuracy on the test set.} 
        \end{subfigure}
        %\vfill
        %\begin{subfigure}[b]{1\textwidth}   
        %    \centering 
        %    \includegraphics[width=.98\textwidth]{parts/chap-4/img-svm/svm-nl-kappa.png}
        %    \caption{Mean Cohen's kappa coefficient.} 
        %\end{subfigure}
        \caption[Comparison of RBFSVM models.]{Evaluation of three different models in function of the training set size. The one-against-all (or parallel) model is in dash-dotted line, the tree-bases model (or parallel) are the plain and dotted line. For the plain line, the order of the SVMs is \{Normal, DoS, Prob, R2L, U2R\} and the dotted line is \{Probe, U2R, R2L, DoS, Normal\}. Every result is the mean of 5 different experiments with different training and test set.}
        \label{fig:svm-nl}
\end{figure}

\begin{table}[h!]
    \centering
    \begin{tabularx}{\textwidth}{lXXXXXX}
    \hlineI
    Model & Normal & Probe & DoS & R2L & U2R & Total \\ \hlineI
    \textbf{Tree 1} with $n=15,000$ & & & & & &\\
    Accuracy [\%] & 97.82 & 99.08 & 99.01 & 91.71 & 24.00 & 98.20\\ 
    MCC [\%] & 96.62 & 99.03 & 98.54 & 87.47 & 32.25 & 82.78\\ 
    Kappa [\%] & 23.65 & 42.23 & 42.14 & 93.71 & 99.69 & 94.39\\  \hline
    Obs. Normal  & 2935 & 7 & 21 & 33 & 5 & \\ 
    Obs. Probe  & 18 & 2229 & 2 & 0 & 0 & \\ 
    Obs. DoS  & 19 & 2 & 2228 & 1 & 0 & \\ 
    Obs. R2L  & 16 & 1 & 1 & 215 & 1 & \\ 
    Obs. U2R  & 5 & 0 & 0 & 6 & 4 & \\   \hlineI
    
    \textbf{Tree 2} with $n=15,000$ & & & & & &\\
    Accuracy [\%] & 97.98 & 98.25 & 99.20 & 91.45 & 28.00 & 98.08\\ 
    MCC [\%] & 96.23 & 98.44 & 98.56 & 89.30 & $\emptyset$ & $\emptyset$\\ 
    Kappa [\%] & 23.46 & 42.56 & 42.04 & 93.85 & 99.72 & 94.00\\  \hline
    Obs. Normal  & 2939 & 8 & 24 & 26 & 2 & \\ 
    Obs. Probe  & 36 & 2211 & 4 & 0 & 0 & \\ 
    Obs. DoS  & 17 & 1 & 2232 & 0 & 0 & \\ 
    Obs. R2L  & 19 & 1 & 0 & 214 & 0 & \\ 
    Obs. U2R  & 7 & 0 & 0 & 3 & 4 & \\  \hlineI
    
    \textbf{O-A-A} with $n=15,000$ & & & & & &\\
    Accuracy [\%] & 98.33 & 99.40 & 99.15 & 92.14 & 24.00 & 98.55\\ 
    MCC [\%] & 97.28 & 99.41 & 98.84 & 88.54 & 33.55 & 83.52\\ 
    Kappa [\%] & 23.38 & 42.13 & 42.14 & 93.76 & 99.72 & 95.47\\   \hline 
    Obs. Normal  & 2950 & 4 & 15 & 30 & 1 & \\ 
    Obs. Probe  & 12 & 2237 & 1 & 0 & 0 & \\ 
    Obs. DoS  & 18 & 1 & 2231 & 0 & 0 & \\ 
    Obs. R2L  & 15 & 1 & 0 & 216 & 2 & \\ 
    Obs. U2R  & 5 & 0 & 1 & 5 & 4 & \\ \hlineI
    \end{tabularx}
    \caption[Detailed comparison of RBFSVM models.]{Detailed results of the RBF-SVM classification algorithm for different multi-class models for $n=15,000$. The first one is a tree-based model of order \{Normal, DoS, Prob, R2L, U2R\} and the second one a one-against-all model. Every result if the mean of 5 independent experiments.}
    \label{tab:svm-nl}
\end{table}

\subsubsection{Support vector reduction}
A difference with linear support vector machines is that the use of a kernel matrix doesn't allow us to evaluate a new data-point in the primal space anymore. This primal space had the dimension of the feature space and the training size had not a lot of influence on the evaluation of a new query point as a consequence. However, this is not the case anymore and the evaluation now happens in a support vector space. As a reminder, the evaluation of a new data-point $x$ in the SVM is now given by
\begin{equation}
    \sum_{i=1}^{n_{sv}} \alpha_i y_i K(x,x_i),
\end{equation}
where $K(x,x_i)$ is the kernel function and $n_{sv}$ the number of support vectors. For each of our models trained before, the number of support vectors is given in figure~\ref{fig:svm-nl-sv}. The number of support vectors is much higher for the one-against-all model as suspected as it has one more support vector machine than the tree-based model. However, an interesting observation is that both tree-based models --- although having the same number of support vector machines --- have a significant divergence in the total number of support vectors. Furthermore, it is the best of both models that comprises the less number of support vectors. In this latter case, the total number of support vectors never exceeds significantly a thousand. Let's however see if we can reduce this number further.

\begin{figure}[h!]
    \centering
    \includegraphics[width=1\textwidth]{parts/chap-4/img-svm/svm-nl-sv.png}
    \caption[Number of support vectors for different models.]{Total number of support vectors for different models based on the training set size. The one-against-all (or parallel) model is in dash-dotted line, the tree-bases model (or parallel) are the plain and dotted line. For the plain line, the order of the SVMs is \{Normal, DoS, Prob, R2L, U2R\} and the dotted line is \{Probe, U2R, R2L, DoS, Normal\}. Every result is the mean of 5 different experiments with different training and test sets.}
    \label{fig:svm-nl-sv}
\end{figure}

As said before, $C$ is optimized through validation. This directly controls the number of support vectors. To improve the speed of the evaluation, one must thus reduce the number of support vectors. This can be controlled by the box constraint $C$ which --- as a reminder --- represents the trade-off between the objective of a support vector machine and the influence of the slack variables --- we could call the corresponding data-points the "difficult" ones. In this sense, a high value of $C$ will just make the boundary absolutely fit to every variable and tolerate no wrongly classified data-point to the model. In other words, the difficult data-points will have a much higher influence. This is even more clear in the dual as the box constraint $C$ directly represents an upper bound on the $\alpha_i$. A high value of $C$ leads to less regularization and the model to fit to these specific difficult points, which increases the risk of overfitting. Let's investigate how far we can increase the box constraint $C$ without suffering from overfitting. This way, we can reduce the MPC evaluation without having too much impact on the accuracy.

Figure~\ref{fig:svm-nl-red} shows how the number of support vector decreases as the box constraint parameter $C$ increases. The result of the last SVM shows more variability as the low number of data-points of the classes R2L and U2R which it classifies. In general, we can observe that the after a certain value, the number of support vectors doesn't decrease anymore and the accuracy, which is almost perfect, stagnates. However, we see that we aren't really subject to overfitting. The stagnation of the number of support vectors and the absence of ovefitting tells us that all data-points are within the good side of the boundary. In this sense, imposing the misclassified ones to fit to the boundary has no effect, hence the stagnation. In a certain sense, this corresponds to the ideal boundary. What is more, the very high box constraint $C$ also diminishes the impact of the first term of the boundary, which controls its smoothening. However, finding the optimal $C$ without imposing it to be high already naturally tends to a high value. This indicates that the optimal boundary without the constraint of lowering the number of support vectors is not smooth. In this sense, the boundary is only defined by the critical data-points. These sole critical data-points are the only ones the support vectors consists of. This is also the reason for the stagnation of their number.

\begin{figure}[h!]
        \begin{subfigure}[b]{.47\textwidth}  
            \centering 
            \includegraphics[width=.98\textwidth]{parts/chap-4/img-svm/non-lin/svm1.png}
            \caption{First SVM.} 
        \end{subfigure}
        \hfill
        \begin{subfigure}[b]{.47\textwidth}   
            \centering 
            \includegraphics[width=.98\textwidth]{parts/chap-4/img-svm/non-lin/svm2.png}
            \caption{Second SVM.} 
        \end{subfigure}
        \hfill
        \begin{subfigure}[b]{.47\textwidth}   
            \centering 
            \includegraphics[width=.98\textwidth]{parts/chap-4/img-svm/non-lin/svm3.png}
            \caption{Third SVM.} 
        \end{subfigure}
        \hfill
        \begin{subfigure}[b]{.47\textwidth}   
            \centering 
            \includegraphics[width=.98\textwidth]{parts/chap-4/img-svm/non-lin/svm4.png}
            \caption{Fourth SVM.} 
        \end{subfigure}
        \caption[Support vector reduction.]{Support vector reduction for the tree-based model with SVM order \{Normal, DoS, Prob, R2L, U2R\} with $n=15,000$ data-points.}
        \label{fig:svm-nl-red}
\end{figure}

By taking a very high value of $C$ to reduce the number of support vectors, we still obtain an accuracy similar to before (table~\ref{tab:svm-nl-1} compared to table~\ref{tab:svm-nl}). However the number of support vectors went from 931 with the classically validated $C$ to 588 with a high value of $C$. This is a decrease of 37\% and we can expect a similar increase in time. For the other tree-based model, the number of support vectors went from 908 to 643 representing a decrease of 29\% For the one-against-all model, it went from 1393 to 1031 representing a decrease of 35\%.

\begin{table}[h!]
    \centering
    \begin{tabularx}{\textwidth}{lcccccccc}
    \hlineI
    Model &&& Normal & Probe & DoS & R2L & U2R & Total \\ \hlineI
    \multicolumn{9}{l}{\textbf{Tree} with $n=15,000$ and support vector reduction}\\
    Accuracy [\%] &&& 98.69 & 99.71 & 99.25 & 90.77 & 74 & 98.89\\ 
    MCC [\%] &&& 97.93 & 99.51 & 99.00 & 90.79 & 63.39 & 90.13\\ 
    Kappa [\%] &&& 23.16 & 41.78 & 41.92 & 94.32 & 99.70 & 96.54 \\ \hline
    Obs. Normal  &&& 2961 & 7 & 11 & 18 & 4 & \\ 
    Obs. Probe && & 3 & 2249 & 4 & 0 & 0 & \\ 
    Obs. DoS && & 16 & 1 & 2239 & 0 & 0 & \\ 
    Obs. R2L && & 17 & 1 & 0 & 201 & 2 & \\ 
    Obs. U2R && & 2 & 0 & 0 & 1 & 7 & \\  \hlineI
    
     \multicolumn{9}{l}{\textbf{O-A-A} with $n=15,000$ and support vector reduction}\\
    Accuracy [\%] &&& 98.72 & 99.69 & 99.34 & 92.04 & 74 & 98.96\\ 
    MCC [\%] &&& 98.07 & 99.61 & 98.98 & 91.60 & 63.42 & 90.34\\ 
    Kappa [\%] &&& 23.16 & 41.81 & 41.87 & 94.29 & 99.69 & 96.75\\    \hline 
    Obs. Normal && & 2962 & 4 & 13 & 17 & 4 & \\ 
    Obs. Probe && & 4 & 2249 & 3 & 0 & 0 & \\ 
    Obs. DoS && & 14 & 1 & 2241 & 0 & 0 & \\ 
    Obs. R2L && & 14 & 0 & 1 & 203 & 2 & \\ 
    Obs. U2R && & 1 & 0 & 1 & 1 & 7 & \\  \hlineI
    \end{tabularx}
    \caption[Detailed comparison of RBFSVM models with support vector reduction.]{Detailed results of the RBF-SVM classification algorithm for different multi-class models for $n=15,000$ with support vector reduction. The first one is a tree-based model of order \{Normal, DoS, Prob, R2L, U2R\} and the second one a one-against-all model. Every result if the mean of 5 independent experiments.}
    \label{tab:svm-nl-1}
\end{table}

\subsubsection{PCA with support vector reduction}
Each tested data-point has to be computed against all support vectors. Now that the number of support vectors has been reduced, let us investigate if we can also reduce the evaluation of each support vector in itself. As a reminder, the radial based kernel function is given by
\begin{equation}
    K(x,y) = e^{\frac{\norm{x-y}^2}{2\sigma^2}},
\end{equation}
and the norm is given by
\begin{equation}
    \norm{x-y} = \sum_{j=1}^{n_f} (x_j - y_j)^2,
\end{equation}
where $n_f$ is the number of features.

By applying a PCA decomposition, we would be able to approximate the norm and limit the sum to $n_{pca}$ terms instead of $n_f$ terms. The evaluation against each support vector would then be reduced. The cost of doing this is the need to transform each tested data-point through PCA. As we saw before this evaluation is of complexity $\mathcal{O}(n_{pca}n_f)$.  Contrary to before, this is far smaller than the complexity needed to evaluate the SVM due to the much higher number of support vectors. The need for evaluating one of the two in clear is thus not justified anymore. Indeed, evaluating the SVM-part in clear would reveal the feature vectors which are highly sensitive data and the evaluation of the PCA-part in clear is significantly less costly than the SVM. We will thus only interest us in evaluating the whole (PCA and SVM) in MPC.

Furthermore, the norm in the feature space gives the same weight to each feature. Using a PCA decomposition will vary the weight as the principal components of the transformed feature are a scalar product of the original feature vector with the PCA coefficients, which can be seen as weights. 

Results of the PCA decomposition with RBF-SVM are given in figure~\ref{fig:svm-nl-pca}. Compared to the PCA reduction with linear support vector machines, a lower number of principal components retained is feasible here (8 gave not satisfying results with the SVMs with linear kernel, and seems here satisfying). A low number of principal components seems also to impose a higher number of support vectors. This makes sense as the more information loss during the PCA reduction has to be compensated by keeping more information in the SVM under the form of support vectors. Table~\ref{tab:svm-nl-pca} shows the results with 16 principal components. We can clearly see that the performance is as good as without any PCA reduction.

\begin{figure}[h!]
        \begin{subfigure}[b]{.97\textwidth}  
            \centering 
            \includegraphics[width=.98\textwidth]{parts/chap-4/img-svm/nl-pca/acc.png}
            %\caption{Accuracy in function of the number of principal components.} 
        \end{subfigure}
        %\vfill
        %\begin{subfigure}[b]{.97\textwidth}  
        %    \centering 
        %    \includegraphics[width=.98\textwidth]{parts/chap-4/img-svm/nl-pca/mcc.png}
        %    \caption{Matthews correlation coefficient in function of the number of principal components.} 
        %\end{subfigure}
        %\vfill
        %\begin{subfigure}[b]{.97\textwidth}  
        %    \centering 
        %    \includegraphics[width=.98\textwidth]{parts/chap-4/img-svm/nl-pca/kappa.png}
        %    \caption{Cohen's kappa coefficient in function of the number of principal components.} 
        %\end{subfigure}
        %\begin{subfigure}[b]{.97\textwidth}  
        %    \centering 
        %    \includegraphics[width=.98\textwidth]{parts/chap-4/img-svm/nl-pca/sv.png}
        %    \caption{Total number of support vectors in function of the number of principal components.} 
        %\end{subfigure}
        \caption[Comparison of different PCA-RBFSVM models.]{Influence of the number of principal components retained in a RBF-SVM with support with support vector reduction and PCA reduction. The one-against-all (or parallel) model is in dash-dotted line, the tree-bases model (or parallel) are the plain and dotted line. For the plain line, the order of the SVMs is \{Normal, DoS, Prob, R2L, U2R\} and the dotted line is \{Probe, U2R, R2L, DoS, Normal\}. Every result is the mean of 5 different experiments with different training and test set.}
        \label{fig:svm-nl-pca}
\end{figure}

\begin{table}[h!]
    \centering
    \begin{tabularx}{\textwidth}{lcccccccc}
    \hlineI
    Model &&& Normal & Probe & DoS & R2L & U2R & Total \\ \hlineI
    \multicolumn{9}{l}{\textbf{Tree} with $n=15,000$, $n_{pca}=16$ and support vector reduction}\\
    Accuracy [\%] &&& 98.31 & 99.55 & 99.00 & 97.40 & 54.29 & 98.81\\ 
    MCC [\%] &&& 97.74 & 99.42 & 98.65 & 92.45 & 57.14 & 89.08\\ 
    Kappa [\%] &&& 23.38 & 41.75 & 41.88 & 94.17 & 99.83 & 96.28\\   \hline
    Obs. Normal  &&& 2949 & 7 & 17 & 23 & 3 & \\ 
    Obs. Probe  &&& 4 & 2249 & 3 & 3 & 0 & \\ 
    Obs. DoS  &&& 20 & 1 & 2236 & 2 & 0 & \\ 
    Obs. R2L  &&& 5 & 0 & 0 & 209 & 0 & \\ 
    Obs. U2R  &&& 3 & 0 & 0 & 1 & 4 & \\  \hlineI
    
     \multicolumn{9}{l}{\textbf{O-A-A} with $n=15,000$, $n_{pca}=16$ and support vector reduction}\\
    Accuracy [\%] &&& 98.58 & 99.64 & 99.01 & 95.72 & 45.71 & 98.89\\ 
    MCC [\%] &&& 97.83 & 99.56 & 98.71 & 93.47 & 42.85 & 86.48\\ 
    Kappa [\%] &&& 23.20 & 41.73 & 41.89 & 94.33 & 99.80 & 96.52\\    \hline 
    Obs. Normal && & 2957 & 5 & 16 & 16 & 6 & \\ 
    Obs. Probe && & 4 & 2251 & 3 & 1 & 0 & \\ 
    Obs. DoS && & 21 & 1 & 2237 & 1 & 0 & \\ 
    Obs. R2L && & 9 & 0 & 0 & 206 & 0 & \\ 
    Obs. U2R && & 4 & 0 & 0 & 0 & 3 & \\  \hlineI
    \end{tabularx}
    \caption[Detailed comparison of different RBFSVM models with PCA reduction.]{Detailed results of the RBF-SVM classification algorithm for different multi-class models for $n=15,000$ with support vector reduction and principal component analysis decomposition. The first one is a tree-based model of order \{Normal, DoS, Prob, R2L, U2R\} and the second one a one-against-all model. Every result if the mean of 5 independent experiments.}
    \label{tab:svm-nl-pca}
\end{table}


\subsubsection{$\chi^2$ with support vector reduction}
Similarly, to limit the number of terms in the sum needed for the norm, we could also perform a $\chi^2$ feature selection as we did before. Computing the selected features with radial based kernel function support vector machines gives very similar results as with the linear one. However, although both methods are keeping 33 of the 43 features, these are not exactly the same. This indicates that some features have a more linear impact (linear kernel functions) on the output and some others have a more local impact (radial based kernels functions). For the rest, apart from a few exceptions, the retained inputs are the same, indicating their clear contribution independently how this contribution is taken into account. Figure~\ref{fig:svm-nl-chi2} shows the $\chi^2$ measure of each SVM of a tree-based model. It is very clear that the model with few training data obtains much lower scores.

Another detail is that a feasible cut-off value on the $\chi^2$ measure with RBFSVM is much higher than with linear SVMs. This indicates that the corresponding certainty\footnote{The $\chi^2$ measure used here is based on the classical $\chi^2$ test for categorical data which uses a p-value relevance estimation. Here, we cannot use this test directly as we don't have categorical data as feature inputs. In a certain sense, for each of our tests on each support vector machine, we have 4 input classes used ($t_p$, $t_n$, $f_p$, $f_n$) and two output classes (1 or 0). This corresponds to a $\chi^2$ with three degrees of freedom. The cut-off value chosen here (10) corresponds to a p-value of .018566. This is statistically satisfying. However, this is not the goal we pursue here. The only thing that matters is finding clever ways to improve the speed of the evaluation of a test-set through our MPC models. For the linear based SVM, the p-value was not satisfying (p > 0.5). However, it still allowed us to improve the speed without loosing in accuracy.} is much higher. But the most interesting part is that here again, reducing the number of feature inputs by pure selection allows us to keep the same high performance while reducing the model's complexity and evaluation time (table~\ref{tab:svm-nl-chi2}).

\begin{figure}[h!]
        \begin{subfigure}[b]{.97\textwidth}  
            \centering 
            \includegraphics[width=.98\textwidth]{parts/chap-4/img-svm/nl-chi2/svm3.png}
            \caption{First SVM of the model.} 
        \end{subfigure}
        \vfill
         \begin{subfigure}[b]{.97\textwidth}  
            \centering 
            \includegraphics[width=.98\textwidth]{parts/chap-4/img-svm/nl-chi2/svm2.png}
            \caption{Second SVM of the model.} 
        \end{subfigure}
        \vfill
         \begin{subfigure}[b]{.97\textwidth}  
            \centering 
            \includegraphics[width=.98\textwidth]{parts/chap-4/img-svm/nl-chi2/svm4.png}
            \caption{Third SVM of the model.} 
        \end{subfigure}
        \vfill
         \begin{subfigure}[b]{.97\textwidth}  
            \centering 
            \includegraphics[width=.98\textwidth]{parts/chap-4/img-svm/nl-chi2/svm1.png}
            \caption{Fourth SVM of the model.} 
        \end{subfigure}
        \caption[$\chi^2$ measure of each feature.]{$\chi^2$ measure for each feature in each SVM of the tree-based model with order \{Normal, DoS, Prob, R2L, U2R\}.}
        \label{fig:svm-nl-chi2}
\end{figure}

\begin{table}[h!]
    \centering
    \begin{tabularx}{\textwidth}{lcccccccc}
    \hlineI
    Model &&& Normal & Probe & DoS & R2L & U2R & Total \\ \hline
    \multicolumn{9}{l}{\textbf{Tree} with $n=15,000$, $n_{pca}=16$ and support vector reduction}\\
    Accuracy [\%] &&& 98.55 & 99.74 & 99.14 & 94.14 & 56.92 & 98.87\\ 
    MCC [\%]  &&& 97.76 & 99.54 & 98.75 & 93.76 & 60.11 & 89.98\\ 
    Kappa [\%] &&& 23.25 & 41.81 & 41.96 & 94.26 & 99.67 & 96.48\\   \hline
    Obs. Normal  &&& 2957 & 6 & 19 & 14 & 5 & \\ 
    Obs. Probe && & 4 & 2249 & 2 & 0 & 0 & \\ 
    Obs. DoS && & 18 & 1 & 2236 & 0 & 0 & \\ 
    Obs. R2L && & 12 & 1 & 0 & 209 & 0 & \\ 
    Obs. U2R && & 6 & 0 & 0 & 0 & 7 & \\ \hlineI
    
     \multicolumn{9}{l}{\textbf{O-A-A} with $n=15,000$, $n_{pca}=16$ and support vector reduction}\\
    Accuracy [\%] &&& 98.54 & 99.73 & 99.16 & 94.14 & 61.54 & 98.88\\ 
    MCC [\%] &&& 97.81 & 99.62 & 98.79 & 92.57 & 62.34 & 90.23\\ 
    Kappa [\%] &&& 23.27 & 41.83 & 41.96 & 94.19 & 99.67 & 96.50\\    \hline 
    Obs. Normal && & 2956 & 4 & 17 & 19 & 4 & \\ 
    Obs. Probe && & 4 & 2249 & 1 & 0 & 0 & \\ 
    Obs. DoS && & 17 & 1 & 2236 & 0 & 0 & \\ 
    Obs. R2L && & 11 & 1 & 2 & 209 & 0 & \\ 
    Obs. U2R && & 5 & 0 & 0 & 0 & 8 & \\  \hlineI
    \end{tabularx}
    \caption[Detailed comparison of different RBFSVM models with $\chi^2$ feature selection.]{Detailed results of the RBF-SVM classification algorithm for different multi-class models for $n=15,000$ with support vector reduction and $\chi^2$ feature selection. The first one is a tree-based model of order \{Normal, DoS, Prob, R2L, U2R\} and the second one a one-against-all model. Every result if the mean of 5 independent experiments.}
    \label{tab:svm-nl-chi2}
\end{table}

\subsubsection{Secure evaluation}
Smilarly to linear SVM models, we can now estimate the different models according to the 4 performance indicators (figure~\ref{fig:eval-rbfsvm}). As observed above, $n=15,000$ samples are a good value for the training set size as the learning performance doesn't increase significantly for higher values. All the model tested are thus based on this value. The models tested are
\begin{itemize}
    \item \textbf{(P)RBFSVM:} secure support vector machine with radial based function kernel;
    \item \textbf{(P)PCA8-(P)RBFSVM:} secure support vector machine with radial based function kernel with secure principal component analysis decomposition with 8 principal components retained;
    \item \textbf{(P)PCA16-(P)RBFSVM:} secure support vector machine with radial based function kernel with secure principal component analysis decomposition with 16 principal components retained;
    \item \textbf{$\chi^2$-(P)RBFSVM:} secure support vector machine with radial based function kernel with $\chi^2$ feature selection.
\end{itemize}

\begin{figure}[h!]
        \begin{subfigure}[b]{.49\textwidth}  
            \centering 
            \includegraphics[width=.98\textwidth]{parts/chap-4/img-svm/rbfsvm-timing/acc.png}
            \caption{Classification performance.} 
        \end{subfigure}
        \hfill
        \begin{subfigure}[b]{.49\textwidth}   
            \centering 
            \includegraphics[width=.98\textwidth]{parts/chap-4/img-svm/rbfsvm-timing/rounds.png}
            \caption{Round cost.} 
        \end{subfigure}
        \hfill
        \begin{subfigure}[b]{.49\textwidth}   
            \centering 
            \includegraphics[width=.98\textwidth]{parts/chap-4/img-svm/rbfsvm-timing/time.png}
            \caption{Computational cost.} 
        \end{subfigure}
        \hfill
        \begin{subfigure}[b]{.49\textwidth}   
            \centering 
            \includegraphics[width=.98\textwidth]{parts/chap-4/img-svm/rbfsvm-timing/comm.png}
            \caption{Communication cost.} 
        \end{subfigure}
        \caption[Comparison of the different RBFSVM models using MPC.]{Comparison of different protocols for secure support vector machine with radial based kernel function evaluation with and without various feature size reduction methods. The black bars correspond to the one-against-all multi-class model and the white ones to the tree-based multi-class model.}
        \label{fig:eval-rbfsvm}
\end{figure}

Except for the PCA decomposition with 8 components, all models here have a very good accuracy ($\sim 98\%$) but are demanding much more evaluation time than their counterparts with linear kernel functions. This is caused by the fact briefly evoked here above: the non-linear support vector machines are evaluated in the dual space.

The feature space is not the only speed factor anymore as the evaluation of the kernel function also has a significant cost, but still plays a role.

The PCA reduction is now marginal to the distance computation with all support vectors and its Gaussian evaluation. This makes the PCA reduction with 16 components the clear winner here though the non-linear SVMs are not very satisfying in general. Similar accuracies can be achieved much faster using the nearest neighbors algorithm investigated here after.

\FloatBarrier



\section{Nearest neighbours}
\subsection{Classical nearest neighbours}
Figure~\ref{fig:knn} shows the results of the $k$-NN algorithm in function of the number of training data-points for a different number of neighbours. We can see that the performance is very similar to radial based function support vector machines.

A first observation is that at the contrary to SVMs models, the performance always keep increasing with added number of training points. This makes sense as support vector machines have a limited complexity due to their limited number of coefficients to be trained in their primal form. This is not totally true for SMVs using kernel functions and being evaluated in the dual, but their complexity is based on the number of support vectors which are consisting of only data-points close to the decision boundary. Adding more training points only improves the performance when those are close to the decision boundary if we neglect the smoothening of the boundary --- which in our case appears to be not very important. In $k$-NN algorithm, all data-points participate to the decision boundary. Augmenting the density of data-points allow the decision boundary to be influenced by any new data-point. The need for intricate boundaries --- or high complexity --- is also an explanation of the good performance of $k$-NN.

To understand the high variability of Matthew's correlation coefficient, we must have a more detailed look at the individual results (table~\ref{tab:knn-1} and \ref{tab:knn-2}). Matthew's correlation coefficient is computed on the binary correlation matrices for each class. For the total one, the mean is taken. As already mentioned before, the big advantage of this coefficient is that it takes into account the prior probabilities of both binary classes. This has a huge influence on the U2R class --- and in a lesser manner the R2L class --- that has very few instances. The accuracy is thus very good as very few points as classified in this category as should be for the very big majority, but the classifier is not very successful for the ones that should. As discussed before, not much can be done except from augmenting the instances of these classes.

The $k$-NN does as good as the best radial based function support vector machine for $n=10,000$ instances and surpasses all the previously tested algorithms for more training points, where this is very clear for $n=100,000$.

\begin{figure}[ht!]
        \begin{subfigure}[b]{.97\textwidth}  
            \centering 
            \includegraphics[width=.98\textwidth]{parts/chap-4/img-knn/knn-acc.png}
            \caption{Accuracy in function of the number of training points for a different number of neighbours.}
        \end{subfigure}
        \vfill
        \begin{subfigure}[b]{.97\textwidth}  
            \centering 
            \includegraphics[width=.98\textwidth]{parts/chap-4/img-knn/knn-mcc.png}
            \caption{Matthew's correlation coefficient in function of the number of training points for a different number of neighbours.} 
        \end{subfigure}
        \vfill
        \begin{subfigure}[b]{.97\textwidth}  
            \centering 
            \includegraphics[width=.98\textwidth]{parts/chap-4/img-knn/knn-kappa.png}
            \caption{Cohen's kappa coefficient in function of the number of training points for a different number of neighbours.} 
        \end{subfigure}
        \vfill
        \caption{Performance of the $k$-NN algorithm in function of the number of training data-points for a different number of neighbours. The plain line is for $k=1$, the dashed is $k=2$ and the dotted is $k=3$.}
        \label{fig:knn}
\end{figure}

\begin{table}[ht!]
    \centering
    \begin{tabularx}{\textwidth}{lcccccccc}
    \hlineI
    Model &&& Normal & Probe & Dos & R2L & U2R & Total \\ \hlineI
    \multicolumn{9}{l}{$k=1$ with $n=10,000$}\\
    Accuracy [\%] &&& 98.17 & 99.45 & 98.80 & 98.03 & 43.33 & 98.66\\ 
    MCC [\%] &&& 97.49 & 99.19 & 98.63 & 92.76 & 39.68 & 85.55\\ 
    Kappa [\%] &&& 23.48 & 41.80 & 42.04 & 94.07 & 99.74 & 95.81\\    \hline
    Obs. Normal  &&& 2945 & 11 & 15 & 25 & 4 & \\ 
    Obs. Probe  &&& 10 & 2246 & 2 & 1 & 0 & \\ 
    Obs. DoS  &&& 23 & 3 & 2231 & 1 & 1 & \\ 
    Obs. R2L  &&& 2 & 0 & 0 & 214 & 2 & \\ 
    Obs. U2R  &&& 3 & 0 & 0 & 2 & 4 & \\  \hlineI
    
    \multicolumn{9}{l}{$k=2$ with $n=10,000$}\\
    Accuracy [\%] &&& 98.26 & 99.37 & 98.26 & 95.93 & 30 & 98.42\\ 
    MCC [\%]  &&& 97.08 & 98.86 & 98.44 & 90.31 & 38.74 & 84.69\\ 
    Kappa [\%] &&& 23.24 & 41.49 & 42.03 & 94.82 & 99.75 & 95.07\\  \hline
    Obs. Normal  &&& 2948 & 17 & 9 & 24 & 2 & \\ 
    Obs. Probe && & 12 & 2252 & 1 & 1 & 0 & \\ 
    Obs. DoS && & 34 & 5 & 2227 & 1 & 0 & \\ 
    Obs. R2L && & 6 & 0 & 0 & 181 & 1 & \\ 
    Obs. U2R && & 3 & 0 & 0 & 5 & 4 & \\ \hlineI
    
    \multicolumn{9}{l}{$k=3$ with $n=10,000$}\\
    Accuracy [\%] &&& 97.05 & 98.90 & 98.80 & 97.81 & 55.00 & 98.08\\ 
    MCC [\%] &&& 96.36 & 98.63 & 98.31 & 89.11 & 39.73 & 84.43\\ 
    Kappa [\%] &&& 24.10 & 41.88 & 41.87 & 94.06 & 99.69 & 94.00\\    \hline 
    Obs. Normal && & 2912 & 15 & 22 & 43 & 10 & \\ 
    Obs. Probe && & 18 & 2236 & 5 & 1 & 0 & \\ 
    Obs. DoS && & 22 & 4 & 2234 & 1 & 0 & \\ 
    Obs. R2L && & 3 & 0 & 0 & 205 & 2 & \\ 
    Obs. U2R && & 3 & 0 & 0 & 1 & 4 & \\  \hlineI
    \end{tabularx}
    \caption{Detailed results of the $k$-NN classification algorithm for two different values of the number of neighbours $k$ and for a big and a small training data-set. The accuracy, true positive rate (TP), true negative rate (TN), false positive rate (FP) and false negative rate (FN) are given for each target class.}
    \label{tab:knn-1}
\end{table}

\begin{table}[ht!]
    \centering
    \begin{tabularx}{\textwidth}{lcccccccc}
    \hlineI
    Model &&& Normal & Probe & Dos & R2L & U2R & Total \\ \hlineI
    \multicolumn{9}{l}{$k=1$ with $n=100,000$}\\
    Accuracy [\%] &&& 99.56 & 99.82 & 99.63 & 93.81 & 58.33 & 99.48\\ 
    MCC [\%] &&& 99.00 & 99.73 & 99.57 & 95.39 & 62.16 & 91.17\\ 
    Kappa [\%] &&& 22.61 & 41.51 & 41.58 & 94.88 & 99.85 & 98.36\\  \hline
    Obs. Normal  &&& 2987 & 4 & 4 & 5 & 1 & \\ 
    Obs. Probe  &&& 2 & 2260 & 2 & 0 & 0 & \\ 
    Obs. DoS  &&& 8 & 0 & 2256 & 0 & 0 & \\ 
    Obs. R2L  &&& 12 & 0 & 0 & 190 & 1 & \\ 
    Obs. U2R  &&& 2 & 0 & 0 & 1 & 4 & \\   \hlineI
    
    \multicolumn{9}{l}{$k=2$ with $n=100,000$}\\
    Accuracy [\%] &&& 99.19 & 99.92 & 99.57 & 94.32 & 48.89 & 99.33\\ 
    MCC [\%]  &&& 98.64 & 99.72 & 99.35 & 95.07 & 65.47 & 91.65\\ 
    Kappa [\%] &&& 22.86 & 41.52 & 41.64 & 94.73 & 99.82 & 97.90\\   \hline
    Obs. Normal  &&& 2976 & 7 & 11 & 7 & 0 & \\ 
    Obs. Probe && & 2 & 2260 & 0 & 0 & 0 & \\ 
    Obs. DoS && & 9 & 1 & 2252 & 0 & 0 & \\ 
    Obs. R2L && & 12 & 0 & 0 & 194 & 0 & \\ 
    Obs. U2R && & 4 & 0 & 0 & 1 & 4 & \\ \hlineI
    
    \multicolumn{9}{l}{$k=3$ with $n=100,000$}\\
    Accuracy [\%] &&& 99.32 & 99.77 & 99.66 & 89.85 & 40.83 & 99.22\\ 
    MCC [\%] &&& 98.48 & 99.65 & 99.37 & 93.37 & 54.52 & 89.08\\ 
    Kappa [\%] &&& 22.71 & 41.52 & 41.52 & 95.11 & 99.77 & 97.56\\   \hline 
    Obs. Normal && & 2980 & 5 & 10 & 4 & 1 & \\ 
    Obs. Probe && & 4 & 2259 & 2 & 0 & 0 & \\ 
    Obs. DoS && & 8 & 0 & 2256 & 0 & 0 & \\ 
    Obs. R2L && & 19 & 1 & 0 & 177 & 0 & \\ 
    Obs. U2R && & 6 & 0 & 1 & 1 & 5 & \\  \hlineI
    \end{tabularx}
    \caption{Detailed results of the $k$-NN classification algorithm for two different values of the number of neighbours $k$ and for a big and a small training data-set. The accuracy, true positive rate (TP), true negative rate (TN), false positive rate (FP) and false negative rate (FN) are given for each target class.}
    \label{tab:knn-2}
\end{table}


\subsection{Training set reduction}
The training is linearly dependent on the number of instances in the training set --- the ones that are compared to to search for the nearest neighbours. To reduce the evaluation time, we must find a way to reduce this number and the find only the relevant data-points.

\subsubsection{$k$-means clustering}
\begin{wrapfigure}[13]{r}{0.45\textwidth}
\begin{center}
    \includegraphics[width=.45\textwidth]{parts/chap-4/img-knn/kmeans.png}
    \caption{Within cluster sum of distances in function of the number of clusters chosen.}
    \label{fig:kmeans}
\end{center}
\end{wrapfigure}
To select the number of clusters, the elbow rule suggests to take 5 clusters, which is by the way the same results obtained by \cite{Soheily-Khah2018IntrusionDataset}, but with a different data-set (figure~\ref{fig:kmeans}). We can then only keep the normal instances near the decision boundary, i.e. the cluster with the closest mean to the other classes. This gives very disappointing results indipendently of the number of nearest neighbours $k$ chosen. Table~\ref{tab:knn-kmeans} also shows the results where the sole most-distant cluster is discarded. However, the results are still very disappointing although logically better than keeping the sole best cluster, we can see that a lot of normal data-points are abusively considered as various attacks. The model lacks information about the variety of normal attacks, variety that we just supressed.
This suggests that they are no specific regions where the data-points are not relevant. In other words, the decision boundary goes through all clusters, all regions of the hyper-space where points are present. This once again confirms the complexity of the decision boundary and the problem in general. We could try to augment the number of clusters to discard smaller groups and do this iteratively always discard the worse one, but after all, the direction this is going to is select the points individually and based on group methods. This problem is better tackled with condensed nearest neighbours.

\begin{table}[ht!]
    \centering
    \begin{tabularx}{\textwidth}{lcccccccc}
    \hlineI
    Model &&& Normal & Probe & Dos & R2L & U2R & Total \\ \hlineI
    \multicolumn{9}{l}{$k=1$ with $n=100,000$}\\
    Accuracy [\%] &&& 95.19 & 99.29 & 98.78 & 96.92 & 76 & 97.45\\ 
    MCC [\%] &&& 94.88 & 97.05 & 98.29 & 89.79 & 74.85 & 90.97\\ 
    Kappa [\%] &&& 25.09 & 41.02 & 41.67 & 94.85 & 99.60 & 92.02\\   \hline
    Obs. Normal  &&& 2856 & 12 & 25 & 5 & 3 & \\ 
    Obs. Probe  &&& 84 & 2251 & 2 & 0 & 0 & \\ 
    Obs. DoS  &&& 23 & 4 & 2239 & 0 & 0 & \\ 
    Obs. R2L  &&& 34 & 1 & 0 & 179 & 0 & \\ 
    Obs. U2R  &&& 4 & 0 & 0 & 0 & 11 & \\   \hlineI
    
    \multicolumn{9}{l}{$k=2$ with $n=100,000$}\\
    Accuracy [\%] &&& 95.03 & 99.26 & 98.28 & 97.82 & 55.00 & 97.27\\ 
    MCC [\%]  &&& 94.59 & 97.16 & 97.86 & 89.90 & 33.75 & 82.65\\ 
    Kappa [\%] &&& 25.26 & 41.32 & 42.08 & 93.91 & 99.82 & 91.48\\    \hline
    Obs. Normal  &&& 2851 & 13 & 34 & 4 & 1 & \\ 
    Obs. Probe && & 73 & 2243 & 5 & 0 & 0 & \\ 
    Obs. DoS && & 27 & 3 & 2221 & 0 & 0 & \\  
    Obs. R2L && & 43 & 1 & 0 & 211 & 1 & \\ 
    Obs. U2R && & 6 & 0 & 0 & 1 & 2 & \\  \hlineI
    
    \multicolumn{9}{l}{$k=3$ with $n=100,000$}\\
    Accuracy [\%] &&& 86.61 & 98.75 & 98.54 & 98.14 & 72.22 & 93.94\\
    MCC [\%] &&& 87.98 & 93.01 & 96.85 & 82.97 & 61.41 & 84.44\\ 
    Kappa [\%] &&& 29.48 & 40.32 & 41.77 & 93.22 & 99.72 & 81.06\\    \hline 
    Obs. Normal && & 2598 & 22 & 25 & 2 & 2 & \\ 
    Obs. Probe && & 246 & 2229 & 4 & 0 & 0 & \\ 
    Obs. DoS && & 65 & 5 & 2224 & 0 & 0 & \\ 
    Obs. R2L && & 87 & 1 & 4 & 216 & 1 & \\
    Obs. U2R && & 4 & 0 & 0 & 2 & 7 & \\  \hlineI
    \end{tabularx}
    \caption{Detailed results of the $k$-NN classification algorithm for two different values of the number of neighbours $k$ and for a big and a small training data-set. The accuracy, true positive rate (TP), true negative rate (TN), false positive rate (FP) and false negative rate (FN) are given for each target class.}
    \label{tab:knn-kmeans}
\end{table}

\subsubsection{Condensed nearest neighbours}
The clustering-based methods give good results for huge data-sets and \cite{Soheily-Khah2018IntrusionDataset} has successfully applied it to \emph{random forest classifiers}. In our case, we have a lesser number of data-points and we do not try to discard the majority of them, but intelligently select the relevant ones to reduce our number of instances in the training data-set. The condensed nearest neighbours is an instance-wise selection method and not based on larger groups as are clustering methods. However, the better results of the $k=1$ does not allow us to discard the outliers. Table~\ref{} shows the results of the condensed nearest neighbours.

\begin{figure}[ht!]
        \begin{subfigure}[b]{.97\textwidth}  
            \centering 
            \includegraphics[width=.98\textwidth]{parts/chap-4/img-knn/cnn/acc.png}
            \caption{Accuracy with condensed nearest neighbours.}
        \end{subfigure}
        \vfill
        \begin{subfigure}[b]{.97\textwidth}  
            \centering 
            \includegraphics[width=.98\textwidth]{parts/chap-4/img-knn/cnn/mcc.png}
            \caption{Matthew's correlation coefficient with condensed nearest neighbours.} 
        \end{subfigure}
        \vfill
        \begin{subfigure}[b]{.97\textwidth}  
            \centering 
            \includegraphics[width=.98\textwidth]{parts/chap-4/img-knn/cnn/kappa.png}
            \caption{Cohen's kappa coefficient with condensed neighbours.} 
        \end{subfigure}
        \vfill
        \begin{subfigure}[b]{.97\textwidth}  
            \centering 
            \includegraphics[width=.98\textwidth]{parts/chap-4/img-knn/cnn/red.png}
            \caption{Number of data-points after reduction.} 
        \end{subfigure}
        \caption{Performance of the $k$-NN algorithm in function of the number of training data-points for a different number of neighbours. The plain line is for $k=1$, the dashed is $k=2$ and the dotted is $k=3$.}
        \label{fig:knn-cnn}
\end{figure}

\begin{table}[ht!]
    \centering
    \begin{tabularx}{\textwidth}{lcccccccc}
    \hlineI
    Model &&& Normal & Probe & Dos & R2L & U2R & Total \\ \hlineI
    \multicolumn{9}{l}{$k=1$ with $n=10,000$}\\
    Accuracy [\%] &&& 97.63 & 99.11 & 98.26 & 98.16 & 80 & 98.25\\ 
    MCC [\%] &&& 96.73 & 99.02 & 98.02 & 90.27 & 59.45 & 88.70\\ 
    Kappa [\%] &&& 23.74 & 41.94 & 42.19 & 93.92 & 99.70 & 94.52\\    \hline
    Obs. Normal  &&& 2929 & 14 & 31 & 3 & 1 & \\ 
    Obs. Probe  &&& 9 & 2238 & 1 & 1 & 0 & \\ 
    Obs. DoS  &&& 20 & 4 & 2219 & 0 & 0 & \\ 
    Obs. R2L  &&& 36 & 1 & 4 & 213 & 1 & \\ 
    Obs. U2R  &&& 6 & 0 & 3 & 0 & 6 & \\    \hlineI
    
    \multicolumn{9}{l}{$k=2$ with $n=10,000$}\\
    Accuracy [\%] &&& 93.67 & 94.48 & 78.75 & 68.56 & 36.67 & 88.81\\ 
    MCC [\%]  &&& 85.64 & 84.77 & 82.47 & 78.84 & 36.00 & 73.54\\ 
    Kappa [\%] &&& 24.58 & 40.61 & 48.94 & 95.17 & 99.86 & 65.03\\     \hline
    Obs. Normal  &&& 2810 & 44 & 238 & 66 & 2 & \\ 
    Obs. Probe && & 172 & 2135 & 241 & 0 & 0 & \\ 
    Obs. DoS && & 6 & 81 & 1780 & 0 & 0 & \\ 
    Obs. R2L && & 10 & 0 & 1 & 147 & 2 & \\ 
    Obs. U2R && & 2 & 0 & 0 & 1 & 2 & \\   \hlineI
    
    \multicolumn{9}{l}{$k=3$ with $n=10,000$}\\
    Accuracy [\%] &&& 94.37 & 90.02 & 88.40 & 86.43 & 36.47 & 91.01\\ 
    MCC [\%] &&& 90.39 & 85.97 & 85.30 & 85.11 & 51.17 & 79.59\\ 
    Kappa [\%] &&& 25.04 & 43.77 & 44.66 & 94.26 & 99.67 & 71.91\\    \hline 
    Obs. Normal &&& 2831 & 41 & 106 & 29 & 10 & \\ 
    Obs. Probe && & 101 & 2029 & 155 & 0 & 0 & \\ 
    Obs. DoS &&& 35 & 182 & 1993 & 0 & 0 & \\ 
    Obs. R2L &&& 32 & 2 & 1 & 191 & 0 & \\ 
    Obs. U2R &&& 1 & 0 & 0 & 1 & 6 & \\   \hlineI
    \end{tabularx}
    \caption{Detailed results of the $k$-NN classification algorithm for two different values of the number of neighbours $k$ and for a big and a small training data-set. The accuracy, true positive rate (TP), true negative rate (TN), false positive rate (FP) and false negative rate (FN) are given for each target class.}
    \label{tab:knn-cnn}
\end{table}

\subsection{Feature size reduction}

\subsubsection{Principal components analysis}



\subsubsection{$\chi^2$ feature selection}
%\section{Which algorithm choose ?}

\lipsum[1-3]

\chapter{Conclusion and future works}
Privacy-friendly machine learning algorithms for intrusion detection systems are appealing for network defense and more generally to the protection of private information.

Secure linear support vector machines offer solutions at a very low cost and good accuracy ($\sim 95\%$), where nearest neighbors algorithm offer a higher accuracy ($\sim 98\%$), but at high cost. The use of condensed nearest neighbors allows to drastically diminish that cost to make the algorithm practically usable. As for now, non-linear support vector machines didn't prove to be competitive against nearest neighbors methods using MPC.

Various feature reduction methods can be used to reduce the evaluation costs even more, such as PCA or $\chi^2$. When algorithms are fast, $\chi^2$ is to be preferred as it requires no pre-process. However, when the algorithms are slower and the pre-process is proportionally less expensive, PCA reduction is to be preferred as it reduces the feature vector into less dimensions for the same accuracy.

This research raises several questions for further work. One of them is whether the proposed algorithm can be improved to make them even faster. For example, it could be possible to increase the rapidity of the RBFSVM multi-class models by avoiding computing the kernel function with the same support vectors appearing multiple times (in the different SVMs of the multi-class model). However, we could predict that this is would not lead to a significant gain as the support vectors are by essence of a support vector machine, the samples near the decision boundaries. Though, these decision boundaries should be different for each class. Hence, few redundancy is expected.

Another improvement could be using more efficient algorithms, e.g. for the minimum selection phase of the nearest neighbors. The quickselect algorithm has been proposed, but presents a lot of concerns about index leakage and thus a breach in the privacy. However, solutions exist to exchange two indices of an array in a secret manner~\cite{Aly2014SecurelyProblems}.

In addition to algorithmic improvements, other dimensionality reduction algorithms exist that could maybe lead to further gain in execution cost, e.g. non-negative matrix factorization, kernel principal component analysis, linear discriminant analysis or canonical correlation analysis. Most of them have never been implemented in the (non-secure) case of intrusion detection systems. Kernel principal component analysis (KPCA) are an exception as they have been tested and deliver good results~\cite{Elkhadir2016IntrusionMethods}. Unfortunately, they are evaluated in a very similar manner to RBFSVMs and thus we expect to encounter the same problem: the high cost of computing the kernel matrix.

Further work could also include the investigation of training models directly using MPC instead of using models separately trained. This could allow different data-base owners to aggregate their data in a secret manner and to train a model on it. This way, we should be able to gain much more of the performances of distributed intrusion detection systems. A first step is the use ensemble classifiers where each model is trained on its own and the results are then aggregated using a (weighted) majority vote. This just requires to train the weights of the majority vote in a secret manner. Furthermore, it should allow the models to be more resistant against attacks that try to exploit the output of the model against specific queries trying to reconstruct some information about the data-base. Ensemble models would dilute the information obtained between different models.



\captionsetup[figure]{list=no}
\captionsetup[table]{list=no}
%\captionsetup[algorithm]{list=no}

\part*{Appendix}
\appendix
\chapter{Implementations}
\label{app:impl}
All the code used can be found at \url{https://github.com/hdeplaen/masterthesis-src}.

\section{MATLAB}
The MATLAB implementation has been constructed as a toolbox. The toolbox is built around the notion of \emph{experts}, which is an entire model as described in the thesis (one-against-all RBFSVMs or $k$-NN with consensed nearest neighbors). The different experts usable are:
\begin{itemize}
    \item \verb='knn'=: normal nearest neighbors;
    \item \verb='kmeans-knn'=: nearest neighbors with $k$-means instance set reduction;
    \item \verb='knn-cnn'=: consensed nearest neighbors;
    \item \verb='knn-cnn-chi2'=: condensed nearest neighbors with $\chi^2$ feature selection;
    \item \verb='par-svm'=: one-against-all (parallel) SVM multi-class model;
    \item \verb='seq-svm'=: tree-based (sequential) SVM multi-class model;
    \item \verb='par-svm'=: one-against-all (parallel) SVM multi-class model with $\chi^2$ feature selection;
    \item \verb='seq-svm'=: tree-based (sequential) SVM multi-class model with $\chi^2$ feature selection.
\end{itemize}

Here is a short list of the main functions that can be used:
\begin{itemize}
    \item \verb=load_data=: extraction of the data-set with conversion from categorical to numerical values (makes use of subroutines \verb=to_numeric=, \verb=cat2freq= and \verb=cat2bin=);
    \item \verb=normalize_data=: normalizes the data according to the values of the training set;
    \item \verb=bagging=: creates different training and test sets according to section~\ref{sec:dataset-bagging};
    \item \verb=pca_reduction=: performs the PCA reduction to the specified number of components;
    \item \verb=kmeans_clustering=: performs $k$-means clustering according to the training set;
    \item \verb=train_expert=: generates and expert model with training if required;
    \item \verb=eval_expert=: evaluates the expert according to the test set;
    \item \verb=plot_perf=: computes and eventually plots the performance of the classifier on the test set;
    \item \verb=export_expert=: exports the expert model for MPC-usage.
\end{itemize}

More information about each function (e.g. input and output arguments) can be found by typing \verb=help function=, where \verb=function= is the queried function.

Here follows a short example of the toolbox used to load, train, evaluate and export a condensed nearest neighbors expert with PCA reduction:
\begin{lstlisting}[language=Matlab]
%% PRELIMINARIES
k = 1;                  % number of nearest neighbors
n = 10000;              % number of elements in the training sets
num_bags = 5;           % number of experiments
disp_pca = false;       % don't plot the PCA componenets
expert = 'knn-cnn';     % type of expert used
data-set = 'nsl-kdd';   % data-set used

% preallocate the experiments results
corr = zeros(num_bags,5,5);
accm = zeros(num_bags,5);
mccm = zeros(num_bags,5);
kappam = zeros(num_bags,5);
acc = zeros(num_bags);
mcc = zeros(num_bags);
kappa = zeros(num_bags);
num_nb = zeros(num_bags);

%% GENERATE TRAINING AND TEST SETS
[trainX,trainY,testX,testY] = load_kdd(data_set,classes_red) ;
[BagTrainX,BagTrainY] = bagging(n(idxn), num_bags, trainX, trainY) ;
[BagTestX,BagTestY] = bagging(10000, num_bags, testX, testY) ;
        
% EXECUTE EACH EXPERIMENT
for idx_bag = 1:num_bags
    % extract the corresponding training set for this experiment
    locX = BagTrainX{idx_bag} ;
    locY = BagTrainY{idx_bag} ;
            
    % extract the corresponding test set for this experiment
    locXtest = BagTestX{idx_bag} ;
    locYtest = BagTestY{idx_bag} ;
            
    % normalize
    [locX,locY,locXtest,locYtest] = normalize_data(locX,locY,locXtest,locYtest) ;
            
    % PCA
    [trainX,testX] = pca_reduction(trainX,testX,n_pca,disp_pca) ;
            
    %% train and evaluate expert
    params_knn.k = k ;
    expert_knn = train_expert(locX,locY, expert, params_knn) ;
    eval_knn = eval_expert(expert_knn, locXtest, locYtest) ;
    export_expert(expert_knn, 'MyFirstExpert') ;
            
    [corr_, accm_, mccm_, kappam_, acc_, mcc_, kappa_] = plot_perf(eval_knn,locYtest) ;
    
    % store the results of each experiment
    corr(idx_bag,:,:) = corr_ ;
    accm(idx_bag,:) = accm_ ;
    mccm(idx_bag,:) = mccm_ ;
    kappam(idx_bag,:) = kappam_ ;
    acc(idx_bag) = acc_ ;
    mcc(idx_bag) = mcc_ ;
    kappa(idx_bag) = kappa_ ;
    num_nb(idx_bag) = expert_knn.num_nb ;
end
    
%% PLOT RESULTS
% mean of all experiments
corr = round(mean(corr(:,:,:),1));
accm = mean(accm(:,:),1);
mccm = mean(mccm(:,:),1);
kappam = mean(kappam(:,:),1);
acc = mean(acc(:),1);
mcc = mean(mcc(:),1);
kappa = mean(kappa(:),1);
num_nb = mean(num_nb(:),1);

% display the mean results
disp('k = 1') ;
disp('acc');
disp(squeeze(100*acc) ;
disp(squeeze(100*accm(:,:)));
disp('mcc') ;
disp(squeeze(100*mcc)) ;
disp(squeeze(100*mccm(:,:)));
disp('kappa') ;
disp(squeeze(100*kappa)) ;
disp(squeeze(100*kappam(:,:)));
disp('corr') ;
disp(squeeze(corr(:,:,:))) ;
\end{lstlisting}

\section{SCALE-MAMBA}
To compile and run the SCALE-MAMBA code, the framework first has to be installed. The installation and compilation are explained in the documentation~\cite{Aly2018SCALEDocumentation}. The programs have to be used with training data produced by the MATLAB code and place in the \verb=data= folder of each program. The following programs can be found:
\begin{itemize}
    \item \verb=knn_1=: Secure nearest neighbors evaluation with $k=1$;
    \item \verb=knn_chi2=: Secure nearest neighbors evaluation with $k=1$ and $\chi^2$ feature selection;
    \item \verb=knn_1_pca=: Secure nearest neighbors evaluation with $k=1$ and PCA reduction;
    \item \verb=knn_n=: Secure nearest neighbors evaluation with $k>1$;
    \item \verb=svm_lin=: Secure linear support vector machines evaluation;
    \item \verb=svm_lin_chi2=: Secure linear support vector machines with $\chi^2$ feature selection;
    \item \verb=svm_lin_pca=: Secure linear support vector machines with PCA reduction (the PCA or SVM evaluation can be done securely or in clear by commenting the corresponding lines);
    \item \verb=svm_rbf=: Secure support vector machines with radial based kernel function;
    \item \verb=svm_rbf_chi2=: Secure support vector machines with radial based kernel function with $\chi^2$ feature selection;
    \item \verb=svm_rbf_pca=: Secure support vector machines with radial based kernel function with PCA reduction.
\end{itemize}
\chapter{Additional experimental results}
\label{app:add}

\newpage
\section{Support Vector Machines}
\subsection{Linear SVMs}
\label{app:lsvm}
\begin{table}[h!]
    \centering
    \begin{tabularx}{\textwidth}{lXXXXXX}
    \hlineI
    Model & Normal & Probe & DoS & R2L & U2R & Total \\ \hlineI
    \textbf{Tree 1} with $n=2000$ & & & & & &\\
    Accuracy [\%] & 87.67 & 96.36 & 96.13 & 94.30 & 46.67 & 92.81\\ 
    MCC & 86.34 & 95.69 & 91.36 & 72.66 & 35.29 & 70.86\\ 
    Kappa & 29.17 & 42.85 & 41.90 & 92.27 & 99.66 & 70.88\\ \hline
    Obs. Normal  & 2630 & 38 & 175 & 145 & 12 & \\ 
    Obs. Probe  & 56 & 2172 & 20 & 5 & 0 & \\ 
    Obs. DoS  & 63 & 17 & 2167 & 6 & 1 & \\ 
    Obs. R2L  & 13 & 0 & 0 & 215 & 0 & \\ 
    Obs. U2R  & 0 & 0 & 0 & 5 & 4 & \\  \hlineI
    
    \textbf{Tree 2} with $n=2000$ & & & & & &\\
    Accuracy [\%] & 90.55 & 96.22 & 95.37 & 91.93 & 44.44 & 91.96\\ 
    MCC & 87.67 & 96.01 & 91.27 & 78.55 & 34.90 & 77.45\\ 
    Kappa & 27.41 & 43.01 & 42.32 & 93.14 & 99.70 & 74.04\\  \hline 
    Obs. Normal  & 2717 & 27 & 161 & 86 & 9 & \\ 
    Obs. Probe  & 65 & 2169 & 17 & 2 & 0 & \\ 
    Obs. DoS  & 86 & 14 & 2150 & 4 & 0 & \\ 
    Obs. R2L  & 17 & 1 & 0 & 210 & 1 & \\ 
    Obs. U2R  & 0 & 0 & 0 & 5 & 4 & \\ \hlineI
    
    \textbf{O-A-A} with $n=2000$ & & & & & &\\
    Accuracy [\%] & 91.70 & 95.41 & 93.40 & 92.46 & 44.44 & 92.07\\ 
    MCC & 86.90 & 95.87 & 90.26 & 79.28 & 42.14 & 74.74\\ 
    Kappa & 26.47 & 43.43 & 43.22 & 93.16 & 99.75 & 75.22\\  \hline
    Obs. Normal  & 2751 & 11 & 149 & 83 & 6 & \\ 
    Obs. Probe  & 88 & 2151 & 14 & 1 & 0 & \\ 
    Obs. DoS  & 127 & 17 & 2105 & 5 & 0 & \\ 
    Obs. R2L  & 17 & 0 & 0 & 211 & 0 & \\ 
    Obs. U2R  & 0 & 0 & 0 & 5 & 4 & \\  \hlineI
    \end{tabularx}
    \caption{Detailed results of the linear SVM classification algorithm for different multi-class models for $n=2000$.}
    \label{tab:svm-l-0}
\end{table}

\begin{table}[h!]
    \centering
    \begin{tabularx}{\textwidth}{lXXXXXX}
    \hlineI
    Model & Normal & Probe & DoS & R2L & U2R & Total \\ \hlineI
    \textbf{Tree 1} with $n=30,000$ & & & & & &\\
    Accuracy [\%] & 91.37 & 97.40 & 96.34 & 94.07 & 73.33 & 94.62\\ 
    MCC [\%] & 89.00 & 96.03 & 93.15 & 85.20 & 67.38 & 86.15\\
    Kappa [\%] & 27.06 & 42.34 & 42.25 & 93.59 & 99.66 & 83.19\\ \hline
    Obs. Normal  & 2741 & 63 & 136 & 55 & 5 & \\ 
    Obs. Probe  & 53 & 2195 & 2 & 3 & 0 & \\ 
    Obs. DoS  & 76 & 5 & 2171 & 1 & 0 & \\ 
    Obs. R2L  & 13 & 0 & 0 & 213 & 0 & \\ 
    Obs. U2R  & 2 & 0 & 0 & 1 & 9 & \\  \hlineI
    
    \textbf{Tree 2} with $n=30,000$ & & & & & &\\
    Accuracy [\%] & 92.56 & 97.11 & 96.18 & 93.27 & 66.67 & 95.12\\ 
    MCC & 89.84 & 96.82 & 92.39 & 87.16 & 73.46 & 89.12\\ 
    Kappa & 26.35 & 42.70 & 42.14 & 93.79 & 99.72 & 84.74\\ \hline 
    Obs. Normal  & 2777 & 29 & 151 & 42 & 2 & \\ 
    Obs. Probe  & 56 & 2189 & 9 & 0 & 0 & \\ 
    Obs. DoS  & 80 & 6 & 2168 & 0 & 0 & \\ 
    Obs. R2L  & 14 & 1 & 0 & 211 & 0 & \\ 
    Obs. U2R  & 0 & 0 & 0 & 4 & 8 & \\ \hlineI
    
    \textbf{O-A-A} with $n=30,000$ & & & & & &\\
    Accuracy [\%] & 93.03 & 96.67 & 96.53 & 94.60 & 70 & 94.62\\ 
    MCC & 90.13 & 96.84 & 92.81 & 88.31 & 77.52 & 87.93\\ 
    Kappa & 26.07 & 42.96 & 42.05 & 93.78 & 99.72 & 84.12\\ \hline
    Obs. Normal  & 2791 & 18 & 150 & 40 & 1 & \\ 
    Obs. Probe  & 71 & 2179 & 4 & 0 & 0 & \\ 
    Obs. DoS  & 70 & 8 & 2176 & 0 & 0 & \\ 
    Obs. R2L  & 12 & 0 & 0 & 214 & 0 & \\ 
    Obs. U2R  & 0 & 0 & 0 & 3 & 8 &\\ \hlineI
    \end{tabularx}
    \caption{Detailed results of the linear SVM classification algorithm for different multi-class models for $n=30,000$.}
    \label{tab:svm-l-b}
\end{table}

\begin{table}[h!]
    \centering
    \begin{tabularx}{\textwidth}{lXXXXXX}
    \hlineI
    Model & Normal & Probe & DoS & R2L & U2R & Total \\ \hlineI
    \textbf{Tree 1} $n=100,000$ & & & & & &\\
    Accuracy [\%] & 93.19 & 97.77 & 95.68 & 35.75 & 10.77 & 93.96\\ 
    MCC & 87.08 & 96.69 & 92.77 & 50.96 & 18.95 & 69.29\\ 
    Kappa & 25.33 & 41.94 & 42.19 & 96.34 & 99.79 & 80.28\\ \hline
    Obs. Normal  & 2796 & 48 & 133 & 22 & 1 & \\ 
    Obs. Probe  &48 & 2214 & 2 & 0 & 0 & \\ 
    Obs. DoS  & 91 & 7 & 2167 & 0 & 0 & \\ 
    Obs. R2L  & 123 & 0 & 0 & 69 & 1 & \\ 
    Obs. U2R  & 11 & 0 & 0 & 1 & 1 & \\  \hlineI
    
    \textbf{Tree 2} $n=100,000$ & & & & & &\\
    Accuracy [\%] & 93.51 & 97.35 & 96.74 & 80.41 & 36.92 & 95.41\\ 
    MCC & 90.34 & 97.31 & 92.79 & 82.31 & 54.36 & 87.31\\ 
    Kappa & 25.66 & 42.33 & 41.57 & 95.17 & 99.76 & 85.67\\ \hline
    Obs. Normal  & 2805 & 20 & 150 & 24 & 0 & \\ 
    Obs. Probe  & 49 & 2205 & 11 & 0 & 0 & \\ 
    Obs. DoS  & 67 & 6 & 2191 & 1 & 0 & \\ 
    Obs. R2L  & 37 & 0 & 0 & 155 & 1 & \\ 
    Obs. U2R  & 6 & 0 & 0 & 2 & 5 & \\ \hlineI
    
    \textbf{O-A-A} $n=100,000$ & & & & & &\\
    Accuracy [\%] & 93.39 & 96.97 & 97.58 & 85.49 & 61.54 & 93.96\\ 
    MCC & 90.77 & 97.50 & 93.26 & 83.32 & 71.71 & 83.42\\ 
    Kappa & 25.82 & 42.60 & 41.19 & 94.93 & 99.71 & 84.87\\ \hline
    Obs. Normal  & 2802 & 7 & 164 & 27 & 0 & \\ 
    Obs. Probe  & 67 & 2196 & 2 & 0 & 0 & \\ 
    Obs. DoS  & 45 & 4 & 2210 & 6 & 0 & \\ 
    Obs. R2L  & 27 & 0 & 0 & 165 & 1 & \\ 
    Obs. U2R  & 2 & 0 & 0 & 3 & 8 & \\ \hlineI
    \end{tabularx}
    \caption{Detailed results of the linear SVM classification algorithm for different multi-class models for $n=100,000$.}
    \label{tab:svm-l-2}
\end{table}

\FloatBarrier 
\newpage
\subsection{Linear SVM with PCA decomposition}
\label{app:lsvm-pca}

\begin{table}[h!]
    \centering
    \begin{tabularx}{\textwidth}{lXXXXXX}
    \hlineI
    Model & Normal & Probe & DoS & R2L & U2R & Total \\ \hlineI
    \textbf{Tree} with $n_{pca}=8$ & & & & & &\\
    Accuracy [\%] & 91.49 & 92.01 & 91.33 & 93.33 & 3.64 & 91.51\\ 
    MCC [\%] & 87.21 & 91.53 & 87.34 & 70.16 & $\emptyset$ & $\emptyset$\\ 
    Kappa [\%] & 26.55 & 43.84 & 43.22 & 93.69 & 99.84 & 71.11\\ \hline
    Obs. Normal & 2745 & 46 & 123 & 86 & 0 & \\ 
    Obs. Probe  &93 & 2089 & 87 & 1 & 0 & \\ 
    Obs. DoS  & 104 & 42 & 2073 & 51 & 0 & \\ 
    Obs. R2L  & 12 & 0 & 0 & 168 & 0 & \\ 
    Obs. U2R  & 5 & 0 & 0 & 6 & 0 & \\  \hlineI
    
    \textbf{O-A-A} with $n_{pca}=8$ & & & & & &\\
    Accuracy [\%] & 87.50 & 91.54 & 94.28 & 95.33 & 12.73 & 90.75 \\ 
    MCC [\%] & 85.62 & 91.02 & 88.42 & 71.14 & 4.13 & 68.07 \\ 
    Kappa [\%] & 29.07 & 43.99 & 41.76 & 93.60 & 98.90 & 71.11 \\  \hline
    Obs. Normal  & 2625 & 50 & 137 & 127 & 61 & \\ 
    Obs. Probe  & 76 & 2078 & 112 & 2 & 1 & \\ 
    Obs. DoS  & 67 & 44 & 2140 & 11 & 9 & \\ 
    Obs. R2L  & 7 & 0 & 0 & 172 & 1 & \\ 
    Obs. U2R  & 2 & 0 & 0 & 8 & 1 & \\ \hlineI
    \end{tabularx}
    \caption{Detailed results of the linear SVM classification algorithm for different multi-class models with PCA decomposition and $n_{pca}=8$ components kept. The first one is a tree-based model of order \{Normal, DoS, Prob, R2L, U2R\} and the second one a one-against-all model. Every result if the mean of 5 independent experiments.}
\end{table}

\begin{table}[h!]
    \centering
    \begin{tabularx}{\textwidth}{lXXXXXX}
    \hlineI
    Model & Normal & Probe & DoS & R2L & U2R & Total \\ \hlineI
    \textbf{Tree} with $n_{pca}=16$ & & & & & &\\
    Accuracy [\%] & 91.33 & 94.94 & 94.19 & 96.11 & 33.33 & 93.24\\ 
    MCC & 87.13 & 95.35 & 90.77 & 79.68 & $\emptyset$ & $\emptyset$\\ 
    Kappa & 26.76 & 43.51 & 42.83 & 93.25 & 99.62 & 78.86 \\  \hline
    Obs. Normal  & 2740 & 12 & 150 & 89 & 8 & \\ 
    Obs. Probe  &98 & 2142 & 16 & 1 & 0 & \\ 
    Obs. DoS  & 100 & 22 & 2125 & 8 & 1 & \\ 
    Obs. R2L  & 8 & 0 & 0 & 208 & 0 & \\ 
    Obs. U2R  & 6 & 0 & 0 & 4 & 5 & \\   \hlineI
    
    \textbf{O-A-A} with $n_{pca}=16$ & & & & & &\\
    Accuracy [\%] & 91.11 & 95.55 & 95.63 & 95.09 & 18.67 & 93.31\\ 
    MCC & 88.16 & 95.62 & 91.18 & 80.59 & 25.99 & 76.31\\ 
    Kappa & 27.08 & 43.23 & 42.09 & 93.42 & 99.74 & 80.29\\  \hline
    Obs. Normal  & 2733 & 24 & 162 & 80 & 1 & \\ 
    Obs. Probe  & 72 & 2156 & 26 & 3 & 0 & \\ 
    Obs. DoS  & 80 & 15 & 2157 & 2 & 1 & \\ 
    Obs. R2L  & 10 & 0 & 1 & 205 & 0 & \\ 
    Obs. U2R  & 6 & 0 & 0 & 6 & 3 & \\  \hlineI
    \end{tabularx}
    \caption{Detailed results of the linear SVM classification algorithm for different multi-class models with PCA decomposition and $n_{pca}=16$ components kept. The first one is a tree-based model of order \{Normal, DoS, Prob, R2L, U2R\} and the second one a one-against-all model. Every result if the mean of 5 independent experiments.}
\end{table}

\FloatBarrier 
\newpage
\subsection{Radial function based kernel support vector machines}
\label{app:rbfsvm}

\begin{table}[t]
    \centering
    \begin{tabularx}{\textwidth}{lXXXXXX}
    \hlineI
    Model & Normal & Probe & DoS & R2L & U2R & Total \\ \hlineI
    \textbf{Tree 1} with $n=15,000$ & & & & & &\\
    Accuracy [\%] & 97.82 & 99.08 & 99.01 & 91.71 & 24.00 & 98.20\\ 
    MCC [\%] & 96.62 & 99.03 & 98.54 & 87.47 & 32.25 & 82.78\\ 
    Kappa [\%] & 23.65 & 42.23 & 42.14 & 93.71 & 99.69 & 94.39\\  \hline
    Obs. Normal  & 2935 & 7 & 21 & 33 & 5 & \\ 
    Obs. Probe  & 18 & 2229 & 2 & 0 & 0 & \\ 
    Obs. DoS  & 19 & 2 & 2228 & 1 & 0 & \\ 
    Obs. R2L  & 16 & 1 & 1 & 215 & 1 & \\ 
    Obs. U2R  & 5 & 0 & 0 & 6 & 4 & \\   \hlineI
    
    \textbf{Tree 2} with $n=15,000$ & & & & & &\\
    Accuracy [\%] & 97.98 & 98.25 & 99.20 & 91.45 & 28.00 & 98.08\\ 
    MCC [\%] & 96.23 & 98.44 & 98.56 & 89.30 & $\emptyset$ & $\emptyset$\\ 
    Kappa [\%] & 23.46 & 42.56 & 42.04 & 93.85 & 99.72 & 94.00\\  \hline
    Obs. Normal  & 2939 & 8 & 24 & 26 & 2 & \\ 
    Obs. Probe  & 36 & 2211 & 4 & 0 & 0 & \\ 
    Obs. DoS  & 17 & 1 & 2232 & 0 & 0 & \\ 
    Obs. R2L  & 19 & 1 & 0 & 214 & 0 & \\ 
    Obs. U2R  & 7 & 0 & 0 & 3 & 4 & \\  \hlineI
    
    \textbf{O-A-A} with $n=15,000$ & & & & & &\\
    Accuracy [\%] & 98.33 & 99.40 & 99.15 & 92.14 & 24.00 & 98.55\\ 
    MCC [\%] & 97.28 & 99.41 & 98.84 & 88.54 & 33.55 & 83.52\\ 
    Kappa [\%] & 23.38 & 42.13 & 42.14 & 93.76 & 99.72 & 95.47\\   \hline 
    Obs. Normal  & 2950 & 4 & 15 & 30 & 1 & \\ 
    Obs. Probe  & 12 & 2237 & 1 & 0 & 0 & \\ 
    Obs. DoS  & 18 & 1 & 2231 & 0 & 0 & \\ 
    Obs. R2L  & 15 & 1 & 0 & 216 & 2 & \\ 
    Obs. U2R  & 5 & 0 & 1 & 5 & 4 & \\ \hlineI
    \end{tabularx}
    \caption{Detailed results of the RBF-SVM classification algorithm for different multi-class models for $n=15,000$. The first one is a tree-based model of order \{Normal, DoS, Prob, R2L, U2R\} and the second one a one-against-all model. Every result if the mean of 5 independent experiments.}
\end{table}

\begin{table}[t]
    \centering
    \begin{tabularx}{\textwidth}{lXXXXXX}
    \hlineI
    Model & Normal & Probe & DoS & R2L & U2R & Total \\ \hlineI
    \textbf{Tree 1} with $n=50,000$ & & & & & &\\
    Accuracy [\%] & 98.79 & 99.38 & 99.32 & 92.35 & 35.56 & 98.88\\ 
    MCC [\%] & 97.82 & 99.42 & 98.98 & 91.23 & 48.81 & 87.25\\ 
    Kappa [\%] & 23.00 & 41.75 & 41.68 & 94.69 & 99.82 & 96.46\\  \hline
    Obs. Normal  & 2964 & 3 & 16 & 16 & 1 & \\ 
    Obs. Probe  & 13 & 2248 & 1 & 0 & 0 & \\ 
    Obs. DoS  & 13 & 2 & 2247 & 0 & 0 & \\ 
    Obs. R2L  & 15 & 0 & 0 & 188 & 1 & \\ 
    Obs. U2R  & 3 & 0 & 0 & 3 & 3 & \\  \hlineI
    
    \textbf{Tree 2} with $n=50,000$ & & & & & &\\
    Accuracy [\%] & 98.93 & 99.02 & 99.27 & 91.76 & 40.00 & 98.80\\ 
    MCC [\%] & 97.57 & 99.14 & 98.99 & 92.25 & 53.45 & 88.28\\ 
    Kappa [\%] & 22.86 & 41.89 & 41.71 & 94.78 & 99.82 & 96.24\\   \hline
    Obs. Normal  & 2968 & 5 & 15 & 12 & 1 & \\ 
    Obs. Probe  & 22 & 2240 & 1 & 0 & 0 & \\ 
    Obs. DoS  & 16 & 0 & 2245 & 0 & 0 & \\ 
    Obs. R2L  & 16 & 0 & 0 & 187 & 1 & \\ 
    Obs. U2R  & 3 & 0 & 0 & 2 & 4 & \\   \hlineI
    
    \textbf{O-A-A} with $n=50,000$ & & & & & &\\
    Accuracy [\%] & 98.80 & 99.59 & 99.37 & 91.86 & 37.78 & 98.95\\ 
    MCC [\%] & 97.94 & 99.61 & 99.03 & 91.09 & 45.08 & 86.55\\ 
    Kappa [\%] & 23.02 & 41.67 & 41.66 & 94.71 & 99.81 & 96.70\\   \hline
    Obs. Normal  & 2964 & 3 & 16 & 16 & 1 & \\ 
    Obs. Probe  & 8 & 2253 & 1 & 0 & 0 & \\ 
    Obs. DoS  & 13 & 1 & 2248 & 0 & 0 & \\ 
    Obs. R2L  & 15 & 0 & 0 & 187 & 1 & \\ 
    Obs. U2R  & 3 & 0 & 0 & 2 & 3 & \\  \hlineI
    \end{tabularx}
    \caption{Detailed results of the RBF-SVM classification algorithm for different multi-class models for $n=50,000$. The first one is a tree-based model of order \{Normal, DoS, Prob, R2L, U2R\} and the second one a one-against-all model. Every result if the mean of 5 independent experiments.}
\end{table}

\FloatBarrier 
\newpage
\section{Nearest neighbours}
\label{app:knn}

\begin{table}[h!]
    \centering
    \begin{tabularx}{\textwidth}{lXXXXXXXX}
    \hlineI
    Model &&& Normal & Probe & DoS & R2L & U2R & Total \\ \hlineI
    \multicolumn{9}{l}{$k=1$ with $n=10,000$}\\
    Accuracy [\%] &&& 98.17 & 99.45 & 98.80 & 98.03 & 43.33 & 98.66\\ 
    MCC [\%] &&& 97.49 & 99.19 & 98.63 & 92.76 & 39.68 & 85.55\\ 
    Kappa [\%] &&& 23.48 & 41.80 & 42.04 & 94.07 & 99.74 & 95.81\\    \hline
    Obs. Normal  &&& 2945 & 11 & 15 & 25 & 4 & \\ 
    Obs. Probe  &&& 10 & 2246 & 2 & 1 & 0 & \\ 
    Obs. DoS  &&& 23 & 3 & 2231 & 1 & 1 & \\ 
    Obs. R2L  &&& 2 & 0 & 0 & 214 & 2 & \\ 
    Obs. U2R  &&& 3 & 0 & 0 & 2 & 4 & \\  \hlineI
    
    \multicolumn{9}{l}{$k=2$ with $n=10,000$}\\
    Accuracy [\%] &&& 98.26 & 99.37 & 98.26 & 95.93 & 30 & 98.42\\ 
    MCC [\%]  &&& 97.08 & 98.86 & 98.44 & 90.31 & 38.74 & 84.69\\ 
    Kappa [\%] &&& 23.24 & 41.49 & 42.03 & 94.82 & 99.75 & 95.07\\  \hline
    Obs. Normal  &&& 2948 & 17 & 9 & 24 & 2 & \\ 
    Obs. Probe && & 12 & 2252 & 1 & 1 & 0 & \\ 
    Obs. DoS && & 34 & 5 & 2227 & 1 & 0 & \\ 
    Obs. R2L && & 6 & 0 & 0 & 181 & 1 & \\ 
    Obs. U2R && & 3 & 0 & 0 & 5 & 4 & \\ \hlineI
    
    \multicolumn{9}{l}{$k=3$ with $n=10,000$}\\
    Accuracy [\%] &&& 97.05 & 98.90 & 98.80 & 97.81 & 55.00 & 98.08\\ 
    MCC [\%] &&& 96.36 & 98.63 & 98.31 & 89.11 & 39.73 & 84.43\\ 
    Kappa [\%] &&& 24.10 & 41.88 & 41.87 & 94.06 & 99.69 & 94.00\\    \hline 
    Obs. Normal && & 2912 & 15 & 22 & 43 & 10 & \\ 
    Obs. Probe && & 18 & 2236 & 5 & 1 & 0 & \\ 
    Obs. DoS && & 22 & 4 & 2234 & 1 & 0 & \\ 
    Obs. R2L && & 3 & 0 & 0 & 205 & 2 & \\ 
    Obs. U2R && & 3 & 0 & 0 & 1 & 4 & \\  \hlineI
    \end{tabularx}
    \caption{Detailed results of the $k$-NN classification algorithm for two different values of the number of neighbours $k$ and for a small training data-set.}
\end{table}

\begin{table}[h!]
    \centering
    \begin{tabularx}{\textwidth}{lXXXXXXXX}
    \hlineI
    Model &&& Normal & Probe & DoS & R2L & U2R & Total \\ \hlineI
    \multicolumn{9}{l}{$k=1$ with $n=100,000$}\\
    Accuracy [\%] &&& 99.56 & 99.82 & 99.63 & 93.81 & 58.33 & 99.48\\ 
    MCC [\%] &&& 99.00 & 99.73 & 99.57 & 95.39 & 62.16 & 91.17\\ 
    Kappa [\%] &&& 22.61 & 41.51 & 41.58 & 94.88 & 99.85 & 98.36\\  \hline
    Obs. Normal  &&& 2987 & 4 & 4 & 5 & 1 & \\ 
    Obs. Probe  &&& 2 & 2260 & 2 & 0 & 0 & \\ 
    Obs. DoS  &&& 8 & 0 & 2256 & 0 & 0 & \\ 
    Obs. R2L  &&& 12 & 0 & 0 & 190 & 1 & \\ 
    Obs. U2R  &&& 2 & 0 & 0 & 1 & 4 & \\   \hlineI
    
    \multicolumn{9}{l}{$k=2$ with $n=100,000$}\\
    Accuracy [\%] &&& 99.19 & 99.92 & 99.57 & 94.32 & 48.89 & 99.33\\ 
    MCC [\%]  &&& 98.64 & 99.72 & 99.35 & 95.07 & 65.47 & 91.65\\ 
    Kappa [\%] &&& 22.86 & 41.52 & 41.64 & 94.73 & 99.82 & 97.90\\   \hline
    Obs. Normal  &&& 2976 & 7 & 11 & 7 & 0 & \\ 
    Obs. Probe && & 2 & 2260 & 0 & 0 & 0 & \\ 
    Obs. DoS && & 9 & 1 & 2252 & 0 & 0 & \\ 
    Obs. R2L && & 12 & 0 & 0 & 194 & 0 & \\ 
    Obs. U2R && & 4 & 0 & 0 & 1 & 4 & \\ \hlineI
    
    \multicolumn{9}{l}{$k=3$ with $n=100,000$}\\
    Accuracy [\%] &&& 99.32 & 99.77 & 99.66 & 89.85 & 40.83 & 99.22\\ 
    MCC [\%] &&& 98.48 & 99.65 & 99.37 & 93.37 & 54.52 & 89.08\\ 
    Kappa [\%] &&& 22.71 & 41.52 & 41.52 & 95.11 & 99.77 & 97.56\\   \hline 
    Obs. Normal && & 2980 & 5 & 10 & 4 & 1 & \\ 
    Obs. Probe && & 4 & 2259 & 2 & 0 & 0 & \\ 
    Obs. DoS && & 8 & 0 & 2256 & 0 & 0 & \\ 
    Obs. R2L && & 19 & 1 & 0 & 177 & 0 & \\ 
    Obs. U2R && & 6 & 0 & 1 & 1 & 5 & \\  \hlineI
    \end{tabularx}
    \caption{Detailed results of the $k$-NN classification algorithm for two different values of the number of neighbours $k$ and a big training data-set.}
\end{table}

\FloatBarrier 
\newpage
\subsection{$k$-means}
\begin{table}[h!]
    \centering
    \begin{tabularx}{\textwidth}{lXXXXXXXX}
    \hlineI
    Model &&& Normal & Probe & DoS & R2L & U2R & Total \\ \hlineI
    \multicolumn{9}{l}{$k=1$ with $n=100,000$}\\
    Accuracy [\%] &&& 95.19 & 99.29 & 98.78 & 96.92 & 76 & 97.45\\ 
    MCC [\%] &&& 94.88 & 97.05 & 98.29 & 89.79 & 74.85 & 90.97\\ 
    Kappa [\%] &&& 25.09 & 41.02 & 41.67 & 94.85 & 99.60 & 92.02\\   \hline
    Obs. Normal  &&& 2856 & 12 & 25 & 5 & 3 & \\ 
    Obs. Probe  &&& 84 & 2251 & 2 & 0 & 0 & \\ 
    Obs. DoS  &&& 23 & 4 & 2239 & 0 & 0 & \\ 
    Obs. R2L  &&& 34 & 1 & 0 & 179 & 0 & \\ 
    Obs. U2R  &&& 4 & 0 & 0 & 0 & 11 & \\   \hlineI
    
    \multicolumn{9}{l}{$k=2$ with $n=100,000$}\\
    Accuracy [\%] &&& 95.03 & 99.26 & 98.28 & 97.82 & 55.00 & 97.27\\ 
    MCC [\%]  &&& 94.59 & 97.16 & 97.86 & 89.90 & 33.75 & 82.65\\ 
    Kappa [\%] &&& 25.26 & 41.32 & 42.08 & 93.91 & 99.82 & 91.48\\    \hline
    Obs. Normal  &&& 2851 & 13 & 34 & 4 & 1 & \\ 
    Obs. Probe && & 73 & 2243 & 5 & 0 & 0 & \\ 
    Obs. DoS && & 27 & 3 & 2221 & 0 & 0 & \\  
    Obs. R2L && & 43 & 1 & 0 & 211 & 1 & \\ 
    Obs. U2R && & 6 & 0 & 0 & 1 & 2 & \\  \hlineI
    
    \multicolumn{9}{l}{$k=3$ with $n=100,000$}\\
    Accuracy [\%] &&& 86.61 & 98.75 & 98.54 & 98.14 & 72.22 & 93.94\\
    MCC [\%] &&& 87.98 & 93.01 & 96.85 & 82.97 & 61.41 & 84.44\\ 
    Kappa [\%] &&& 29.48 & 40.32 & 41.77 & 93.22 & 99.72 & 81.06\\    \hline 
    Obs. Normal && & 2598 & 22 & 25 & 2 & 2 & \\ 
    Obs. Probe && & 246 & 2229 & 4 & 0 & 0 & \\ 
    Obs. DoS && & 65 & 5 & 2224 & 0 & 0 & \\ 
    Obs. R2L && & 87 & 1 & 4 & 216 & 1 & \\
    Obs. U2R && & 4 & 0 & 0 & 2 & 7 & \\  \hlineI
    \end{tabularx}
    \caption{Detailed results of the $k$-NN classification algorithm for two different values of the number of neighbours $k$ and $k$-means data reduction.}
\end{table}

\FloatBarrier 
\newpage
\subsection{Condensed nearest neighbors}
\begin{table}[h!]
    \centering
    \begin{tabularx}{\textwidth}{lXXXXXXXX}
    \hlineI
    Model &&& Normal & Probe & DoS & R2L & U2R & Total \\ \hlineI
    \multicolumn{9}{l}{$k=1$ with $n=10,000$}\\
    Accuracy [\%] &&& 97.63 & 99.11 & 98.26 & 98.16 & 80 & 98.25\\ 
    MCC [\%] &&& 96.73 & 99.02 & 98.02 & 90.27 & 59.45 & 88.70\\ 
    Kappa [\%] &&& 23.74 & 41.94 & 42.19 & 93.92 & 99.70 & 94.52\\    \hline
    Obs. Normal  &&& 2929 & 14 & 31 & 3 & 1 & \\ 
    Obs. Probe  &&& 9 & 2238 & 1 & 1 & 0 & \\ 
    Obs. DoS  &&& 20 & 4 & 2219 & 0 & 0 & \\ 
    Obs. R2L  &&& 36 & 1 & 4 & 213 & 1 & \\ 
    Obs. U2R  &&& 6 & 0 & 3 & 0 & 6 & \\ \hlineI
    
    \multicolumn{9}{l}{$k=2$ with $n=10,000$}\\
    Accuracy [\%] &&& 93.67 & 94.48 & 78.75 & 68.56 & 36.67 & 88.81\\ 
    MCC [\%]  &&& 85.64 & 84.77 & 82.47 & 78.84 & 36.00 & 73.54\\ 
    Kappa [\%] &&& 24.58 & 40.61 & 48.94 & 95.17 & 99.86 & 65.03\\     \hline
    Obs. Normal  &&& 2810 & 44 & 238 & 66 & 2 & \\ 
    Obs. Probe && & 172 & 2135 & 241 & 0 & 0 & \\ 
    Obs. DoS && & 6 & 81 & 1780 & 0 & 0 & \\ 
    Obs. R2L && & 10 & 0 & 1 & 147 & 2 & \\ 
    Obs. U2R && & 2 & 0 & 0 & 1 & 2 & \\   \hlineI
    
    \multicolumn{9}{l}{$k=3$ with $n=10,000$}\\
    Accuracy [\%] &&& 94.37 & 90.02 & 88.40 & 86.43 & 36.47 & 91.01\\ 
    MCC [\%] &&& 90.39 & 85.97 & 85.30 & 85.11 & 51.17 & 79.59\\ 
    Kappa [\%] &&& 25.04 & 43.77 & 44.66 & 94.26 & 99.67 & 71.91\\    \hline 
    Obs. Normal &&& 2831 & 41 & 106 & 29 & 10 & \\ 
    Obs. Probe && & 101 & 2029 & 155 & 0 & 0 & \\ 
    Obs. DoS &&& 35 & 182 & 1993 & 0 & 0 & \\ 
    Obs. R2L &&& 32 & 2 & 1 & 191 & 0 & \\ 
    Obs. U2R &&& 1 & 0 & 0 & 1 & 6 & \\   \hlineI
    \end{tabularx}
    \caption{Detailed results of the $k$-NN classification algorithm for $k=1,2,3$ with condensed neighbours reduction.}
\end{table}

\FloatBarrier 
\newpage
\subsection{CNN with PCA}
\label{app:knn-cnn-pca}
\begin{table}[h!]
    \centering
    \begin{tabularx}{\textwidth}{lXXXXXXXX}
    \hlineI
    Model &&& Normal & Probe & DoS & R2L & U2R & Total \\ \hlineI
    \multicolumn{9}{l}{$k=1$ with $n=10,000$ and $n_{pca}=8$}\\
    Accuracy [\%] &&& 97.27 & 98.81 & 98.18 & 95 & 51.67 & 97.85\\  
    MCC [\%] &&& 96.15 & 98.25 & 97.79 & 89.08 & 43.98 & 85.05\\  
    Kappa [\%] &&& 23.90 & 41.87 & 42.13 & 94.29 & 99.63 & 93.29\\     \hline
    Obs. Normal  &&& 2918 & 16 & 32 & 7 & 5 & \\  
    Obs. Probe  &&& 20 & 2233 & 9 & 0 & 0 & \\ 
    Obs. DoS  &&& 21 & 8 & 2219 & 0 & 0 & \\ 
    Obs. R2L  &&& 34 & 2 & 0 & 198 & 1 & \\ 
    Obs. U2R  &&& 7 & 0 & 0 & 4 & 6 & \\ \hlineI
    
    \multicolumn{9}{l}{$k=2$ with $n=10,000$ and $n_{pca}=8$}\\
    Accuracy [\%] &&& 92.44 & 99.45 & 66.28 & 68.14 & 4 & 86.03\\  
    MCC [\%]  &&& 84.16 & 82.18 & 75.89 & 74.38 & $\emptyset$ & $\emptyset$\\ 
    Kappa [\%] &&& 25.45 & 36.97 & 53.77 & 94.51 & 99.85 & 56.35\\    \hline
    Obs. Normal  &&& 2773 & 9 & 275 & 72 & 6 & \\  
    Obs. Probe && & 188 & 2241 & 482 & 1 & 0 & \\ 
    Obs. DoS && & 6 & 3 & 1493 & 0 & 0 & \\ 
    Obs. R2L && & 33 & 1 & 2 & 157 & 4 & \\  
    Obs. U2R && & 1 & 0 & 0 & 1 & 0 & \\    \hlineI
    
    \multicolumn{9}{l}{$k=3$ with $n=10,000$ and $n_{pca}=8$}\\
    Accuracy [\%] &&& 95.82 & 89.26 & 81.71 & 86.70 & 16 & 89.43\\ 
    MCC [\%] &&& 91.91 & 81.79 & 79.87 & 81.48 & 19.28 & 70.87\\ 
    Kappa [\%] &&& 24.22 & 43.09 & 46.82 & 94.36 & 99.79 & 66.98\\     \hline 
    Obs. Normal &&& 2875 & 59 & 82 & 26 & 6 & \\ 
    Obs. Probe && & 2875 & 59 & 82 & 26 & 6 & \\ 
    Obs. DoS &&& 35 & 181 & 1847 & 0 & 0 & \\ 
    Obs. R2L &&& 49 & 2 & 1 & 179 & 2 & \\  
    Obs. U2R &&& 4 & 0 & 0 & 1 & 2 & \\    \hlineI
    \end{tabularx}
    \caption{Detailed results of the $k$-NN classification algorithm for $k=1,2,3$ with condensed neighbours reduction and PCA decomposition ($n_{pca}=8$).}
\end{table}

\begin{table}[h!]
    \centering
    \begin{tabularx}{\textwidth}{lXXXXXXXX}
    \hlineI
    Model &&& Normal & Probe & DoS & R2L & U2R & Total \\ \hlineI
    \multicolumn{9}{l}{$k=1$ with $n=10,000$ and $n_{pca}=16$}\\
    Accuracy [\%] &&& 97.16 & 98.68 & 98.61 & 96.78 & 86 & 98.00\\  
    MCC [\%] &&& 96.30 & 98.33 & 98.22 & 89.69 & 55.24 & 87.56\\ 
    Kappa [\%] &&& 24.00 & 41.91 & 41.92 & 94.30 & 99.56 & 93.77\\     \hline
    Obs. Normal  &&& 2915 & 22 & 24 & 5 & 0 & \\ 
    Obs. Probe  &&& 17 & 2232 & 7 & 0 & 0 & \\ 
    Obs. DoS  &&& 19 & 6 & 2231 & 0 & 0 & \\ 
    Obs. R2L  &&& 35 & 2 & 1 & 198 & 1 & \\ 
    Obs. U2R  &&& 14 & 0 & 0 & 1 & 9 & \\  \hlineI
    
    \multicolumn{9}{l}{$k=2$ with $n=10,000$ and $n_{pca}=16$}\\
    Accuracy [\%] &&& 90.79 & 96.59 & 71.94 & 75.38 & 17.50 & 86.50\\
    MCC [\%]  &&& 85.92 & 80.50 & 77.43 & 78.06 & $\emptyset$ & $\emptyset$\\ 
    Kappa [\%] &&& 26.70 & 37.87 & 50.78 & 94.95 & 99.86 & 57.81\\    \hline
    Obs. Normal  &&& 2724 & 16 & 179 & 48 & 4 & \\ 
    Obs. Probe && & 202 & 2187 & 453 & 0 & 0 & \\  
    Obs. DoS && & 37 & 59 & 1629 & 0 & 0 & \\ 
    Obs. R2L && & 36 & 3 & 2 & 150 & 2 & \\  
    Obs. U2R && & 1 & 0 & 0 & 0 & 1 & \\    \hlineI
    
    \multicolumn{9}{l}{$k=3$ with $n=10,000$ and $n_{pca}=16$}\\
    Accuracy [\%] &&& 92.27 & 92.52 & 94.92 & 83.77 & 12.73 & 92.79\\
    MCC [\%] &&& 90.67 & 88.72 & 90.60 & 80.12 & 15.14 & 73.05\\ 
    Kappa [\%] &&& 26.51 & 42.91 & 42.02 & 94.88 & 99.74 & 77.48\\  \hline 
    Obs. Normal &&& 2768 & 24 & 51 & 30 & 6 & \\  
    Obs. Probe && & 140 & 2096 & 56 & 0 & 1 & \\ 
    Obs. DoS &&& 51 & 143 & 2151 & 0 & 0 & \\ 
    Obs. R2L &&& 35 & 2 & 7 & 160 & 3 & \\ 
    Obs. U2R &&& 6 & 0 & 1 & 1 & 1 & \\    \hlineI
    \end{tabularx}
    \caption{Detailed results of the $k$-NN classification algorithm for $k=1,2,3$ with condensed neighbours reduction and PCA decomposition ($n_{pca}=16$).}
\end{table}

\FloatBarrier 
\newpage
\subsection{CNN with $\chi^2$ feature selection}
\label{app:knn-cnn-chi2}

\begin{table}[h!]
    \centering
    \begin{tabularx}{\textwidth}{lXXXXXXXX}
    \hlineI
    Model &&& Normal & Probe & DoS & R2L & U2R & Total \\ \hlineI
    \multicolumn{9}{l}{$k=1$ with $n=10,000$, $\chi^2$ and CNN.}\\
    Accuracy [\%] &&& 97.77 & 99.16 & 98.85 & 97.60 & 88.89 & 98.48\\ 
    MCC [\%] &&& 96.98 & 99.03 & 98.91 & 92.61 & 45.63 & 86.63\\ 
    Kappa [\%] &&& 23.66 & 41.82 & 41.97 & 94.35 & 99.44 & 95.24\\       \hline
    Obs. Normal  &&& 2933 & 15 & 24 & 5 & 0 & \\ 
    Obs. Probe  &&& 12 & 2242 & 0 & 0 & 0 & \\ 
    Obs. DoS  &&& 7 & 2 & 2235 & 0 & 0 & \\
    Obs. R2L  &&& 22 & 2 & 2 & 203 & 1 & \\ 
    Obs. U2R  &&& 26 & 0 & 0 & 0 & 8 & \\    \hlineI
    
    \multicolumn{9}{l}{$k=2$ with $n=10,000$, $\chi^2$ and CNN.}\\
    Accuracy [\%] &&& 96.80 & 99.69 & 58.14 & 79.68 & 7.69 & 85.75\\ 
    MCC [\%] &&& 89.53 & 79.61 & 70.04 & 84.74 & 19.56 & 68.70\\ 
    Kappa [\%] &&& 23.02 & 35.61 & 56.12 & 95.48 & 99.81 & 55.47\\      \hline
    Obs. Normal  &&& 2904 & 2 & 251 & 38 & 7 & \\ 
    Obs. Probe  &&& 82 & 2260 & 698 & 0 & 0 & \\ 
    Obs. DoS  &&& 3 & 5 & 1318 & 0 & 0 & \\ 
    Obs. R2L  &&& 10 & 0 & 0 & 149 & 5 & \\ 
    Obs. U2R  &&& 1 & 0 & 0 & 0 & 1 & \\   \hlineI
    
    \multicolumn{9}{l}{$k=3$ with $n=10,000$, $\chi^2$ and CNN.}\\
    Accuracy [\%] &&& 94.83 & 84.84 & 96.02 & 86.63 & 54.55 & 91.99\\ 
    MCC [\%] &&& 93.37 & 86.87 & 86.26 & 77.55 & 46.62 & 78.13\\ 
    Kappa [\%] &&& 25.22 & 46.90 & 40.33 & 94.22 & 99.66 & 74.96\\   \hline
    Obs. Normal  &&& 2845 & 9 & 50 & 27 & 2 & \\ 
    Obs. Probe  &&& 37 & 1919 & 36 & 0 & 0 & \\ 
    Obs. DoS  &&& 53 & 324 & 2172 & 0 & 0 & \\ 
    Obs. R2L  &&& 56 & 10 & 4 & 175 & 3 & \\ 
    Obs. U2R  &&& 9 & 0 & 0 & 0 & 6 & \\    \hlineI
    \end{tabularx}
    \caption{Detailed results of the $k$-NN classification algorithm with condensed neighbours reduction and $\chi^2$ (33) feature selection.}
\end{table}

\begin{table}[h!]
    \centering
    \begin{tabularx}{\textwidth}{lXXXXXXXX}
    \hlineI
    Model &&& Normal & Probe & DoS & R2L & U2R & Total \\ \hlineI
    \multicolumn{9}{l}{$k=1$ with $n=10,000$, $\chi^2$ and CNN.}\\
    Accuracy [\%] &&& 91.23 & 93.80 & 93.66 & 90.54 & 87.50 & 97.87\\ 
    MCC [\%] &&& 88.48 & 93.96 & 92.37 & 67.63 & 27.50 & 83.44\\ 
    Kappa [\%] &&& 27.06 & 43.81 & 43.51 & 92.25 & 98.86 & 93.34\\      \hline
    Obs. Normal  &&& 2737 & 55 & 91 & 12 & 1 & \\ 
    Obs. Probe  &&& 45 & 2117 & 7 & 0 & 0 & \\ 
    Obs. DoS  &&& 68 & 32 & 2114 & 0 & 0 & \\ 
    Obs. R2L  &&& 106 & 44 & 34 & 201 & 0 & \\ 
    Obs. U2R  &&& 44 & 9 & 11 & 9 & 7 & \\     \hlineI
    
    \multicolumn{9}{l}{$k=2$ with $n=10,000$, $\chi^2$ and CNN.}\\
    Accuracy [\%] &&& 66.67 & 98.58 & 65.00 & 53.68 & 66.67 & 88.21\\ 
    MCC [\%] &&& 53.49 & 74.83 & 73.74 & 63.41 & 10.96 & 72.02\\ 
    Kappa [\%] &&& 37.25 & 35.09 & 54.09 & 94.98 & 97.21 & 63.14\\     \hline
    Obs. Normal  &&& 2000 & 17 & 613 & 54 & 0 & \\ 
    Obs. Probe  &&& 841 & 2222 & 118 & 3 & 0 & \\ 
    Obs. DoS  &&& 23 & 6 & 1465 & 12 & 0 & \\ 
    Obs. R2L  &&& 21 & 0 & 14 & 124 & 2 & \\ 
    Obs. U2R  &&& 115 & 9 & 44 & 38 & 4 & \\   \hlineI
    
    \multicolumn{9}{l}{$k=3$ with $n=10,000$, $\chi^2$ and CNN.}\\
    Accuracy [\%] &&& 91.27 & 4.60 & 85.09 & 80.09 & 0 & 76.65\\ 
    MCC [\%] &&& 75.19 & 10.80 & 47.24 & 62.94 & 0.35 & 60.45\\ 
    Kappa [\%] &&& 24.62 & 69.50 & 34.40 & 93.11 & 98.92 & 27.07\\    \hline
    Obs. Normal  &&& 2738 & 412 & 249 & 30 & 1 & \\ 
    Obs. Probe  &&& 58 & 104 & 5 & 0 & 0 & \\ 
    Obs. DoS  &&& 94 & 1722 & 1923 & 2 & 0 & \\ 
    Obs. R2L  &&& 70 & 17 & 64 & 169 & 9 & \\ 
    Obs. U2R  &&& 40 & 5 & 19 & 10 & 0 & \\    \hlineI
    \end{tabularx}
    \caption{Detailed results of the $k$-NN classification algorithm with condensed neighbours reduction and $\chi^2$ (20) feature selection.}
\end{table}
\chapter{The Quickselect algorithm}
\label{app:qs}

The \emph{quickselect} algorithm uses an auxiliary subroutine that reorganizes a given partition --- specified by two indices --- into elements smaller than a given pivot point left of it and bigger, right of it. The main algorithm then makes recursive calls to that same subroutine until the pivot ends at the $k$\textsuperscript{th} position. The algorithm and its subroutine are given at \ref{alg:knn-qs-1} and \ref{alg:knn-qs-2}.

\begin{center}
\begin{algorithm}[H]
 \KwData{list $\left\{y_i\right\}$, partition start index $i_{a}$, partition end index $i_{b}$, pivot index $i_{p}$}
 \KwResult{new pivot index $j_p$, sorted list $\left\{y_j\right\}$ of $\left\{y_{i_{a}} , \ldots , y_{i_{b}}\right\}$ where $\forall j < j_{p} : y_j < y_{i_{p}}$, $\forall j > j_{p} : y_j > y_{i_{p}}$ and $y_{j_{p}} = y_{i_{p}}$}
 \DontPrintSemicolon
  \SetKwFunction{FPartition}{partition}
  \SetKwProg{Fn}{function}{:}{}
  \Fn{\FPartition{$\left\{y_i\right\}$, $i_a$, $i_b$, $i_p$}}{
 $p \leftarrow y_{i_{p}}$\;
 $l \leftarrow a$\;
 \For{ $i = i_a$ \KwTo $i_b-1$}{
  \If{$y_i < p$}{
   swap $y_i$ and $y_k$\;
   $k \leftarrow k+1$\;
   }
   swap $y_b$ and $y_k$
 }
 $j_p \leftarrow l$\;
 \Return{$j_p$, $\left\{y_j\right\}$}}
 \caption{Quickselect's partition subroutine}
 \label{alg:knn-qs-2}
\end{algorithm}
\end{center}

\begin{center}
\begin{algorithm}[H]
 \KwData{list $\left\{y_i\right\}$, partition start index $i_{a}$, partition end index $i_{b}$, number of smallest elements wanted $k$}
 \KwResult{new list $\left\{y_j\right\}$ containing $k$ smallest elements}
 \DontPrintSemicolon
  \SetKwFunction{FSelect}{select}
  \SetKwFunction{FPartition}{partition}
  \SetKwProg{Fn}{function}{:}{}
  \Fn{\FSelect{$\left\{y_i\right\}$, $i_a$, $i_b$, $k$}}{
\eIf{$i_a = i_b$}{
\Return $\left\{y_1 , \ldots , y_{i_a}\right\}$}{
$l, \left\{y_j\right\} \leftarrow$ \FPartition{$\left\{y_i\right\}$, $i_a$, $i_b$, $l$}
\tcp*[r]{assign random $l$ if not already assigned} \;
\uIf{$k=l$}{
\Return{$\left\{y_1 , \ldots , y_k\right\}$}}
\uElseIf{$k<l$}{
\Return{\FSelect{$\left\{y_i\right\}$, $i_a$, $l+1$, $k$}}}
\uElse{\Return{\FSelect{$\left\{y_i\right\}$, $l+1$, $i_b$, $k$}}}
}
}
 \caption{Quickselect's main routine}
 \label{alg:knn-sq-2}
\end{algorithm}

This algorithm has been implemented but the compilation results in an infinite loop. As the MAMBA language first develops all the python code, it develops the function containing the recursive call the that same function and adds it to the SCALE code. It does so with the new call and adds the new recursive call to the SCALE code. This goes on forever. Classical loops are also fully developed. The only way to really execute a loop is to use certain specific calls to specially developed routines that allow real loops on the SCALE code. Unfortunately, no such routines exist for recusrive calls.

\end{center}

\backmatter
%\bibliographystyle{ieeetr}
%\bibliography{misc/references.bib}
\printbibliography

\end{document}
